
% Appendix chapter ----------------------------------------------------
\chapter*{Appendix}
\addcontentsline{toc}{chapter}{Appendix}
\label{ch:Appendix}

\section*{Odd variants}
\addcontentsline{toc}{section}{Odd variants}
\label{sec:Appendix/Odd variants}

Odd variants ...

Note that in Odd Classical Game, since it's played on 7 x 7 board,
there is no en-passant move. This is so because of very small board
there is no room for a Pawn to perform 2-field initial move without,
at the same time, preventing opponent to do the same at the same file.

\section*{Sides of chess board}
\addcontentsline{toc}{section}{Sides of chess board}
\label{sec:Appendix/Sides of chess board}

Sides of chess board ... light, dark, left, right.

\section*{Initial setups}
\addcontentsline{toc}{section}{Initial setups}
\label{sec:Appendix/Initial setups}
Initial setups ...

% Initial setup for Light player is (mirrored for Dark one):
% \texttt{PPPPPPPPPP \\
%         RGNBQKBNGR}, \\
% or more conveniently, as seen in this image:

% Initial setup for Light player is (mirrored for Dark one):
% \texttt{PPPPPPPPPPPP \\
%         RGANBQKBNAGR}, \\
% or more conveniently, as seen in this image:

% Initial setup for Light player is (mirrored for Dark one):
% \texttt{PPPPPPPPPPPPPP \\
%         RGAUNBQKBNUAGR}, \\
% or more conveniently, as seen in this image:

% Initial setup for Light player is (mirrored for Dark one):
% \texttt{PPPPPPPPPPPPPPPP \\
%         RGAUWNBQKBNWUAGR}, \\
% or more conveniently, as seen in this image:

% Initial setup for Light player is (mirrored for Dark one):
% \texttt{PPPPPPPPPPPPPPPPPP \\
%         TRGAUWNBQKBNWUAGRT}, \\
% or more conveniently, as seen in this image:

% Initial setup for Light player is (mirrored for Dark one):
% \texttt{PPPPPPPPPPPPPPPPPPPP \\
%         TRGAUWCNBQKBNCWUAGRT}, \\
% or more conveniently, as seen in this image:

% Initial setup for Light player is (mirrored for Dark one):
% \texttt{PPPPPPPPPPPPPPPPPPPPPP \\
%         TRGAUWCSNBQKBNSCWUAGRT}, \\
% or more conveniently, as seen in this image:

% Initial setup for Light player is (mirrored for Dark one):
% \texttt{PPPPPPPPPPPPPPPPPPPPPPPP \\
%         TRGAHUWCSNBQKBNSCWUHAGRT}, \\
% or more conveniently, as seen in this image:

% Initial setup for Light player is (mirrored for Dark one):
% \texttt{PPPPPPPPPPPPPPPPPPPPPPPP \\
%         TRGAHUWCSNBQKBNSCWUHAGRT}, \\
% or more conveniently, as seen in this image:

% Initial setup for Light player is (mirrored for Dark one):
% \texttt{PPPPPPPPPPPPPPPPPPPPPPPPPP \\
%         TRGAHIUWCSNBQKBNSCWUIHAGRT}, \\
% or more conveniently, as seen in this image:

% TODO :: add to "Pieces" section
%
% \section*{Passive pieces}
% \addcontentsline{toc}{section}{Passive pieces}
% \label{sec:Appendix/Passive pieces}
%
% Passive pieces are ...
%
% Activating passive piece with Pawn ... capture-fields vs. step-fields.

\section*{Movement of Wave}
\addcontentsline{toc}{section}{Movement of Wave}
\label{sec:Appendix/Movement of Wave}

Movement of Wave ... as multi-step piece activating it  ...

% TODO :: table <activated-by> : <moves-like>

\section*{Monolith initial positions}
\addcontentsline{toc}{section}{Monolith initial positions}
\label{sec:Appendix/Monolith initial positions}

Monolith initial positions ...

\section*{Royal powers}
\addcontentsline{toc}{section}{Royal powers}
\label{sec:Appendix/Royal powers}

Royal powers ...

    Refers to unique attributes of the king: being subject to check and checkmate, the inability to be captured, and the ability to castle.
    Protection from effects/actions induced/performed by non-active pieces, i.e. Pyramid, Wave, Star, Monolith.

% https://en.wikipedia.org/wiki/Glossary_of_chess#Royal_powers

\section*{Promotions}
\addcontentsline{toc}{section}{Promotions}
\label{sec:Appendix/Promotions}

Pawn can be promoted to any piece except Pawn, King, Star or Monolith.
Pawn can only be promoted to a piece of the same color.

Promotions are forced, i.e. Pawn has to be promoted immediately, in the following
variants: Classical Chess, Croatian Ties, Mayan Ascendancy and Conquest of Tlalocan.
Forced promotion means that Pawn has to be promoted in the same move in which it
reached opposite end of chessboard. If it was promoted by Pyramid, it has to be
promoted in the very same ply in which it was reached by that Pyramid.

Promotions are not forced in all the other variants. Additionaly, Pawn can be promoted
at some point later in game. Promotion in that case is whole move, i.e. move in which
only promotion is made. During that time (between been tagged for promotion and actual
promotion itself), Pawn must not move, i.e. it has to be actually promoted in the same
field it was tagged for promotion. If tagged Pawn moves before it gets promoted, it
looses its' tag, i.e. can no longer be promoted.

Pawn can be promoted to Queen if and only if existing Queen has been captured, in the
following variants: Nineteen, Tamoanchan Revisited and One. In these variants only one
Queen, in the same color, is ever allowed to be present on chessboard. In all the other
variants, each side can have multiple Queens present on chessboard at the same time.

\clearpage % ..........................................................
% ---------------------------------------------------- Appendix chapter
