
% Remarks chapter -----------------------------------------------------
\chapter*{Remarks}
\addcontentsline{toc}{chapter}{Remarks}

\section*{Converting opponent's Rook}
\addcontentsline{toc}{section}{Converting opponent's Rook in own Rook initial position}

Converting opponent's Rook in own Rook initial position ... regarding castling.

\section*{Converting opponent's Pawn}
\addcontentsline{toc}{section}{Converting opponent's Pawn in own Pawn initial position}

Converting opponent's Pawn in own Pawn initial position ... regarding en-passant, promotion.

\section*{Activating Pawn, en passant}
\addcontentsline{toc}{section}{Activating Pawn in its' initial position}

Activating Pawn in its' initial position ... en passant.

\section*{Well defined game}
\addcontentsline{toc}{section}{Well defined game}

Well defined game is such where all information pertainable to a game
is plainly visible on a board. Chess in its' origin is very close to
that goal, with the exceptions being castling, and turn.

\subsection*{Chips}
\addcontentsline{toc}{subsection}{Chips}
Chips ...
Coins could be used insead of chips.

\subsection*{Turn-chip}
\addcontentsline{toc}{subsection}{Turn-chip}
Turn-chip, also Zed, in algebraic notation Z, is a single chip used to
denote which player's turn is ongoing. It's used in positional notation,
where color of, otherwise empty, field occupied by Zed denotes which
player "has the move".

\section*{Non-movement rules}
\addcontentsline{toc}{section}{Non-movement rules}

50-move rule, ...

Destination filed == starting field, ...

...

\clearpage % ..........................................................
% ----------------------------------------------------- Remarks chapter
