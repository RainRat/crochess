
% Copyright (c) 2015 - 2021 Mario Mlačak, mmlacak@gmail.com
% Licensed and published as Public Domain work.

% Terms chapter -------------------------------------------------------
\chapter*{Terms}
\addcontentsline{toc}{chapter}{Terms}
\label{ch:Terms}

This chapter defines some terms as used in this book.

\section*{Turn}
\addcontentsline{toc}{section}{Turn}
\label{sec:Terms/Turn}
Turn denotes player who hasn't finished his (or her) move, i.e. the one who "has the move",
the one who "is on the move",
\hyperref[tbl:Appendix/Introduction/Abbreviations]{see \algfmt{FIDE 1.3}}.

\section*{Move}
\addcontentsline{toc}{section}{Move}
\label{sec:Terms/Move}
Move is completed movement of all pieces, performed sequentialy, by one player, in one turn.

\section*{Cycle}
\addcontentsline{toc}{section}{Cycle}
\label{sec:Terms/Cycle}
Cycle consists of light player's move, followed by dark player's move.

\section*{Game score}
\addcontentsline{toc}{section}{Game score}
\label{sec:Terms/Game score}
Game score is a numbered list of cycles, in order in which they were played during a game.

\section*{Momentum}
\addcontentsline{toc}{section}{Momentum}
\label{sec:Terms/Momentum}
Momentum is count of fields traveled over by a piece.

\section*{Cascade}
\addcontentsline{toc}{section}{Cascade}
\label{sec:Terms/Cascade}
Cascade is a move where at least 2 pieces have moved.

\section*{Ply}
\addcontentsline{toc}{section}{Ply}
\label{sec:Terms/Ply}
Ply is completed movement of a piece, from its starting position to its destination field.

\section*{Activation}
\addcontentsline{toc}{section}{Activation}
\label{sec:Terms/Activation}
Activation is a ply in which a piece captures field previously occupied by another piece,
and transfer all of its momentum to activated piece.

\section*{Push-pull activation}
\addcontentsline{toc}{section}{Push-pull activation}
\label{sec:Terms/Push-pull activation}
Activation of a piece which in the same move started a cascade.

\section*{Step-fields}
\addcontentsline{toc}{section}{Step-fields}
\label{sec:Terms/Step-fields}
Step-fields are all fields where a piece can end it's movement.

\section*{Step}
\addcontentsline{toc}{section}{Step}
\label{sec:Terms/Step}
Step is a movement of a piece from one step-field to next.

\section*{Rush}
\addcontentsline{toc}{section}{Rush}
\label{sec:Terms/Rush}
Rush is Pawn's longer initial movement, i.e. from its starting position, for at least 2 fields forward.
Rushing Pawn presents opponent with en passant opportunity.

\section*{Tag}
\addcontentsline{toc}{section}{Tag}
\label{sec:Terms/Tag}
Tag is a delayed opportunity link between a piece and a field at which it stands.
Only one tag at any given time can be applied to a piece.

Piece can be tagged for castling, promotion or rushing; doing any of these things consumes tag, and cannot be repeated again.
For instance, Pawn can be rushed for less than maximum allowed in a variant; regardless, rushed Pawn cannot be rushed again.

Tag, and opportunity it represents, is definitely lost when tagged piece is moved, captured, converted, activated or displaced.

Initially, all Pawns are tagged for rushing, and all Rooks and Kings are tagged for castling.
Later in game, Pawns can be tagged for promotion.

As a special case, Serpent can be tagged for Pawn-sacrifice; this tag is not delayed, so it has to be used in the same move in which it has been obtained.

\section*{Capture-fields}
\addcontentsline{toc}{section}{Capture-fields}
\label{sec:Terms/Capture-fields}
Capture-fields are all fields where a piece can capture opponent's piece.
Usually, these are the same as step-fields, except for Pawn and Shaman.

\section*{Neighboring-fields}
\addcontentsline{toc}{section}{Neighboring-fields}
\label{sec:Terms/Neighboring-fields}
Neighboring-fields are all fields immediately surrounding a particular field horizontally,
vertically and diagonally.

% \section*{Royal-fields}
% \addcontentsline{toc}{section}{Royal-fields}
% \label{sec:Terms/Royal-fields}
% Royal-fields are all fields surrounding a particular field horizontally,
% vertically and diagonally, i.e. they are step-fields of a King.

\section*{Portal-fields}
\addcontentsline{toc}{section}{Portal-fields}
\label{sec:Terms/Portal-fields}
Portal-fields are all fields immediately surrounding a particular field horizontally,
vertically and diagonally.

These are used in teleportation context, i.e. in a cascade involving a Star or Monolith.

\section*{Displacement-fields}
\addcontentsline{toc}{section}{Displacement-fields}
\label{sec:Terms/Displacement-fields}
Displacement-fields are all fields where a piece can be moved to directly.
Displacement is not affected by how piece normally moves.

These are used in trance-journey context, i.e. in a cascade involving entranced light Shaman.

\section*{Oblation}
\addcontentsline{toc}{section}{Oblation}
\label{sec:Terms/Oblation}
Oblation is removal of a piece from chessboard by rules or circumstances,
without being captured by opponent.

\section*{Chip}
\addcontentsline{toc}{section}{Chip}
\label{sec:Terms/Chip}
Chip is a device not playable by either player, used to denote various states
of a game. For instance, it could be used to denote King's ability to castle,
which Pawn is tagged to be promoted, or which player's turn is ongoing.

\section*{Piece}
\addcontentsline{toc}{section}{Piece}
\label{sec:Terms/Piece}
Piece is an item on chessboard playable by players or a Star, i.e. piece is
any item except chips.

\section*{Passive piece}
\addcontentsline{toc}{section}{Passive piece}
\label{sec:Terms/Passive piece}
Passive piece is any which needs to be activated, before it can move.
These are Pyramid and Wave.

\section*{Figure}
\addcontentsline{toc}{section}{Figure}
\label{sec:Terms/Figure}
Figure is any piece, except Pawn.

\section*{Pawn row}
\addcontentsline{toc}{section}{Pawn row}
\label{sec:Terms/Pawn row}
Pawn row is any row which contains Pawns in its full length on initial setup
of chessboard.

In early variants (up to Nineteen), for light player that is second row, for
dark player second to last row. In Nineteen and later variants an additional
rank of Pawns was added, and so Pawns rows are second and third for light
player, second to last and third to last for dark player.

Note that scout Pawns do not fill up row completely, and so these Pawns are
not located at Pawns rows.

\section*{Figure row}
\addcontentsline{toc}{section}{Figure row}
\label{sec:Terms/Figure row}
Figure row is row that contains figures on initial setup of chessboard.
For light player that is first row, for dark player it is last row.

\section*{Piece row}
\addcontentsline{toc}{section}{Piece row}
\label{sec:Terms/Piece row}
Piece row is either Pawn row or figure row.

\clearpage % ..........................................................
% ------------------------------------------------------- Terms chapter
