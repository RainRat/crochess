
% Copyright (c) 2015 - 2021 Mario Mlačak, mmlacak@gmail.com
% Published as Public Domain work, under CC0 1.0 Universal Public Domain Dedication. See LICENSING, COPYING files for details.

% Terms chapter -------------------------------------------------------
\chapter*{Terms}
\addcontentsline{toc}{chapter}{Terms}
\label{ch:Terms}

This chapter defines some terms as used in this book.

\section*{Turn}
\addcontentsline{toc}{section}{Turn}
\label{sec:Terms/Turn}
Turn denotes player who hasn't finished his (or her) move, i.e. the one who
"has the move", the one who "is on the move",
\hyperref[tbl:Appendix/Introduction/Abbreviations]{see \algfmt{FIDE 1.3}}.

\section*{Chip}
\addcontentsline{toc}{section}{Chip}
\label{sec:Terms/Chip}
Chip is any item on a chessboard not movable by any player, used to denote
various states of a game. For instance, it could be used to denote King's
ability to castle, which Pawn is tagged for promotion, or which player's
turn is ongoing.

\section*{Piece}
\addcontentsline{toc}{section}{Piece}
\label{sec:Terms/Piece}
Piece is any item on a chessboard movable by one, or both players, i.e. piece
is any item except chips.

In later variants, not every piece is owned by a player. Pieces without owner
are Stars, and Monoliths.

\section*{Material}
\addcontentsline{toc}{section}{Material}
\label{sec:Terms/Material}

Material is any piece, except Wave.

\section*{Materiel}
\addcontentsline{toc}{section}{Materiel}
\label{sec:Terms/Materiel}

Materiel is any own piece which can capture opponent's pieces.

Materiel is any piece except Wave, Starchild (those cannot capture),
and Star, Monolith (those are not owned, and cannot capture).

\section*{Trooper}
\addcontentsline{toc}{section}{Trooper}
\label{sec:Terms/Trooper}
Trooper is either a Scout, or a Grenadier.

\section*{Private}
\addcontentsline{toc}{section}{Private}
\label{sec:Terms/Private}
Private is one of Pawn, Scout, or Grenadier; i.e. private is either a Pawn,
or a trooper.

\section*{Figure}
\addcontentsline{toc}{section}{Figure}
\label{sec:Terms/Figure}
Figure is any piece, except Pawn.

\section*{Move}
\addcontentsline{toc}{section}{Move}
\label{sec:Terms/Move}
Move is completed movement of chosen and all affected pieces, performed
sequentialy, by one player, in one turn.

\section*{Cycle}
\addcontentsline{toc}{section}{Cycle}
\label{sec:Terms/Cycle}
Cycle consists of light player's move, followed by dark player's move.

\section*{Game score}
\addcontentsline{toc}{section}{Game score}
\label{sec:Terms/Game score}
Game score is a numbered list of cycles, in order in which they were played
during a game.

\section*{Momentum}
\addcontentsline{toc}{section}{Momentum}
\label{sec:Terms/Momentum}
Momentum is count of fields traveled over by a piece.

\section*{Cascade}
\addcontentsline{toc}{section}{Cascade}
\label{sec:Terms/Cascade}
Cascade is a move where at least 2 pieces have moved.

\section*{Ply}
\addcontentsline{toc}{section}{Ply}
\label{sec:Terms/Ply}
Ply is completed movement of a piece, from its starting position to its destination
field.

\section*{Path}
\addcontentsline{toc}{section}{Path}
\label{sec:Terms/Path}
Path is list of all fields traveled in a single ply, and in the same order.

\section*{Oblation}
\addcontentsline{toc}{section}{Oblation}
\label{sec:Terms/Oblation}
Oblation is removal of a piece from chessboard by rules or circumstances,
without being captured by opponent.

\section*{Activation}
\addcontentsline{toc}{section}{Activation}
\label{sec:Terms/Activation}
Activation is act of capturing field at which piece stands, without capturing that
piece itself. Activating piece transfers all of its momentum to activated piece.
Activated piece then has to move to some other field, or it's oblationed.

\section*{Passive piece}
\addcontentsline{toc}{section}{Passive piece}
\label{sec:Terms/Passive piece}
Passive piece is any which needs to be activated, before it can move.
These are Pyramid, Wave, and Star.

\section*{Push-pull activation}
\addcontentsline{toc}{section}{Push-pull activation}
\label{sec:Terms/Push-pull activation}
Activation of a piece which in the same move started a cascade.

\section*{Step-fields}
\addcontentsline{toc}{section}{Step-fields}
\label{sec:Terms/Step-fields}
Step-fields are all fields where a piece can end its movement.

\section*{Capture-fields}
\addcontentsline{toc}{section}{Capture-fields}
\label{sec:Terms/Capture-fields}
Capture-fields are all fields where a piece can capture opponent's piece.
Usually, these are the same as step-fields, except for Pawn, Scout, Grenadier,
and Shaman.

Some pieces cannot capture opponent's pieces, so they have only step-fields
but no capture-fields; these are Wave, Star, Monolith, and Starchild.

\section*{Neighboring-fields}
\addcontentsline{toc}{section}{Neighboring-fields}
\label{sec:Terms/Neighboring-fields}
Neighboring-fields are all fields immediately surrounding a particular field
horizontally, vertically and diagonally.

\section*{Portal-fields}
\addcontentsline{toc}{section}{Portal-fields}
\label{sec:Terms/Portal-fields}
Portal-fields are neighboring-fields around a Star, or a Monolith. Empty
portal-fields can be used as a destination after a material (i.e. non-Wave)
piece teleported.

\section*{Miracle-fields}
\addcontentsline{toc}{section}{Miracle-fields}
\label{sec:Terms/Royal-fields}
Miracle-fields are neighboring-fields around a Starchild, where any own piece
can be activated (except King), opponent's Starchild, or any Star.

Empty miracle-fields can also be used as a destination, in case of resurrecting
a Wave, or a Starchild.

\section*{Activator}
\addcontentsline{toc}{section}{Activator}
\label{sec:Terms/Activator}
Activator is any material piece in a cascade.

Wave inherits its (step-, capture-, or miracle-) fields from activating piece;
activating piece itself can be Wave with inherited fields; inheriting chain
starts with an activator.

Usually, activator refers to last material piece in a cascade preceding Wave,
from which that Wave ultimately inherited (step-, capture-, or miracle-) fields.

\section*{Step}
\addcontentsline{toc}{section}{Step}
\label{sec:Terms/Step}
Step is a movement of a piece from one step-, or capture-field to the next of
the same kind; or from a starting field to a destination (step-, capture-, or
miracle-) field.

\section*{Rush}
\addcontentsline{toc}{section}{Rush}
\label{sec:Terms/Rush}
Rush is initial movement of a private longer then its usual, i.e. from its starting
position, for at least 2 fields forward. Rushing private presents opponent with en
passant opportunity.

\section*{Displacement-fields}
\addcontentsline{toc}{section}{Displacement-fields}
\label{sec:Terms/Displacement-fields}
Displacement-fields are all fields where a piece can be displaced.

Displacement-fields are different from step-, and capture-fields, and form a fixed
pattern regardless which piece is being displaced.

\section*{Displacement}
\addcontentsline{toc}{section}{Displacement}
\label{sec:Terms/Displacement}
Displacement is act of moving a piece onto an empty displacement-field.

Displacement can be initiated when a piece encounters another on its step-,
or capture-fields. No momentum is transferred to, or used by displaced piece.
Displacement can be performed even if initiating piece does not have any
momentum. After displacement, initiating piece can continue its movement
as if no action has been taken.

For instance, Serpent can displace Pawns it encounters; Shaman can displace
all pieces, except Kings, Stars, Monoliths during its trance-journey.

\section*{Tag}
\addcontentsline{toc}{section}{Tag}
\label{sec:Terms/Tag}
Tag is a delayed opportunity link between a piece and a field at which it stands.
Only one tag at any given time can be applied to a piece.

Piece can be tagged for castling, promotion or rushing; doing any of these things
consumes tag, and cannot be repeated again. For instance, Pawn can be rushed for
less than maximum allowed in a variant; regardless, rushed Pawn cannot be rushed
again.

Tag, and opportunity it represents, is definitely lost when tagged piece is moved,
captured, converted, activated or displaced.

Initially, all privates are tagged for rushing, and all Rooks and Kings are tagged
for castling. Later in game, Pawns can be tagged for promotion.

As a special case, Serpent can be tagged for Pawn-sacrifice; this tag is not
delayed, so it has to be used in the same move in which it has been obtained.

\section*{Pawn row}
\addcontentsline{toc}{section}{Pawn row}
\label{sec:Terms/Pawn row}
Pawn row is any row which contains Pawns on initial setup of chessboard.

In early variants (up to Nineteen), for light player that is second row, for
dark player second to last row. In Nineteen variant an additional rank of
Pawns was added, and so Pawns rows are second and third for light player,
second to last and third to last for dark player.

\section*{Private row}
\addcontentsline{toc}{section}{Private row}
\label{sec:Terms/Private row}
Private row is any row which contains privates on initial setup of chessboard.

Privates are added in Hemera's Dawn variant; for light player Scouts are positioned
onto third and fourth row, Grenadiers replace some of Pawns on second and third row;
changes are mirrored for dark player.

So, private rows contain rows with Scouts, in addition to Pawn rows.

\section*{Figure row}
\addcontentsline{toc}{section}{Figure row}
\label{sec:Terms/Figure row}
Figure row is row that contains figures on initial setup of chessboard.
For light player that is first row, for dark player it is last row.

\section*{Piece row}
\addcontentsline{toc}{section}{Piece row}
\label{sec:Terms/Piece row}
Piece row is either private row or figure row.

\clearpage % ..........................................................
% ------------------------------------------------------- Terms chapter
