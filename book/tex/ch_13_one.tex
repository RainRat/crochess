
% Copyright (c) 2015 - 2021 Mario Mlačak, mmlacak@gmail.com
% Published as Public Domain work, under CC0 1.0 Universal Public Domain Dedication. See LICENSING, COPYING files for details.

% One chapter =========================================================
\chapter*{One}
\addcontentsline{toc}{chapter}{One}
\label{ch:One}

\begin{flushright}
\parbox{0.8\textwidth}{
\emph{God is not external to anyone, but is present with all things, though
they are ignorant that he is so. \newline
\hspace*{\fill}{\textperiodcentered \textperiodcentered \textperiodcentered \hspace*{0.2em} Plotinus} } }
\end{flushright}

\noindent
One is chess variant which is played on 26 x 26 board, with white and
darker violet fields, and with light purple and fuchsia pieces. Star
colors are reversed colors of ordinary pieces, i.e. fuchsia and light
purple. A new piece is introduced, Starchild.

\clearpage % ..........................................................
% Starchild ***********************************************************

\section*{Starchild}
\addcontentsline{toc}{section}{Starchild}
\label{sec:One/Starchild}

\vspace*{-0.8\baselineskip}
\noindent
\begin{wrapfigure}[11]{l}{0.4\textwidth}
\centering
\includegraphics[width=0.4\textwidth, keepaspectratio=true]{pieces/18_starchild.png}
\caption{Starchild}
\label{fig:18_starchild}
\end{wrapfigure}
Starchild cannot capture any piece, cannot check or checkmate opponent's King.
Starchild is celestial piece, it can participate in demoting-to-Pawn syzygy.
Starchild can be demoted to Pawn.

Starchild cannot be converted. Starchild can diverge pieces the same as Shaman.
Starchild can activate own Waves and Starchilds on its step-fields. Starchild can
activate any own piece (except King), opponent's Starchild and any Star on its
miracle-fields. Activated Starchild does not use momentum while moving.

\noindent
\begin{wrapfigure}[11]{l}{0.4\textwidth}
\centering
\includegraphics[width=0.4\textwidth, keepaspectratio=true]{pieces/star/22_one.png}
\caption{Star}
\label{fig:star/22_one}
\end{wrapfigure}
Starchild can't teleport. In a single ply, Starchild can make only one step, from
its starting field to destination in opposite color, without interacting with any
piece on chessboard.

Starchild can resurrect any captured piece, except Kings, Stars, Monoliths. Waves
and Starchilds can be resurrected without resurrecting Starchild being oblationed.
Starchild can take any own piece, except Kings, Waves, Stars, Starchilds and
Monoliths, for a sense-journey.

Star colors in this variant are the opposite of colors of other pieces.

\clearpage % ..........................................................
% Movement ------------------------------------------------------------

\subsection*{Movement}
\addcontentsline{toc}{subsection}{Movement}
\label{sec:One/Starchild/Movement}

\vspace*{-0.9\baselineskip}
\noindent
\begin{wrapfigure}[9]{l}{0.4\textwidth}
\centering
\includegraphics[width=0.346153846\textwidth, keepaspectratio=true]{examples/22_o/scn_o_01_starchild_movement.png}
\caption{Starchild movement}
\label{fig:scn_o_01_starchild_movement}
\end{wrapfigure}
In a single ply, Starchild can make only one step. Starchild can move to any empty
field in opposite color to its starting position, regardless how far away destination
field is located. Starchild's movement is not obstructed by any piece standing between
starting and destination field.

\vspace*{-0.9\baselineskip}
\subsubsection*{Activating on step-fields}
\addcontentsline{toc}{subsubsection}{Activating on step-fields}
\label{sec:One/Starchild/Movement/Activating on step-fields}

\vspace*{-0.9\baselineskip}
\noindent
\begin{wrapfigure}[7]{l}{0.4\textwidth}
\centering
\includegraphics[width=0.346153846\textwidth, keepaspectratio=true]{examples/22_o/scn_o_02_starchild_activating_own_starchild.png}
\caption{Activating Starchild}
\label{fig:scn_o_02_starchild_activating_own_starchild}
\end{wrapfigure}
Starchild can activate own Starchilds on its step-fields, with 1 momentum. Activated
Starchild moves the same as activating Starchild. Activated Starchild does not spend
momentum for its movement.

% \vspace*{-0.9\baselineskip}
\noindent
\begin{wrapfigure}[10]{l}{0.4\textwidth}
\centering
\includegraphics[width=0.346153846\textwidth, keepaspectratio=true]{examples/22_o/scn_o_03_starchild_activating_own_wave.png}
\caption{Activating Wave}
\label{fig:scn_o_03_starchild_activating_own_wave}
\end{wrapfigure}
Starchild can activate own Waves on its step-fields, with 1 momentum. Activated Wave
moves the same as Starchild, i.e. to any empty field in opposite color to its starting
field, without being limited by distance to its destination. \newline
\indent
Activated Wave can also activate opponent's Waves in addition to own Starchilds and
Waves.

\clearpage % ..........................................................

\subsubsection*{Miracle-fields}
\addcontentsline{toc}{subsubsection}{Miracle-fields}
\label{sec:One/Starchild/Movement/Miracle-fields}

% \vspace*{-0.9\baselineskip}
\noindent
\begin{wrapfigure}[5]{l}{0.4\textwidth}
\centering
\includegraphics[width=0.192307692\textwidth, keepaspectratio=true]{examples/22_o/scn_o_05_miracle_fields.png}
\caption{Miracle-fields}
\label{fig:scn_o_05_miracle_fields}
\end{wrapfigure}
Miracle-fields are all fields immediately surrounding Starchild horizontally,
vertically and diagonally. They are the same as step-fields of a King.

\vspace*{2.1\baselineskip}
\subsubsection*{Activating on miracle-fields}
\addcontentsline{toc}{subsubsection}{Activating on miracle-fields}
\label{sec:One/Starchild/Movement/Activating on miracle-fields}

% \vspace*{-0.9\baselineskip}
\noindent
\begin{wrapfigure}[8]{l}{0.4\textwidth}
\centering
\includegraphics[width=0.192307692\textwidth, keepaspectratio=true]{examples/22_o/scn_o_06_starchild_activating_on_miracle_fields.png}
\caption{Activating piece}
\label{fig:scn_o_06_starchild_activating_on_miracle_fields}
\end{wrapfigure}
Fields at which Starchild can activate a piece are miracle-fields; pieces that
can be activated are all own pieces (except King), and opponent's Starchild.

Note, Starchild cannot move to empty miracle-fields, if they're not in opposite
color to starting field; it can only activate a piece on its miracle-field.

Here, Starchild's activation fields are enumerated. Opponent's Bishop and own King
can't be activated, so only own Pyramid can be, with 1 momentum.

\clearpage % ..........................................................

% \vspace*{1.1\baselineskip}
\subsubsection*{Starchild cannot teleport}
\addcontentsline{toc}{subsubsection}{Starchild cannot teleport}
\label{sec:One/Starchild/Movement/Starchild cannot teleport}

% \vspace*{-0.9\baselineskip}
\noindent
\begin{wrapfigure}[6]{l}{0.4\textwidth}
\centering
\includegraphics[width=0.346153846\textwidth, keepaspectratio=true]{examples/22_o/scn_o_07_starchild_not_moving_monolith_init.png}
\caption{Moving into a Monolith}
\label{fig:scn_o_07_starchild_not_moving_monolith_init}
\end{wrapfigure}
Starchild cannot teleport. Starchild trying to capture either step- or miracle-field
at which Monolith (or a Star) is located, will emerge on empty portal-field
surrounding that same Monolith (or a Star).

\vspace*{5.1\baselineskip}
\noindent
\begin{wrapfigure}[8]{l}{0.4\textwidth}
\centering
\includegraphics[width=0.346153846\textwidth, keepaspectratio=true]{examples/22_o/scn_o_08_starchild_not_moving_monolith_end.png}
\caption{Moving out of a Monolith}
\label{fig:scn_o_08_starchild_not_moving_monolith_end}
\end{wrapfigure}
In previous example Starchild moved into a Monolith; here, it emerges on empty
portal-field of that same Monolith.

If there were no empty portal-fields, Starchild would be oblationed, i.e. removed
from chessboard as if captured by opponent.

\clearpage % ..........................................................

% \vspace*{2.1\baselineskip}
\subsubsection*{Moving a Star}
\addcontentsline{toc}{subsubsection}{Moving a Star}
\label{sec:One/Starchild/Movement/Moving a Star}

% \vspace*{-0.9\baselineskip}
\noindent
\begin{wrapfigure}[5]{l}{0.4\textwidth}
\centering
\includegraphics[width=0.192307692\textwidth, keepaspectratio=true]{examples/22_o/scn_o_09_starchild_moving_star_init.png}
\caption{Moving into a Star}
\label{fig:scn_o_09_starchild_moving_star_init}
\end{wrapfigure}
Starchild can activate a Star the same way as any other piece, i.e. by capturing
miracle-field at which Star is located. Activated Star receives 1 momentum.

\vspace*{2.1\baselineskip}
\noindent
\begin{wrapfigure}[4]{l}{0.4\textwidth}
\centering
\includegraphics[width=0.192307692\textwidth, keepaspectratio=true]{examples/22_o/scn_o_10_starchild_moving_star_end.png}
\caption{Star moving}
\label{fig:scn_o_10_starchild_moving_star_end}
\end{wrapfigure}
Once activated, Star can move to any empty miracle-field of activating Starchild,
which all are enumerated in example on the left.

\vspace*{2.3\baselineskip}
\noindent
\begin{wrapfigure}[8]{l}{0.4\textwidth}
\centering
\includegraphics[width=0.192307692\textwidth, keepaspectratio=true]{examples/22_o/scn_o_11_starchild_moving_star_activating.png}
\caption{Activating Starchild}
\label{fig:scn_o_11_starchild_moving_star_activating}
\end{wrapfigure}
Note, even if activated Starchild received more than 1 momentum, Star can move for
only one step.

Here, Star received all of initial 3 momentum gathered by the Rook, since neither
Wave nor Starchild expend momentum for its movement. Nevertheless, activated Star
can move for only one field.

Note that Star movement is optional, Starchild could emerge on empty portal-field
around Star in question; if there are none, Starchild would be oblationed.

\clearpage % ..........................................................

% \vspace*{2.1\baselineskip}
\subsubsection*{Star movement blocked}
\addcontentsline{toc}{subsubsection}{Star movement blocked}
\label{sec:One/Starchild/Movement/Star movement blocked}

% \vspace*{-0.9\baselineskip}
\noindent
\begin{wrapfigure}[4]{l}{0.4\textwidth}
\centering
\includegraphics[width=0.192307692\textwidth, keepaspectratio=true]{examples/22_o/scn_o_12_star_movement_blocked_init.png}
\caption{Activating Star}
\label{fig:scn_o_12_star_movement_blocked_init}
\end{wrapfigure}
If there is no empty field around Star at actual activation, Star is not activated
(it remains at the same field), and Starchild is oblationed.

\vspace*{4.1\baselineskip}
\noindent
\begin{wrapfigure}[7]{l}{0.4\textwidth}
\centering
\includegraphics[width=0.192307692\textwidth, keepaspectratio=true]{examples/22_o/scn_o_13_star_movement_blocked_end.png}
\caption{Star blocked}
\label{fig:scn_o_13_star_movement_blocked_end}
\end{wrapfigure}
Here, Starchild (now "it-the-air") is about to activate Star; grey arrows show path
traveled over by the piece they point to. There is no empty field around Star, so it
cannot be activated; instead, Starchild is oblationed.

\subsubsection*{Rerouting Scout}
\addcontentsline{toc}{subsubsection}{Rerouting Scout}
\label{sec:One/Starchild/Movement/Rerouting Scout}

% \vspace*{-0.9\baselineskip}
\noindent
\begin{wrapfigure}[8]{l}{0.4\textwidth}
\centering
\includegraphics[width=0.384615385\textwidth, keepaspectratio=true]{examples/22_o/scn_o_14_star_rerouting_scout.png}
\caption{Rerouting Scout}
\label{fig:scn_o_14_star_rerouting_scout}
\end{wrapfigure}
In this variant, Star can be activated, and moved onto any field on a chessboard,
and so it can also block other pieces from stepping over field it occupies. \newline
\indent
\hyperref[fig:scn_d_10_monolith_rerouting_scout]{The same as around Monolith},
Scout can be rerouted around a blocking Star, using appropriate
\hyperref[fig:scn_hd_17_scout_forking_steps]{forking step} (here, either left-up,
or left-down), then continuing in initially chosen direction (here, left).

\clearpage % ..........................................................

% \vspace*{1.1\baselineskip}
\subsubsection*{Starchild is completely transparent}
\addcontentsline{toc}{subsubsection}{Starchild is completely transparent}
\label{sec:One/Starchild/Divergence/Starchild is completely transparent}

\vspace*{-1.4\baselineskip}
\noindent
\begin{figure}[!h]
\includegraphics[width=1.0\textwidth, keepaspectratio=true]{examples/22_o/scn_o_15_starchild_is_transparent.png}
\vspace*{-1.3\baselineskip}
\caption{Starchild is transparent}
\label{fig:scn_o_15_starchild_is_transparent}
\end{figure}

\vspace*{-0.4\baselineskip}
\hyperref[fig:scn_mv_07_wave_is_transparent]{Like Wave}, Starchild is transparent
to all pieces, both own and opponent's. Above, light Queen is not blocked by dark
Starchild on its step-field, and can continue moving past it, and could also e.g.
diverge from light Shaman.

% \vspace*{-1.4\baselineskip}
\vspace*{-0.7\baselineskip}
\noindent
\begin{figure}[!h]
\includegraphics[width=1.0\textwidth, keepaspectratio=true]{examples/22_o/scn_o_16_starchild_is_completely_transparent.png}
\vspace*{-1.3\baselineskip}
\caption{Starchild is completely transparent}
\label{fig:scn_o_16_starchild_is_completely_transparent}
\end{figure}

\vspace*{-0.4\baselineskip}
\hyperref[fig:scn_d_02_monolith_first_step]{Unlike Wave}, Starchild is completely
transparent. While Monolith cannot interact with any piece on its step-fields, it's
not blocked by encountered Starchild, and can continue its movement beyond such an
encounter.

\clearpage % ..........................................................

% \vspace*{1.1\baselineskip}
\subsubsection*{Conversion immunity}
\addcontentsline{toc}{subsubsection}{Conversion immunity}
\label{sec:One/Starchild/Movement/Conversion immunity}

\vspace*{-1.4\baselineskip}
\noindent
\begin{figure}[!h]
\includegraphics[width=1.0\textwidth, keepaspectratio=true]{examples/22_o/scn_o_17_starchild_conversion_immunity_init.png}
\caption{Conversion immunity}
\label{fig:scn_o_17_starchild_conversion_immunity_init}
\end{figure}

\hyperref[sec:Mayan Ascendancy/Pyramid/Conversion]{Conversion} is a move in which
activated Pyramid reaches opponent's piece, if it's not King, on own side of board.
Pyramid is then oblationed, and reached piece is replaced by the same piece in own
color. Starchild cannot be converted, instead, original Starchild remains on
chessboard; conversioning Pyramid is still oblationed.

% ------------------------------------------------------------ Movement
\clearpage % ..........................................................
% Divergence ----------------------------------------------------------

\subsection*{Divergence}
\addcontentsline{toc}{subsection}{Divergence}
\label{sec:One/Starchild/Divergence}

\vspace*{-1.5\baselineskip}
\noindent
\begin{figure}[!h]
\includegraphics[width=1.0\textwidth, keepaspectratio=true]{examples/22_o/scn_o_18_starchild_divergence_init.png}
\vspace*{-1.4\baselineskip}
\caption{Starchild is divergent}
\label{fig:scn_o_18_starchild_divergence_init}
\end{figure}

\vspace*{-0.5\baselineskip}
Pieces can diverge from Starchild, similarly
\hyperref[fig:scn_cot_030_own_shaman_is_divergent_init]{from Shaman}. Unlike
Shaman, Starchild is divergent for all pieces that can diverge, own and opponent's.
Kings, Stars, Monoliths, Centaurs, Serpents, Starchilds cannot diverge, in addition
to Waves activated by Unicorns, Centaurs, or Serpents. Here, light Queen can
diverge from first dark Starchild, or (due to Starchild's total transparency)
second Starchild.

\vspace*{-0.9\baselineskip}
\noindent
\begin{figure}[!h]
\includegraphics[width=1.0\textwidth, keepaspectratio=true]{examples/22_o/scn_o_19_starchild_divergence_end.png}
\vspace*{-1.4\baselineskip}
\caption{Diverging Queen}
\label{fig:scn_o_19_starchild_divergence_end}
\end{figure}

\vspace*{-0.5\baselineskip}
\hyperref[fig:scn_cot_059_diverging_first_piece]{As before}, first piece in a move
cannot return to its starting position. Here, light Queen (now "in the air") has
reached own Starchild, and can choose any shown direction as a new one. Light Queen
cannot return to its starting position, i.e. field Q.

\clearpage % ..........................................................

% \vspace*{-1.1\baselineskip}
\subsubsection*{Diverging limits}
\addcontentsline{toc}{subsubsection}{Diverging limits}
\label{sec:One/Starchild/Divergence/Diverging limits}

\vspace*{-0.9\baselineskip}
\noindent
\begin{wrapfigure}[9]{l}{0.4\textwidth}
\centering
\includegraphics[width=0.346153846\textwidth, keepaspectratio=true]{examples/22_o/scn_o_23_activating_piece_surplus_momentum.png}
\vspace*{-0.4\baselineskip}
\caption{Activating with surplus momentum}
\label{fig:scn_o_23_activating_piece_surplus_momentum}
\end{wrapfigure}
When diverging, pieces are restricted by both how much momentum they have, and how
many steps they can make in a single ply. \newline
\indent
Similar to
\hyperref[fig:scn_mv_35_single_step_piece_momentum]{pieces activated with surplus momentum},
diverging pieces also transfer all remaining momentum to a piece they activate.

% In previous example, when wrapped text is declared to be 8 lines, text is
% formatted to 9 lines, and then last line is hanged over image.
% When wrapped text is declared to be 9 lines, text is formatted to 8 lines,
% and line from next example is pulled-in.
% \hspace*{\fill} % <!> Leave it here <!>

\vspace*{0.4\baselineskip}
\noindent
\begin{wrapfigure}[9]{l}{0.4\textwidth}
\centering
\includegraphics[width=0.346153846\textwidth, keepaspectratio=true]{examples/22_o/scn_o_24_diverging_piece_surplus_momentum.png}
\vspace*{-0.4\baselineskip}
\caption{Diverging limits}
\label{fig:scn_o_24_diverging_piece_surplus_momentum}
\end{wrapfigure}
In previous example light Knight reached Starchild with 4 momentum. Here, Knight
is about to diverge from light Starchild, and can perform only one additional step
regardless of how much momentum it has. If Knight would reactivate light Wave, it
would transfer all 3 remaining momentum.

% \TODO :: fix lmodern

\vspace*{-0.9\baselineskip}
\subsubsection*{Starchild cannot diverge}
\addcontentsline{toc}{subsubsection}{Starchild cannot diverge}
\label{sec:One/Starchild/Divergence/Starchild cannot diverge}

\vspace*{-0.9\baselineskip}
\noindent
\begin{wrapfigure}[7]{l}{0.4\textwidth}
\centering
\includegraphics[width=0.346153846\textwidth, keepaspectratio=true]{examples/22_o/scn_o_25_starchild_cannot_diverge.png}
\vspace*{-0.4\baselineskip}
\caption{Starchild cannot diverge}
\label{fig:scn_o_25_starchild_cannot_diverge}
\end{wrapfigure}
Image on the left have two examples presented in parallel; at the top, and to the
bottom. \newline
\indent
Starchild cannot diverge, neither from step- (here, bottom example) nor miracle-fields
(top example).

% ---------------------------------------------------------- Divergence
\clearpage % ..........................................................
% Activating Wave -----------------------------------------------------

% \vspace*{-0.1\baselineskip}
\subsection*{Activating Wave}
\addcontentsline{toc}{subsection}{Activating Wave}
\label{sec:One/Starchild/Activating Wave}

% \vspace*{-0.9\baselineskip}
\noindent
\begin{wrapfigure}[8]{l}{0.4\textwidth}
\centering
\includegraphics[width=0.192307692\textwidth, keepaspectratio=true]{examples/22_o/scn_o_30_starchild_activating_wave_on_miracle_fields.png}
\caption{Activating Wave}
\label{fig:scn_o_30_starchild_activating_wave_on_miracle_fields}
\end{wrapfigure}
Wave activated by Starchild on its miracle-fields can activate a piece, with
1 momentum; any own piece (except King), and opponent's Waves can be activated.

Wave can also move to any empty miracle-field, regardless of color.

\subsubsection*{Wave cannot move a Star}
\addcontentsline{toc}{subsubsection}{Wave cannot move a Star}
\label{sec:One/Starchild/Activating Wave/Wave cannot move a Star}

\noindent
\begin{wrapfigure}[9]{l}{0.4\textwidth}
\centering
\includegraphics[width=0.192307692\textwidth, keepaspectratio=true]{examples/22_o/scn_o_31_starchild_activated_wave_not_moving_a_star.png}
\caption{Not moving a Star}
\label{fig:scn_o_31_starchild_activated_wave_not_moving_a_star}
\end{wrapfigure}
Wave cannot move a Star, even if activated by Starchild on its miracle-field.
Instead, Wave would emerge on any empty portal-field around a Star which it tried
to activate (here, fields 1 and 2). If there is no empty portal-field, Wave is
oblationed.

\clearpage % ..........................................................

\subsubsection*{Wave cannot teleport}
\addcontentsline{toc}{subsubsection}{Wave cannot teleport}
\label{sec:One/Starchild/Activating Wave/Wave cannot teleport}

\vspace*{-0.1\baselineskip}
\noindent
\begin{wrapfigure}[4]{l}{0.4\textwidth}
\centering
\includegraphics[width=0.307692308\textwidth, keepaspectratio=true]{examples/22_o/scn_o_32_starchild_activated_wave_not_teleporting_init.png}
\vspace*{-0.4\baselineskip}
\caption{Moving into a Star}
\label{fig:scn_o_32_starchild_activated_wave_not_teleporting_init}
\end{wrapfigure}
Wave activated by Starchild cannot teleport, regardless if Wave was activated on
Starchild's step- or miracle-field.

\vspace*{5.7\baselineskip}
\noindent
\begin{wrapfigure}[7]{l}{0.4\textwidth}
\centering
\includegraphics[width=0.307692308\textwidth, keepaspectratio=true]{examples/22_o/scn_o_33_starchild_activated_wave_not_teleporting_end.png}
\vspace*{-0.4\baselineskip}
\caption{Moving out of a Star}
\label{fig:scn_o_33_starchild_activated_wave_not_teleporting_end}
\end{wrapfigure}
Instead of teleporting, Wave emerges on empty portal-field around Monolith or a
Star through which it tried to teleport.

If there is no empty portal-field around Monolith (or a Star), Wave is oblationed.

\clearpage % ..........................................................

% \vspace*{1.1\baselineskip}
\subsubsection*{Teleporting Wave}
\addcontentsline{toc}{subsubsection}{Teleporting Wave}
\label{sec:One/Starchild/Activating Wave/Teleporting Wave}

\vspace*{-1.5\baselineskip}
\noindent
\begin{figure}[!h]
\includegraphics[width=1.0\textwidth, keepaspectratio=true]{examples/22_o/scn_o_34_star_moved_wave_teleportation.png}
\vspace*{-1.4\baselineskip}
\caption{Optional Wave teleportation}
\label{fig:scn_o_34_star_moved_wave_teleportation}
\end{figure}

\vspace*{-0.5\baselineskip}
Wave activated by pieces other than Starchild can still teleport as usual. Stars
in this variant can be moved out of their default positions. Teleportation for
Wave reaching a Star is optional, step-fields behind a Star are still accessible.
Here, light Wave could also activate light Queen. So,
\hyperref[fig:scn_d_14_teleport_wave_via_monolith]{Monolith is the only piece Wave cannot pass over},
i.e. ignore as all the other pieces on chessboard.

\clearpage % ..........................................................

\vspace*{-2.1\baselineskip}
\noindent
\begin{figure}[!h]
\includegraphics[width=1.0\textwidth, keepaspectratio=true]{examples/22_o/scn_o_35_star_moved_wave_off_board.png}
\caption{Wave teleported off-board}
\label{fig:scn_o_35_star_moved_wave_off_board}
\end{figure}

Wave can end up with all step-fields off-board after teleportation, due to one or
both Stars moved out of their initial positions. In such a case, Wave is oblationed,
the same as in
\hyperref[fig:scn_d_16_wave_teleported_off_board]{previous variant, Discovery}.

Wave is also removed from chessboard if, after teleportation, all of its step-fields
are blocked; this is again similar to
\hyperref[fig:scn_d_15_teleported_wave_blocked]{previous variant, Discovery}.

\clearpage % ..........................................................

% \vspace*{1.1\baselineskip}
\subsubsection*{Steps after teleportation}
\addcontentsline{toc}{subsubsection}{Steps after teleportation}
\label{sec:One/Starchild/Activating Wave/Steps after teleportation}

\vspace*{-1.5\baselineskip}
\noindent
\begin{figure}[!h]
\includegraphics[width=1.0\textwidth, keepaspectratio=true]{examples/22_o/scn_o_36_steps_after_teleport.png}
\vspace*{-1.4\baselineskip}
\caption{Steps after teleportation}
\label{fig:scn_o_36_steps_after_teleport}
\end{figure}

\vspace*{-0.5\baselineskip}
Wave, activated by Unicorn (or Centaur) after teleportation has to follow two initially
chosen steps (long and short jump) according to a color of step-fields. Similar to
\hyperref[fig:scn_d_19_steps_after_teleport_init]{previous example with Monoliths},
two same-color Stars can be moved onto opposite-color fields. So, a two-step pattern
after teleportation remains the same, but order of steps is reversed. Here, after
teleporting into Star A, Wave emerges from Star B using the same two initial steps,
only in reversed order.

% ----------------------------------------------------- Activating Wave
\clearpage % ..........................................................

\subsection*{Failed trance-journey}
\addcontentsline{toc}{subsection}{Failed trance-journey}
\label{sec:One/Starchild/Failed trance-journey}

\vspace*{-1.5\baselineskip}
\noindent
\begin{figure}[!h]
\includegraphics[width=1.0\textwidth, keepaspectratio=true]{examples/22_o/scn_o_37_trance_journey_failed.png}
\vspace*{-1.4\baselineskip}
\caption{Failed trance-journey}
\label{fig:scn_o_37_trance_journey_failed}
\end{figure}

\vspace*{-0.5\baselineskip}
In \hyperref[fig:scn_cot_082_light_shaman_trance_journey]{trance-journey}, entranced
Shaman cannot interact with Kings, Stars, and Monoliths; so, step-fields occupied by
those pieces are blocked. In this variant
\hyperref[fig:scn_o_09_starchild_moving_star_init]{Stars can be moved}, and could
potentially block step-fields. Since trance-journey is mandatory, if all step-fields
are blocked, entranced Shaman is \hyperref[sec:Terms/Oblation]{oblationed}, i.e.
removed from chessboard as if captured by opponent.

\clearpage % ..........................................................
% Sense-journey -------------------------------------------------------

\subsection*{Sense-journey}
\addcontentsline{toc}{subsection}{Sense-journey}
\label{sec:One/Starchild/Sense-journey}

\vspace*{-0.7\baselineskip}
Sense-journey is initiated by stationary Starchild activating another Starchild
on its uplifting-field; activated Starchild can then activate a piece on its
uplifting-field, which then takes on sense-journey. Activated Starchild is also
called uplifting Starchild, a piece taking on sense-journey is uplifted piece.
Initiating piece can be Shaman instead of a Starchild; stationary Shaman would
activate uplifting Starchild on its
\hyperref[fig:scn_cot_070_trance_fields]{trance-fields}. Colors of Starchilds,
Shaman do not need to match; uplifted piece has to be in the same color as
uplifting Starchild.

\vspace*{-0.9\baselineskip}
\subsubsection*{Uplifting-fields}
\addcontentsline{toc}{subsubsection}{Uplifting-fields}
\label{sec:One/Starchild/Sense-journey/Uplifting-fields}

\vspace*{-0.9\baselineskip}
\noindent
\begin{wrapfigure}[5]{l}{0.4\textwidth}
\centering
\includegraphics[width=0.192307692\textwidth, keepaspectratio=true]{examples/22_o/scn_o_40_uplifting_fields.png}
\vspace*{-0.4\baselineskip}
\caption{Uplifting-fields}
\label{fig:scn_o_40_uplifting_fields}
\end{wrapfigure}
Uplifting-fields are all fields immediately neighboring Starchild horizontally,
vertically, and diagonally. They are the same fields as step-fields of a King.

\vspace*{1.3\baselineskip}

\subsubsection*{Uplifting}
\addcontentsline{toc}{subsubsection}{Uplifting}
\label{sec:One/Starchild/Sense-journey/Uplifting}

\vspace*{-0.9\baselineskip}
\noindent
\begin{wrapfigure}[9]{l}{0.4\textwidth}
\centering
\includegraphics[width=0.346153846\textwidth, keepaspectratio=true]{examples/22_o/scn_o_41_uplifting_init.png}
\vspace*{-0.4\baselineskip}
\caption{Uplifting preparation}
\label{fig:scn_o_41_uplifting_init}
\end{wrapfigure}
In a single ply, Starchild can travel over only one of step-, miracle- or uplifting-fields;
choice can be made only on the very first step, and cannot be changed for duration of the
ply. \newline
\indent
Here, light Starchild can be moved onto field I, so that its uplifting-field is occupied
by dark Starchild. It's illegal to change course during the ply, so light Starchild cannot
uplift dark Starchild outright.

\clearpage % ..........................................................

\vspace*{-2.1\baselineskip}
\noindent
\begin{wrapfigure}[8]{l}{0.4\textwidth}
\centering
\includegraphics[width=0.346153846\textwidth, keepaspectratio=true]{examples/22_o/scn_o_42_uplifting_step.png}
\vspace*{-0.4\baselineskip}
\caption{Uplifting step}
\label{fig:scn_o_42_uplifting_step}
\end{wrapfigure}
Once in a position, stationary Starchild can initiate sense-journey by simply stepping
onto its uplifting-field occupied by the other Starchild; uplifting Starchild can then
uplift any piece on its uplifting-field; uplifted piece then has to go onto sense-journey.

\vspace*{-0.4\baselineskip}

\noindent
\begin{wrapfigure}[8]{l}{0.4\textwidth}
\centering
\includegraphics[width=0.346153846\textwidth, keepaspectratio=true]{examples/22_o/scn_o_43_uplifting_activated.png}
\vspace*{-0.4\baselineskip}
\caption{Activated uplifting step}
\label{fig:scn_o_43_uplifting_activated}
\end{wrapfigure}
Activated Starchild can also initiate sense-journey. This is so, even if initiating
Starchild has no momentum; like in the example on the left. \newline
\indent
Note, sense-journey is mandatory; once a piece (here, dark Bishop) is uplifted it has
to make sense-journey.

\vspace*{-0.4\baselineskip}

\noindent
\begin{wrapfigure}[8]{l}{0.4\textwidth}
\centering
\includegraphics[width=0.346153846\textwidth, keepaspectratio=true]{examples/22_o/scn_o_44_shaman_initiated_uplifting.png}
\vspace*{-0.4\baselineskip}
\caption{Shaman initiated uplifting}
\label{fig:scn_o_44_shaman_initiated_uplifting}
\end{wrapfigure}
Instead of a Starchild, stationary Shaman can initiate sense-journey, by activating
uplifting Starchild on its \hyperref[fig:scn_cot_070_trance_fields]{trance-fields};
this is so regardless if initiating Shaman has been activated or not.

Uplifting Starchild and uplifted piece must belong to the same player; initiating
piece can belong to opponent, regardless if it's Shaman or Starchild. King, Waves cannot
be uplifted; Stars, Monoliths do not belong to any player; Starchilds cannot take on
sense-journey.

Sense-journey has the same movement pattern as trance-journey. If uplifted piece
is light, it can take sense-journey in one chosen direction from
\hyperref[fig:scn_cot_082_light_shaman_trance_journey]{light Shaman's pattern},
otherwise from
\hyperref[fig:scn_cot_084_dark_shaman_trance_journey]{dark Shaman's pattern}.

\clearpage % ..........................................................

\vspace*{-2.1\baselineskip}
\noindent
\begin{figure}[!h]
\includegraphics[width=1.0\textwidth, keepaspectratio=true]{examples/22_o/scn_o_45_dark_piece_sense_journey.png}
\vspace*{-1.4\baselineskip}
\caption{Dark piece sense-journey}
\label{fig:scn_o_45_dark_piece_sense_journey}
\end{figure}

\vspace*{-0.5\baselineskip}
Uplifted piece (here, dark Bishop) can end its sense-journey on any empty step-field,
of chosen direction (color). Occupied step-field (here, by light Rook) does not block
uplifted piece from ending its sense-journey on any empty step-field behind it.
Uplifted piece has to take on sense-journey because it's mandatory, even if it received
no momentum; length of sense-journey does not depend on received momentum. Above,
activated Shaman initiated sense-journey; grey arrows show path traveled over by
piece they point to.

\clearpage % ..........................................................

\subsubsection*{Failed sense-journey}
\addcontentsline{toc}{subsubsection}{Failed sense-journey}
\label{sec:One/Starchild/Sense-journey/Failed sense-journey}

\vspace*{-1.4\baselineskip}
\noindent
\begin{figure}[!h]
\includegraphics[width=1.0\textwidth, keepaspectratio=true]{examples/22_o/scn_o_46_sense_journey_failed.png}
\vspace*{-1.4\baselineskip}
\caption{Failed sense-journey}
\label{fig:scn_o_46_sense_journey_failed}
\end{figure}

\vspace*{-0.5\baselineskip}
In sense-journey, uplifted piece cannot interact with pieces; so, step-fields are
blocked, if not empty. Since sense-journey is mandatory, if all step-fields are
blocked, uplifted piece is \hyperref[sec:Terms/Oblation]{oblationed}, i.e. removed
from chessboard as if captured by opponent. \newline
\indent
Here, sense-journey is completely blocked by own and opponent's Pawns; dark Bishop
after uplifting would be oblationed.

% ------------------------------------------------------- Sense-journey
\clearpage % ..........................................................
% Syzygy --------------------------------------------------------------

\subsection*{Syzygy}
\addcontentsline{toc}{subsection}{Syzygy}
\label{sec:One/Starchild/Syzygy}

\vspace*{-1.4\baselineskip}
\noindent
\begin{figure}[!h]
\includegraphics[width=1.0\textwidth, keepaspectratio=true]{examples/22_o/scn_o_50_syzygy_monolith.png}
% \vspace*{-1.5\baselineskip}
\caption{Demoting-to-Pawn syzygy}
\label{fig:scn_o_50_syzygy_monolith}
\end{figure}

% \vspace*{-0.7\baselineskip}
Starchild is celestial piece, it can participate in
\hyperref[fig:scn_d_22_syzygy_2_stars_init]{demoting-to-Pawn syzygy} in place of
Stars and Monoliths. Starchild can also be demoted.

\hyperref[fig:scn_d_21_syzygy_explain]{Again}, shortest step connecting
Stars, Monoliths, Starchilds is called syzygy-step, fields which are connected by
syzygy-steps are called syzygy-fields.

\clearpage % ..........................................................
% Resurrection syzygy .................................................

\subsubsection*{Resurrection syzygy}
\addcontentsline{toc}{subsubsection}{Resurrection syzygy}
\label{sec:One/Starchild/Syzygy/Resurrection syzygy}

\vspace*{-1.4\baselineskip}
\noindent
\begin{figure}[!h]
\includegraphics[width=1.0\textwidth, keepaspectratio=true]{examples/22_o/scn_o_51_syzygy_starchild_init.png}
\caption{Resurrection syzygy start}
\label{fig:scn_o_51_syzygy_starchild_init}
\end{figure}

If Starchild moves into syzygy, from a field not in any alignment, onto an empty
syzygy-field, it's granted option to resurrect one captured piece. A piece is
resurrected by replacing initiating Starchild, Starchild itself is then oblationed.
Only captured pieces can be resurrected. Kings, Stars and Monoliths cannot be
resurrected.

\clearpage % ..........................................................

\vspace*{-2.1\baselineskip}
\noindent
\begin{figure}[!h]
\includegraphics[width=1.0\textwidth, keepaspectratio=true]{examples/22_o/scn_o_52_syzygy_starchild_end.png}
\vspace*{-1.4\baselineskip}
\caption{Queen resurrected}
\label{fig:scn_o_52_syzygy_starchild_end}
\end{figure}

\vspace*{-0.3\baselineskip}
Here, resurrected Queen replaced initiating Starchild. Note, in this variant
\hyperref[sec:One/Promotion]{promotion is monogamous}, so the only light Queen
had to be captured, before it could be resurrected.

Resurrecting is not limited to player's own pieces, opponent's pieces can be
resurrected as well. For instance, dark Queen could be resurrected instead of
the light one. Again, only captured pieces can be resurrected. In this variant,
to resurrect Queen, a Pawn must not be promoted to one.

\clearpage % ..........................................................

\vspace*{-2.1\baselineskip}
\noindent
\begin{figure}[!h]
\includegraphics[width=1.0\textwidth, keepaspectratio=true]{examples/22_o/scn_o_53_syzygy_starchild_resurrection.png}
\caption{Starchild resurrected}
\label{fig:scn_o_53_syzygy_starchild_resurrection}
\end{figure}

Captured Waves and Starchilds can be resurrected, without initiating Starchild being
oblationed. Chosen piece can emerge on any empty
\hyperref[fig:scn_o_05_miracle_fields]{miracle-field} around initiating Starchild in
syzygy (here, enumerated fields). If all miracle-fields are occupied, resurrection
is not performed.

% ................................................. Resurrection syzygy
\clearpage % ..........................................................
% Existing syzygy .....................................................

\subsubsection*{Existing syzygy}
\addcontentsline{toc}{subsubsection}{Existing syzygy}
\label{sec:One/Starchild/Syzygy/Existing syzygy}

\vspace*{-1.2\baselineskip}
\noindent
\begin{figure}[!h]
\includegraphics[width=1.0\textwidth, keepaspectratio=true]{examples/22_o/scn_o_54_syzygy_existing.png}
\caption{Existing syzygy}
\label{fig:scn_o_54_syzygy_existing}
\end{figure}

Similar to \hyperref[fig:scn_d_26_syzygy_existing]{Monolith entering existing syzygy},
Starchild too can enter already existing, and used syzygy, and resurrect one captured
piece, even if a piece has been resurrected in that syzygy, in a previous turn. \newline
\indent
Here, light Wave has been resurrected by Starchild A in a previous turn; regardless,
Starchild B can enter the same syzygy, and resurrect one captured piece.

% ..................................................... Existing syzygy
\clearpage % ..........................................................

\subsubsection*{Reentering syzygy}
\addcontentsline{toc}{subsubsection}{Reentering syzygy}
\label{sec:One/Starchild/Syzygy/Reentering syzygy}

\vspace*{-1.2\baselineskip}
\noindent
\begin{figure}[!h]
\includegraphics[width=1.0\textwidth, keepaspectratio=true]{examples/22_o/scn_o_55_reentering_syzygies.png}
\caption{Reentering syzygy}
\label{fig:scn_o_55_reentering_syzygies}
\end{figure}

Similar to
\hyperref[fig:scn_d_27_syzygy_reentering_same_move]{Monolith reentering syzygy},
to get option to resurrect captured piece, Starchild has to move from a normal,
non-syzygy field into syzygy. Starchild already in syzygy can move into the same
(or the other) alignment, but cannot resurrect any piece. To be able to resurrect,
Starchild has to move out of alignment in a first move, and then on a next move
it can move into resurrecting syzygy.

\clearpage % ..........................................................

\subsubsection*{Cascading syzygy}
\addcontentsline{toc}{subsubsection}{Cascading syzygy}
\label{sec:One/Starchild/Syzygy/Cascading syzygy}

\vspace*{-1.2\baselineskip}
\noindent
\begin{figure}[!h]
\includegraphics[width=1.0\textwidth, keepaspectratio=true]{examples/22_o/scn_o_56_syzygy_starchild_cascading.png}
\caption{Starchild cascading}
\label{fig:scn_o_56_syzygy_starchild_cascading}
\end{figure}

Starchild moving into syzygy can activate pieces on a destination syzygy-field,
and start a cascade; this move does not grant option to resurrect. To be able to
resurrect, Starchild has to move onto empty syzygy-field, from a field out of
any alignment.

\clearpage % ..........................................................

\subsubsection*{Double syzygy}
\addcontentsline{toc}{subsubsection}{Double syzygy}
\label{sec:One/Starchild/Syzygy/Double syzygy}

\vspace*{-1.2\baselineskip}
\noindent
\begin{figure}[!h]
\includegraphics[width=1.0\textwidth, keepaspectratio=true]{examples/22_o/scn_o_57_two_syzygies_shared_celestial_piece.png}
\caption{Shared celestial piece}
\label{fig:scn_o_57_two_syzygies_shared_celestial_piece}
\end{figure}

Starchild initiating syzygy can activate another Starchild, creating double
(triple, ...) syzygies in the same move; only the very last Starchild in a
cascade can resurrect a captured piece, if it stepped onto an empty syzygy-field.
This is so regardless if syzygies share celestial piece (pictured here), just a
syzygy-field, or are completely independent.

\clearpage % ..........................................................

\subsubsection*{Opponent's Starchild}
\addcontentsline{toc}{subsubsection}{Opponent's Starchild}
\label{sec:One/Starchild/Syzygy/Opponent's Starchild}

\vspace*{-1.2\baselineskip}
\noindent
\begin{figure}[!h]
\includegraphics[width=1.0\textwidth, keepaspectratio=true]{examples/22_o/scn_o_58_syzygy_opponents_starchild.png}
\caption{Opponent's Starchild in syzygy}
\label{fig:scn_o_58_syzygy_opponents_starchild}
\end{figure}

Opponent's Starchild can be activated, and pushed into syzygy, this will also
initiate resurrection. Here, light player is moving dark Starchild into syzygy.

Any Starchild can resurrect any captured piece, regardless of their respective
colors. Here, light player can resurrect any captured piece, regardless if it's
light or dark.

\clearpage % ..........................................................

\subsubsection*{Star-initiated syzygy}
\addcontentsline{toc}{subsubsection}{Star-initiated syzygy}
\label{sec:One/Starchild/Syzygy/Star-initiated syzygy}

\vspace*{-1.2\baselineskip}
\noindent
\begin{figure}[!h]
\includegraphics[width=1.0\textwidth, keepaspectratio=true]{examples/22_o/scn_o_59_syzygy_star.png}
\caption{Star-initiated syzygy}
\label{fig:scn_o_59_syzygy_star}
\end{figure}

Pushing Star into a syzygy brings no additional interactions, i.e. it neither can
demote own figure to Pawn, nor it can resurrect any captured piece.

% -------------------------------------------------------------- Syzygy
\clearpage % ..........................................................
% Starchild and castling ----------------------------------------------

\subsection*{Starchild and castling}
\addcontentsline{toc}{subsection}{Starchild and castling}
\label{sec:One/Starchild/Starchild and castling}

\vspace*{-0.4\baselineskip}
Starchild is transparent, and so it does not block castling when it's positioned
between castling pieces and their destination fields.
\hyperref[fig:scn_mv_11_wave_block_castling_rook]{The same as Wave},
Starchild cannot be activated by castling pieces; so Starchild blocks both King
and Rook from castling onto a field it occupies.

% \clearpage % ..........................................................

\vspace*{-1.1\baselineskip}
\subsubsection*{Castling is not blocked}
\addcontentsline{toc}{subsubsection}{Castling is not blocked}
\label{sec:One/Starchild/Starchild and castling/Castling is not blocked}

\vspace*{-1.3\baselineskip}
\noindent
\begin{figure}[!h]
\includegraphics[width=1.0\textwidth, keepaspectratio=true]{examples/22_o/scn_o_60_king_castling_not_blocked.png}
\vspace*{-1.4\baselineskip}
\caption{Castling King not blocked}
\label{fig:scn_o_60_king_castling_not_blocked}
\end{figure}

\vspace*{-1.3\baselineskip}
\noindent
\begin{figure}[!h]
\includegraphics[width=1.0\textwidth, keepaspectratio=true]{examples/22_o/scn_o_61_rook_castling_not_blocked.png}
\vspace*{-1.4\baselineskip}
\caption{Castling Rook not blocked}
\label{fig:scn_o_61_rook_castling_not_blocked}
\end{figure}

\vspace*{-0.4\baselineskip}
In example at top, King can choose to castle to any enumerated field, except
to field occupied by Starchild. Above, King castled two fields past Starchild,
leaving empty field for castling Rook, so the whole castling move is legal.

% \clearpage % ..........................................................

\vspace*{-1.1\baselineskip}
\subsubsection*{Castling is blocked}
\addcontentsline{toc}{subsubsection}{Castling is blocked}
\label{sec:One/Starchild/Starchild and castling/Castling is blocked}

\vspace*{-1.3\baselineskip}
\noindent
\begin{figure}[!h]
\includegraphics[width=1.0\textwidth, keepaspectratio=true]{examples/22_o/scn_o_62_castling_blocked.png}
\vspace*{-1.4\baselineskip}
\caption{Castling blocked}
\label{fig:scn_o_62_castling_blocked}
\end{figure}

\vspace*{-0.4\baselineskip}
Castling Rook always ends on a field immediately next to King, closer to its starting
position (here, field K). Here, King castled immediately next to Starchild, so there
is no room for castling Rook. Starchild cannot be activated by castling pieces, so
it blocks the complete move.

% ---------------------------------------------- Starchild and castling
% *********************************************************** Starchild
\clearpage % ..........................................................

\section*{Rush, en passant}
\addcontentsline{toc}{section}{Rush, en passant}
\label{sec:One/Rush, en passant}

\vspace*{-1.4\baselineskip}
\noindent
\begin{figure}[!h]
\includegraphics[width=1.0\textwidth, keepaspectratio=true]{en_passants/22_one_en_passant.png}
\vspace*{-1.3\baselineskip}
\caption{En passant}
\label{fig:22_one_en_passant}
\end{figure}

\vspace*{-0.5\baselineskip}
Image above have 6 examples presented in parallel: one for each Pawns A, B,
Scouts C, D, and Grenadiers E, F.

Rush and en passant are identical to those in
\hyperref[fig:14_hemera_s_dawn_en_passant]{Hemera's Dawn variant}.
Own privates (i.e. Pawns, Scouts, and Grenadiers) can be rushed for up to 11
fields in this variant.

\clearpage % ..........................................................

\section*{Promotion}
\addcontentsline{toc}{section}{Promotion}
\label{sec:One/Promotion}

\vspace*{-0.3\baselineskip}
Promotion is non enforced, delayed variety, i.e. it's the same as in
\hyperref[sec:Age of Aquarius/Promotion]{previous chess variant}, Age of Aquarius.

Additionally, promotion in this variant is monogamous. Only one Queen
in the same color can be present on chessboard at any given time.

\hyperref[fig:scn_n_22_only_one_queen]{Similarly to previous variant, Nineteen},
if own Queen is present on a chessboard, opponent's Queen cannot be
converted, and has to be captured instead.

% \clearpage % ..........................................................

\vspace*{-0.3\baselineskip}
\section*{Castling}
\addcontentsline{toc}{section}{Castling}
\label{sec:One/Castling}

\vspace*{-0.3\baselineskip}
Castling is
\hyperref[sec:Nineteen/Castling]{the same as in Nineteen variant},
only difference is that King can move
between 2 and 10 fields across. All other constraints from Nineteen variant still
applies.

\vspace*{-0.7\baselineskip}
\noindent
\begin{figure}[!h]
\includegraphics[width=1.0\textwidth, keepaspectratio=true]{castlings/22_o/one_castling.png}
\vspace*{-1.4\baselineskip}
\caption{Castling}
\label{fig:one_castling}
\end{figure}

\vspace*{-0.3\baselineskip}
In example above, all valid King's castling moves are numbered.

\vspace*{-0.7\baselineskip}
\noindent
\begin{figure}[!h]
\includegraphics[width=1.0\textwidth, keepaspectratio=true]{castlings/22_o/one_castling_right_04.png}
\vspace*{-1.4\baselineskip}
\caption{Castling short right}
\label{fig:one_castling_right_04}
\end{figure}

\vspace*{-0.3\baselineskip}
In this example King was castling short to the right. Initial King's position is
marked with "K". After castling is finished, right Rook ends up at field immediately
left to the King.

\clearpage % ..........................................................

\section*{Initial setup}
\addcontentsline{toc}{section}{Initial setup}
\label{sec:One/Initial setup}

Compared to initial setup of Discovery, Starchild is inserted between Centaur and
Serpent symmetrically, on both sides of chessboard. This can be seen in the image
below:

\noindent
\begin{figure}[h]
\includegraphics[width=1.0\textwidth, keepaspectratio=true]{boards/22_one.png}
\caption{One board}
\label{fig:22_one}
\end{figure}

\clearpage % ..........................................................
% ========================================================= One chapter
