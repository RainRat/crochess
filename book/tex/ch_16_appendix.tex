
% Copyright (c) 2015 - 2020 Mario Mlačak, mmlacak@gmail.com
% Licensed and published as Public Domain work.

% Appendix chapter ----------------------------------------------------
\chapter*{Appendix}
\addcontentsline{toc}{chapter}{Appendix}
\label{ch:Appendix}

Appendix contains description of algebraic notation, extended from the base described in detail here: \\
\href{https://en.wikipedia.org/wiki/Algebraic\_notation\_(chess)}{https://en.wikipedia.org/wiki/Algebraic\_notation\_(chess)}. \\
Description mostly covers short notation, and is writen in monospace font, e.g. \alg{Nc3}.

Parts of classic notation clashes with new developments, and so had to be covered with
exceptions made specificaly for Classical Chess, so that algebraic notation retains
compatibility with its classic form. These exceptions are writen in monospace italics,
e.g. \algc{Nxb3}.

For instance, \alg{0-0}, \alg{O-O} and their Queen's side siblings for castling had
to go in extended algebraic notation, since there are multiple castling choices available.
Another example, \alg{x} as annotation for a capturing move, e.g. \alg{Nxv3}, since
this might also be interpreted as disambiguation.

\clearpage % ..........................................................

\section*{Variants}
\addcontentsline{toc}{section}{Variants}
\label{sec:Appendix/Variants}

\begin{table}[!h]
\centering
\begin{tabular}{ rll }
\toprule
\textbf{Id} & \textbf{Variant} & \textbf{Contains}                  \\
\midrule
\multirow{6}{*}{1} & \multirow{6}{*}{Classical chess} & Pawn        \\
                   &                                  & Knight      \\
                   &                                  & Bishop      \\
                   &                                  & Rook        \\
                   &                                  & Queen       \\
                   &                                  & King        \\ \cmidrule{1-3}
                 2 & Croatian Ties                    & Pegasus     \\
                 3 & Mayan Ascendancy                 & Pyramid     \\
                 4 & Age of Aquarius                  & Unicorn     \\
                 5 & Miranda's Veil                   & Wave        \\
                 6 & Nineteen                         & Star        \\
                 7 & Hemera's Dawn                    & Centaur     \\
                 8 & Tamoanchan Revisited             & Serpent     \\
                 9 & Conquest of Tlalocan             & Shaman      \\
                10 & Discovery                        & Monolith    \\
                11 & One                              & Starchild   \\
\bottomrule
\end{tabular}
\caption{Variants}
\label{tbl:Appendix/Variants}
\end{table}

Each new variant contains all previously introduced pieces. For instance, Age of Aquarius
beside Unicorn also contains Pyramid and Pegasus, on top of all classical pieces.

\clearpage % ..........................................................

\section*{Chessboards}
\addcontentsline{toc}{section}{Chessboards}
\label{sec:Appendix/Chessboards}

\begin{table}[!h]
\centering
\begin{tabular}{ rlrrcrr }
\toprule
\textbf{Id} & \textbf{Variant}      & \multicolumn{2}{c}{ \textbf{Ranks} } & ~ & \multicolumn{2}{c}{ \textbf{Files} }   \\ \cmidrule{3-4} \cmidrule{6-7}
            &                       & \emph{min} & \emph{max}              &   & \emph{min} & \emph{max}                \\
\midrule
          1 & Classical chess       & 1          &  8                      &   & a          & h                         \\ % \cmidrule{1-6}
          2 & Croatian Ties         & 1          & 10                      &   & a          & j                         \\
          3 & Mayan Ascendancy      & 1          & 12                      &   & a          & l                         \\
          4 & Age of Aquarius       & 1          & 14                      &   & a          & n                         \\
          5 & Miranda's Veil        & 1          & 16                      &   & a          & p                         \\
          6 & Nineteen              & 1          & 18                      &   & a          & r                         \\
          7 & Hemera's Dawn         & 1          & 20                      &   & a          & t                         \\
          8 & Tamoanchan Revisited  & 1          & 22                      &   & a          & v                         \\
          9 & Conquest of Tlalocan  & 1          & 24                      &   & a          & x                         \\
         10 & Discovery             & 1          & 24                      &   & a          & x                         \\
         11 & One                   & 1          & 26                      &   & a          & z                         \\
\bottomrule
\end{tabular}
\caption{Chessboards}
\label{tbl:Appendix/Chessboards}
\end{table}

Positions on a chessboard are writen the same as in base algebraic notation, column + row,
e.g. \alg{m2} is initial position of light Pawn in Nineteen variant.

\clearpage % ..........................................................

\section*{Pieces}
\addcontentsline{toc}{section}{Pieces}
\label{sec:Appendix/Pieces}

\begin{table}[!h]
\centering
\begin{tabular}{ rlll }
\toprule
\textbf{Id} & \textbf{Piece} & \textbf{Symbol} & \textbf{Introduced in}           \\
\midrule
1           & Pawn           & P               & \multirow{6}{*}{Classical chess} \\
2           & Knight         & N               &                                  \\
3           & Bishop         & B               &                                  \\
4           & Rook           & R               &                                  \\
5           & Queen          & Q               &                                  \\
6           & King           & K               &                                  \\ \cmidrule{1-4}
7           & Pegasus        & G               & Croatian Ties                    \\
8           & Pyramid        & A               & Mayan Ascendancy                 \\
9           & Unicorn        & U               & Age of Aquarius                  \\
10          & Wave           & W               & Miranda's Veil                   \\
11          & Star           & T               & Nineteen                         \\
12          & Centaur        & C               & Hemera's Dawn                    \\
13          & Serpent        & S               & Tamoanchan Revisited             \\
14          & Shaman         & H               & Conquest of Tlalocan             \\
15          & Monolith       & M               & Discovery                        \\
16          & Starchild      & I               & One                              \\
\bottomrule
\end{tabular}
\caption{Pieces}
\label{tbl:Appendix/Pieces}
\end{table}

Each piece is present in variant in which it is introduced, and all subsequent ones.
For example, Shaman is introduced in Conquest of Tlalocan variant, so it's also present
in succeeding variants, Discovery and One.

\clearpage % ..........................................................

\huge{TODO}
Serpent, Monolith movement limits --\textgreater table
\normalsize{}

\clearpage % ..........................................................

\section*{Movement of Wave}
\addcontentsline{toc}{section}{Movement of Wave}
\label{sec:Appendix/Movement of Wave}

Movement of Wave ... as multi-step piece activating it  ...

% TODO :: table <activated-by> : <moves-like>

\clearpage % ..........................................................

\section*{Initial setups}
\addcontentsline{toc}{section}{Initial setups}
\label{sec:Appendix/Initial setups}
Initial setups ...

% Initial setup for Light player is (mirrored for Dark one):
% \texttt{PPPPPPPPPP \\
%         RGNBQKBNGR}, \\
% or more conveniently, as seen in this image:

% Initial setup for Light player is (mirrored for Dark one):
% \texttt{PPPPPPPPPPPP \\
%         RGANBQKBNAGR}, \\
% or more conveniently, as seen in this image:

% Initial setup for Light player is (mirrored for Dark one):
% \texttt{PPPPPPPPPPPPPP \\
%         RGAUNBQKBNUAGR}, \\
% or more conveniently, as seen in this image:

% Initial setup for Light player is (mirrored for Dark one):
% \texttt{PPPPPPPPPPPPPPPP \\
%         RGAUWNBQKBNWUAGR}, \\
% or more conveniently, as seen in this image:

% Initial setup for Light player is (mirrored for Dark one):
% \texttt{PPPPPPPPPPPPPPPPPP \\
%         TRGAUWNBQKBNWUAGRT}, \\
% or more conveniently, as seen in this image:

% Initial setup for Light player is (mirrored for Dark one):
% \texttt{PPPPPPPPPPPPPPPPPPPP \\
%         TRGAUWCNBQKBNCWUAGRT}, \\
% or more conveniently, as seen in this image:

% Initial setup for Light player is (mirrored for Dark one):
% \texttt{PPPPPPPPPPPPPPPPPPPPPP \\
%         TRGAUWCSNBQKBNSCWUAGRT}, \\
% or more conveniently, as seen in this image:

% Initial setup for Light player is (mirrored for Dark one):
% \texttt{PPPPPPPPPPPPPPPPPPPPPPPP \\
%         TRGAHUWCSNBQKBNSCWUHAGRT}, \\
% or more conveniently, as seen in this image:

% Initial setup for Light player is (mirrored for Dark one):
% \texttt{PPPPPPPPPPPPPPPPPPPPPPPP \\
%         TRGAHUWCSNBQKBNSCWUHAGRT}, \\
% or more conveniently, as seen in this image:

% Initial setup for Light player is (mirrored for Dark one):
% \texttt{PPPPPPPPPPPPPPPPPPPPPPPPPP \\
%         TRGAHIUWCSNBQKBNSCWUIHAGRT}, \\
% or more conveniently, as seen in this image:

% TODO :: add to "Pieces" section
%
% \section*{Passive pieces}
% \addcontentsline{toc}{section}{Passive pieces}
% \label{sec:Definitions/Passive pieces}
%
% Passive pieces are ...
%
% Activating passive piece with Pawn ... capture-fields vs. step-fields.

\clearpage % ..........................................................
% ---------------------------------------------------- Appendix chapter
