
% Copyright (c) 2015 - 2021 Mario Mlačak, mmlacak@gmail.com
% Public Domain work, under CC0 1.0 Universal Public Domain Dedication. See LICENSING, COPYING files for details.

% Appendix chapter ----------------------------------------------------
\chapter*{Appendix}
\addcontentsline{toc}{chapter}{Appendix}
\label{ch:Appendix}

Appendix contains description of algebraic notation, extended from the base described here:\newline
\href{https://en.wikipedia.org/wiki/Algebraic\_notation\_(chess)}{https://en.wikipedia.org/wiki/Algebraic\_notation\_(chess)}.\newline
This description mostly covers short notation, and is written in monospace font, e.g. \alg{Nc3}.

Parts of classic notation clashes with new developments, and so had to be covered with
exceptions made specifically for Classical Chess, so that algebraic notation retains
compatibility with its classic form. These exceptions are written in monospace italics,
e.g. \algcty{Nxb3}.

For instance, \algcty{0-0}, \algcty{O-O} and their Queen's side siblings for castling had
to go in extended algebraic notation, since there are multiple castling choices available.
Another example, \algcty{x} as annotation for a capturing move, e.g. \algcty{Nxv3}, since
this might also be interpreted as disambiguation.

\clearpage % ..........................................................

\section*{Introduction}
\addcontentsline{toc}{section}{Introduction}
\label{sec:Appendix/Introduction}

\begin{table}[!h]
\centering
\begin{tabular}{ lll }
\toprule % ========================================================
\textbf{Symbol}      & \textbf{Description}                      \\
\midrule % --------------------------------------------------------
\algfmt{AN}          & algebraic notation, in general            \\
\cmidrule{1-2} % ..................................................
\algfmt{CAN}         & classic \algfmt{AN}, as described by FIDE \\
                     & handbook and Wikipedia, can be            \\
                     & long, short or minimal                    \\
\algfmt{LAN}         & classic \algfmt{AN}, long form            \\
\algfmt{SAN}         & classic \algfmt{AN}, short form           \\
\algfmt{MAN}         & classic \algfmt{AN}, minimal form         \\
\cmidrule{1-2} % ..................................................
\algfmt{NAN}         & new, extended \algfmt{AN},                \\
                     & can be long or short                      \\
\algfmt{EAN}         & new, extended \algfmt{AN}, short form     \\
\algfmt{XAN}         & new, extended \algfmt{AN}, long form      \\
\cmidrule{1-2} % ..................................................
\algfmt{FIDE}        & FIDE handbook                             \\
\algfmt{FIDE point}  & point in FIDE handbook                    \\
\bottomrule % =====================================================
\end{tabular}
\caption{Abbreviations}
\label{tbl:Appendix/Introduction/Abbreviations}
\end{table}

FIDE Handbook used in this book is defined in chapter\newline
\hyperlink{sec:Prerequisites/FIDE Handbook}{Prerequisites}, as is
\hyperlink{sec:Prerequisites/FIDE point}{\algfmt{FIDE point}}.

Here, \algfmt{CAN} is used to indicate compatibility with Classical Chess notation, even if
examples are written on chessboards for other variants. \algfmt{CAN} almost always means
short notation, and only occasionally long, if appropriate.

\clearpage % ..........................................................

\subsection*{Variants}
\addcontentsline{toc}{subsection}{Variants}
\label{sec:Appendix/Introduction/Variants}

\begin{table}[!h]
\centering
\begin{tabular}{ llc }
\toprule % ===================================================================
\textbf{Variant}                 & \textbf{Pieces}   & \textbf{Delay?}      \\
\midrule % -------------------------------------------------------------------
\multirow{6}{*}{Classical Chess} & Pawn              & \multirow{6}{*}{-}   \\
                                 & Knight            &                      \\
                                 & Bishop            &                      \\
                                 & Rook              &                      \\
                                 & Queen             &                      \\
                                 & King              &                      \\
\cmidrule{1-3} % .............................................................
Croatian Ties                    & Pegasus           & -                    \\
Mayan Ascendancy                 & Pyramid           & -                    \\
Age of Aquarius                  & Unicorn           & +                    \\
Miranda's Veil                   & Wave              & +                    \\
Nineteen                         & Star              & +                    \\
\cmidrule{1-3} % .............................................................
\multirow{3}{*}{Hemera's Dawn}   & Centaur           & \multirow{3}{*}{+}   \\
                                 & Scout             &                      \\
                                 & Grenadier         &                      \\
\cmidrule{1-3} % .............................................................
Tamoanchan Revisited             & Serpent           & +                    \\
Conquest of Tlalocan             & Shaman            & -                    \\
Discovery                        & Monolith          & +                    \\
One                              & Starchild         & +                    \\
\bottomrule % ================================================================
\end{tabular}
\caption{Variants}
\label{tbl:Appendix/Introduction/Variants}
\end{table}

Each new variant contains all previously introduced pieces. For instance, Age of
Aquarius beside Unicorn also contains Pyramid and Pegasus, on top of all classical
pieces.

\emph{Delay?} column holds \emph{+} if a variant supports
\hyperref[sec:Age of Aquarius/Promotion]{delayed promotion}
(i.e. if Pawns can be tagged for later promotion); holds \emph{-} otherwise.

\clearpage % ..........................................................

\subsection*{Chessboards}
\addcontentsline{toc}{subsection}{Chessboards}
\label{sec:Appendix/Introduction/Chessboards}

\begin{table}[!h]
\centering
\begin{tabular}{ lrrcrr }
\toprule % =================================================================================================
\textbf{Variant}      & \multicolumn{2}{c}{ \textbf{Files} } & ~ & \multicolumn{2}{c}{ \textbf{Ranks} }   \\
                      \cmidrule{2-3}                             \cmidrule{5-6} % ..........................
                      & \emph{min} & \emph{max}              &   & \emph{min} & \emph{max/size}           \\
\midrule % -------------------------------------------------------------------------------------------------
Classical Chess       & a          & h                       &   & 1          &  8                        \\
% \cmidrule{1-6} % ...........................................................................................
Croatian Ties         & a          & j                       &   & 1          & 10                        \\
Mayan Ascendancy      & a          & l                       &   & 1          & 12                        \\
Age of Aquarius       & a          & n                       &   & 1          & 14                        \\
Miranda's Veil        & a          & p                       &   & 1          & 16                        \\
Nineteen              & a          & r                       &   & 1          & 18                        \\
Hemera's Dawn         & a          & t                       &   & 1          & 20                        \\
Tamoanchan Revisited  & a          & v                       &   & 1          & 22                        \\
Conquest of Tlalocan  & a          & x                       &   & 1          & 24                        \\
Discovery             & a          & x                       &   & 1          & 24                        \\
One                   & a          & z                       &   & 1          & 26                        \\
\bottomrule % ==============================================================================================
\end{tabular}
\caption{Chessboards}
\label{tbl:Appendix/Introduction/Chessboards}
\end{table}

Positions on a chessboard are written the same as in base algebraic notation, file + rank,
e.g. \alg{m2} is initial position of light Pawn in Nineteen variant.

Maximum rank on a chessboard also represents the size of that board; all chessboards in all
variants are squares. For instance, Hemera's Dawn variant is played on a chessboard with
maximum rank of 20, so board size for that variant is \mbox{20 $\times$ 20}.

\clearpage % ..........................................................

\subsection*{Pieces}
\addcontentsline{toc}{subsection}{Pieces}
\label{sec:Appendix/Introduction/Pieces}

\begin{table}[!h]
\centering
\begin{tabular}{ lcl }
\toprule % ===========================================================
\textbf{Piece} & \textbf{Symbol} & \textbf{Introduced in}           \\
\midrule % -----------------------------------------------------------
Pawn           & P               & \multirow{6}{*}{Classical Chess} \\
Knight         & N               &                                  \\
Bishop         & B               &                                  \\
Rook           & R               &                                  \\
Queen          & Q               &                                  \\
King           & K               &                                  \\
\cmidrule{1-3} % .....................................................
Pegasus        & E               & Croatian Ties                    \\
Pyramid        & A               & Mayan Ascendancy                 \\
Unicorn        & U               & Age of Aquarius                  \\
Wave           & W               & Miranda's Veil                   \\
Star           & T               & Nineteen                         \\
\cmidrule{1-3} % .....................................................
Centaur        & C               & \multirow{3}{*}{Hemera's Dawn}   \\
Scout          & O               &                                  \\
Grenadier      & G               &                                  \\
\cmidrule{1-3} % .....................................................
Serpent        & S               & Tamoanchan Revisited             \\
Shaman         & H               & Conquest of Tlalocan             \\
Monolith       & M               & Discovery                        \\
Starchild      & I               & One                              \\
\bottomrule % ========================================================
\end{tabular}
\caption{Pieces}
\label{tbl:Appendix/Introduction/Pieces}
\end{table}

Each piece is present in variant in which it is introduced, and all subsequent ones.
For example, Shaman is introduced in Conquest of Tlalocan variant, so it's also present
in succeeding variants, Discovery and One.

\clearpage % ..........................................................

\section*{Notation}
\addcontentsline{toc}{section}{Notation}
\label{sec:Appendix/Notation}

Simple movement is denoted the same way as in \algfmt{CAN}, piece (always written as upper
case) + destination field, which consists of rank (always written in lower case) + file
(a number).

In this example of \hyperref[fig:scn_ct_03_define_step_ply]{Pegasus moving to destination
field 3}, movement of the piece would be written as \alg{Ef8}. The same movement in
\algfmt{XAN}, would be written as \alg{Ec2-f8}.

When moving Pawn, symbol is omitted, so only destination field is written. In this example
of \hyperref[fig:04_croatian_ties_en_passant]{Pawn rushing to field 2}, movement can be
written as \alg{h5}. Long notation would be \alg{h2-h5}.

\subsection*{Disambiguation}
\addcontentsline{toc}{subsection}{Disambiguation}
\label{sec:Appendix/Notation/Disambiguation}

Disambiguation is position notation, shortened to minimum necessary to distinguish from
another position(s). It contains one of: just file, just rank, rank + file, in that order
of preference. This is the same as in \algfmt{CAN}, described in:\newline
\href{https://en.wikipedia.org/wiki/Algebraic\_notation\_(chess)\#Disambiguating\_moves}{https://en.wikipedia.org/wiki/Algebraic\_notation\_(chess)\newline
\#Disambiguating\_moves}, see also \algfmt{FIDE~C.10}.
Disambiguation is used in a ply, to distinguish starting position of a piece from others
of the same kind that can end their movement on the same destination field, or can share
portion of a path.

For instance, should \hyperref[fig:scn_ct_03_define_step_ply]{Pegasus simple move example}
had another light Pegasus at \alg{i2} field, move to destination field 3 would be written
as \alg{Ecf8}.

\vfill

\subsection*{Capturing}
\addcontentsline{toc}{subsection}{Capturing}
\label{sec:Appendix/Notation/Capturing}

Capturing move is denoted with \alg{*} (asterisk) at the end, usually followed by the captured
piece. Only for Classical Chess capturing is denoted with \algcty{x}, before destination field.
Here, \hyperref[fig:scn_ct_04_pegasus_movement]{Pegasus could capture opponent's Pawn}, which
would be written as \alg{Eg4*P}, or just \alg{Eg4*}, if captured piece is not needed.

In \algfmt{CAN}, the same move would be written as \algcty{Exg4}. Note, FIDE handbook requires
captures made by Pawn to contain starting file, \algcty{x}, and then destination field; see
\algfmt{FIDE~C.9.3}.\newline
\indent
If \hyperref[fig:scn_mv_023_wave_activation_by_capture_pawn]{Wave activated by Pawn example} had
dark Wave instead of light one, light Pawn would be able to capture it, which in \algfmt{CAN}
would be written as \algcty{exf4}. The same move in new notation is written as \alg{f4*}, and
if captured piece is also written \alg{f4*W}.

\subsection*{Castling}
\addcontentsline{toc}{subsection}{Castling}
\label{sec:Appendix/Notation/Castling}

Castling is noted with \alg{\&} (ampersand), after King's symbol and destination file. This
\hyperref[fig:age_of_aquarius_castling_left_04]{castling example} would be written as \alg{Kd\&},
and this \hyperref[fig:one_castling_right_04]{castling example} as \alg{Kr\&}. File at which
castling Rook ended can be written after \alg{\&}, the same examples would now be written as
\alg{Kd\&e} and \alg{Kr\&q}.

Only for Classical Chess \algcty{0-0} and \algcty{O-O} for King's side, \algcty{0-0-0} and
\algcty{O-O-O} for Queen's side are accepted as castling notation.

\vfill

\subsection*{Ply}
\addcontentsline{toc}{subsection}{Ply}
\label{sec:Appendix/Notation/Ply}

\hyperref[sec:Terms/Ply]{Ply} is a movement of a single piece in a cascading move. Two plies
are separated by \alg{\textasciitilde{}} (tilde). In the example starting with
\hyperref[fig:scn_ma_15_cascading_init]{Queen activating a Pyramid}, which then activates
another Pyramid; example is comprised of series of 4 images, each corresponding to one ply,
while last image depicts ending state. This can be written as
\alg{Qf7\textasciitilde{}Ai7\textasciitilde{}Ai9}.

In \algfmt{XAN}, the same would be written as\newline
\alg{Qk2-f7\textasciitilde{}Af7-i7\textasciitilde{}Ai7-i9}. A pair of \alg{[}, \alg{]}
(square brackets) can be used to gather each ply to help tell them apart:\newline
\alg{[Qk2-f7]\textasciitilde{}[Af7-i7]\textasciitilde{}[Ai7-i9]}.

\subsection*{Pawn promotion}
\addcontentsline{toc}{subsection}{Pawn promotion}
\label{sec:Appendix/Notation/Pawn promotion}

Pawn promotion is also written the same way as in \algfmt{CAN}, as described in detail:\newline
\href{https://en.wikipedia.org/wiki/Algebraic\_notation\_(chess)\#Pawn\_promotion}{https://en.wikipedia.org/wiki/Algebraic\_notation\_(chess)\newline
\#Pawn\_promotion}, with Pawn's destination field + piece to which it was promoted to,
like so: \alg{e8Q}. Inserting \alg{=} (equal sign) between field and promoted-to piece
is also supported, e.g. \alg{e8=Q}.
If \hyperref[fig:scn_aoa_05_delayed_promo_pawn_2_moved]{promotion is being delayed}, usage
of \alg{=} is mandatory, as there is no immediate piece to promote to, e.g. \alg{l14=}.

If Pawn has been promoted, after being tagged for promotion, it is promoted on the same
field at which it has been tagged. Notation in such a case is similar to normal promotion,
only field written is the one already occupied by Pawn being promoted. For instance, Pawn
tagged for promotion in previous example would have its actual promotion written as e.g.
\alg{l14Q}, or as \alg{l14=Q}.

Similarly, \hyperref[fig:scn_ma_05_promo_init]{Pawn promoted by own Pyramid} just writes
chosen piece to promote to, after writing movement of a Pyramid, like so
\alg{Ed8\textasciitilde{}Ah8Q}, or in \algfmt{XAN} as
\alg{[El4-d8]\textasciitilde{}[Ad8-h8=Q]}.

\subsection*{En passant}
\addcontentsline{toc}{subsection}{En passant}
\label{sec:Appendix/Notation/En passant}

En passant is denoted with \alg{:} (colon), after destination field. In this
\hyperref[fig:04_croatian_ties_en_passant]{en passant example} dark Pawn on the right might
capture light Pawn if rushed, which is written as \alg{h3:}. Rank of captured Pawn can be
written after \alg{:}, so our example might now be \alg{h3:5}, if captured Pawn has been
rushed to field 2.

If disambiguation is needed, it is written as previously described. Usually, it's enough to
add starting file before destination field. If previous example had additional dark Pawn
located at \alg{g4}, en passant would be written as \alg{ih3:}, or \alg{ih3:5}.

In \algfmt{CAN}, both en passant and its disambiguation form are written as
\algcty{ixh3 e.p.}, where \algcty{e.p.} stands for en passant; see \algfmt{FIDE~C.9.3}.

% \vfill

\subsection*{Conversion}
\addcontentsline{toc}{subsection}{Conversion}
\label{sec:Appendix/Notation/Conversion}

Conversion is noted with \alg{\%} (percentage) after destination field. Example starting
with \hyperref[fig:scn_ma_08_conversion_init]{Bishop activating Pyramid}, which then converts
opponent's Rook is covered by 3 images, 2 corresponding to 2 plies, and last one is for ending
state. This is written as \alg{Bd5\textasciitilde{}Ah5\%}. Optionally, converted piece can be
written after \alg{\%} symbol, so it would be \alg{Bd5\textasciitilde{}Ah5\%R}. In \algfmt{XAN},
it would be \alg{Bh9-d5\textasciitilde{}Ad5-h5\%}. With both plies gathered and converted piece
noted it would be \alg{[Bh9-d5]\textasciitilde{}[Ad5-h5\%R]}.

Starchild is immune to conversion, Pyramid attempting such a thing is
\hyperref[sec:Terms/Oblation]{oblationed}. Failed conversion is noted with \alg{\%\%}
(double percentage) after destination field. This example of
\hyperref[fig:scn_o_17_starchild_conversion_immunity_init]{conversion immunity} would be
written as \alg{Bl23\textasciitilde{}Ah23\%\%}. In \algfmt{XAN}, with ply gathering,
it would be \alg{[Bs16-l23]\textasciitilde{}[Al23-h23\%\%]}.

\subsection*{Complex movement}
\addcontentsline{toc}{subsection}{Complex movement}
\label{sec:Appendix/Notation/Complex movement}

Individual steps are separated by \alg{.} (single dot), multiple steps are separated by
\alg{..} (two dots). In this example,
\hyperref[fig:scn_hd_05_centaur_multi_step]{Centaur has to choose 2 different steps},
which it will then follow for the rest of ply. Lets say that destination is field 8,
writing it as just \alg{Cp15} is not good enough since at least 2 different paths lead
to the same destination field.

The best way to write it is with both initial steps, i.e. \alg{C.c5.g6..p15}, because
this is exactly definition of such a movement, and will contain no ambiguity. Sometimes,
it might be enough if only first step is written, i.e. \alg{C.c5..p15}. The one of other
paths leading to the same \alg{p15} field would be \alg{C.f2.g6..p15}. Note also \alg{.}
separating piece and the first step, without it first step would be taken as an initial
field.

Not recommended, but still possible is to write \emph{some} step along the way, e.g.
\alg{C..i11..p15}. Care must be taken to write step which really differentiate paths,
otherwise noted path might inadvertently also denote another. For instance, in addition
to original path, \alg{C..j9..p15} might also denote \alg{C.b4.f5..p15}, which happens
to cross \alg{j9} as well.

In \algfmt{XAN}, example would be best written as \alg{Cd3.c5.g6-p15}. Depending on a
situation, it's might be possible to drop either starting field, or one of initial steps,
like so \alg{Cd3.c5-p15}, while keeping path unique.

\subsection*{Capturing-ply}
\addcontentsline{toc}{subsection}{Capturing-ply}
\label{sec:Appendix/Notation/Capturing-ply}

Shaman can capture multiple pieces in one capturing-ply.
\hyperref[fig:scn_cot_004_light_shaman_capture_ply]{In this example} capture-ply just above
horizontal line would be written as \alg{H.h10*.l11*.p12\textasciitilde{}Wn8}, if activated
Wave is moved down, to the right. In \algfmt{XAN} (with starting field, captured pieces and
plies gathered), it would be \alg{[Hd9.h10*P.l11*P.p12]\textasciitilde{}[Wp12-n8]}.

\subsection*{Transparency}
\addcontentsline{toc}{subsection}{Transparency}
\label{sec:Appendix/Notation/Transparency}

Passing over a piece is noted by using \alg{\^{}} (caret) after a step,
optionally followed by a piece which has been passed-over.

For instance, in \hyperref[fig:scn_mv_007_wave_is_transparent]{this example} light
Queen could capture dark Pegasus, which would be written as\newline
\alg{Q..m4\^{}..j7\^{}..g10*}, or in \algfmt{XAN} as\newline
\alg{Qo2..m4\^{}W..j7\^{}W..g10*E}.

\subsection*{Divergence}
\addcontentsline{toc}{subsection}{Divergence}
\label{sec:Appendix/Notation/Divergence}

Diverging a piece is noted by using \alg{/} (slash) after a step, optionally followed
by divergent piece.

For instance, in \hyperref[fig:scn_cot_030_own_shaman_is_divergent_init]{this example}
light Queen could capture dark Pegasus, after diverging from light Shaman.
This could be written as \alg{Q..j7/..m10*}, or in \algfmt{XAN} as \alg{Qo2..j7/H..m10*E}.

Note, diverging from opponent's Starchild is mandatory to write, otherwise it
can be assumed Starchild has been captured. If interaction is not written,
capturing opponent's piece is default action. This is the reason too why
\hyperref[sec:Appendix/Notation/Conversion]{conversion of opponent's pieces}
are also mandatory to write.\newline
\indent
For instance, in
\hyperref[fig:scn_o_31_starchild_divergence_end]{this example} light Queen
could either capture or diverge from either first or second dark Starchild,
due to almost complete
\hyperref[fig:scn_o_15_starchild_is_transparent]{Starchild transparency}.
If diverging from a second dark Starchild, and capturing Pegasus, it would
be written as \alg{Q..k6/..m8*}, or in \algfmt{XAN} with additional transparency
as \alg{Qo2..m4\^{}I..k6/I..m8*E}.

\subsection*{Displacement}
\addcontentsline{toc}{subsection}{Displacement}
\label{sec:Appendix/Notation/Displacement}

Displacing a piece is noted by using \alg{<} (less-than) after a step, followed by
displacement field. Optionally, displaced piece can be written before displacement
field.

For instance, in \hyperref[fig:scn_tr_19_displacement_init]{this example}, Serpent ends
its ply with two Pawns displaced. This could be written as \alg{S..e5<f5..f4<f3..h4},
or in \algfmt{XAN} as\newline
\alg{Sb4..e5<Pf5..f4<Pf3..h4}.

\subsection*{Trance-journey}
\addcontentsline{toc}{subsection}{Trance-journey}
\label{sec:Appendix/Notation/Trance-journey}

Trance-journey is noted with \alg{@} (at sign), instead of normal ply separator
\alg{\textasciitilde{}} (tilde), before
\hyperref[fig:scn_cot_072_entrancement_step]{entranced piece} takes off. This
\hyperref[fig:scn_cot_086_light_light_shaman_interaction_start]{trance-journey example},
if without any interactions with pieces on entranced Shaman's step-fields, would be
written as \alg{Hd12@Hg24}. In \algfmt{XAN}, it would be \alg{[Hd12-e13]@[He13-g24]}.

Displacements are noted by writing \alg{<} (less-than) immediately after step in
which a piece was reached, followed by field of displacement. Optionally, displaced
piece can be written before displacement-field. The same
\hyperref[fig:scn_cot_087_light_light_shaman_interaction_end]{trance-journey example},
now with all interactions taken into account, would be written as
\alg{He13@H..e18<i14..m12<j17..g24}.
In \algfmt{XAN}, with gathered plies and displaced pieces it would be\newline
\alg{[Hd12-e13]@[He13..e18<Ni14..m12<Pj17..g24]}.

Forward-displacement is written the same as ordinary trance-journey displacement,
except entranced Shaman visits a field onto which a piece has been displaced in
previous step, and so the same piece can be displaced again. This sharing of the
same displacement- and step-field can be seen in
\hyperref[fig:scn_cot_098_forward_displacement_start]{an example} which would be
written as \alg{Hu8@H..o10<i12..i12<c14..c14<a16}, or in \algfmt{XAN} as\newline
\alg{[Ht9-u8]@[Hu8..o10<Ri12..i12<Rc14..c14<Ra16]}, if final displacement of
a Rook was chosen to be \alg{a16}.

Captures are noted by writing \alg{*} (asterisk) immediately after step in which
a piece is reached, optionally followed by a captured piece. This
\hyperref[fig:scn_cot_089_dark_light_shaman_interaction_end]{trance-journey example with captures}
would be written as\newline
\alg{He13@H..e18*..m12*..g24}. In \algfmt{XAN}, with gathered plies and captured
pieces it would be\newline
\alg{[Hf12-e13]@[He13..e18*N..m12*P..g24]}.

One peculiarity of dark Shaman's trance-journey is that it starts from the far end
of a pattern inward, towards its initial position. Still, dark Shaman's trance-journey
is noted similar to light's one. For instance, this
\hyperref[fig:scn_cot_090_dark_dark_shaman_interaction_start]{dark Shaman's trance-journey}
would be written as \alg{He12@H..q16*..k14*..c18}.
There is no step between Shaman's initial position and distant starting field of
trance-journey, \alg{\textbackslash} (backslash) is used to separate them, like so
\alg{He12@He12\textbackslash{}w18..q16*..k14*..c18}.
If initial position is omitted, separator (i.e. backslash) is still written, like so
\alg{He12@H\textbackslash{}w18..q16*..k14*..c18}. Now, in \algfmt{XAN} with gathered
plies and noted captured pieces it would look like so\newline
\alg{[Hd13-e12]@[He12\textbackslash{}w18..q16*P..k14*N..c18]}.

Another peculiarity of dark Shaman is dual trance-journey, which is written with \alg{@@}
(double at-sign), optionally followed by list of captured pieces, each separated by \alg{,}
(comma). Each piece can optionally be followed by location where it was captured. Order of
captured pieces in a list is not important. This example of
\hyperref[fig:scn_cot_092_dark_dark_shaman_double_interaction_start]{dark Shaman's dual trance-journey}
is written just as \alg{He12@@} or, with captured pieces, as
\alg{He12@@P,B,R,R,N,B,N}. In \algfmt{XAN}, with gathered plies and capturing locations noted,
it would be written as\newline
\alg{[Hd13-e12]@@}\newline
\alg{Pq16,Bp14,Rd20,Rg6,Nk14,Bj12,Nd10}.\newline
% \alg{[Hd13-e12]@@Pq16,Bp14,Rd20,Rg6,Nk14,Bj12,Nd10}.
Note, sacrificed entranced dark Shaman is \emph{not} to be written in a list of captured
pieces.

Failed trance-journey is noted with \alg{@@@} (triple at-sign) after entrancing ply.
Optionally, oblationed piece can be written after \alg{@@@}. In this
\hyperref[fig:scn_o_36_trance_journey_failed]{failed trance-journey example} all
step-fields are blocked, so entranced light Shaman is oblationed, which is written as
\alg{Hz1@@@}, and in \algfmt{XAN}, with plies gathered and oblationed piece, as
\alg{[Hy2-z1]@@@H}.

\subsection*{Sense-journey}
\addcontentsline{toc}{subsection}{Sense-journey}
\label{sec:Appendix/Notation/Sense-journey}

Sense-journey is noted with \alg{"} (quotation mark), instead of normal ply separator
\alg{\textasciitilde{}} (tilde), before \hyperref[fig:scn_o_42_uplifting_step]{uplifted piece} takes
off. This \hyperref[fig:scn_o_45_dark_piece_sense_journey]{sense-journey example} would
be written as \alg{Ne1\textasciitilde{}Wc3\textasciitilde{}Hd4\textasciitilde{}Ic5"Bn7}.
In \algfmt{XAN}, it would be
\alg{[Nf4-e1]\textasciitilde{}[We1-c3]\textasciitilde{}[Hc3-d4]\textasciitilde{}[Id4-c5]"[Bc5-n7]}.

One peculiarity of sense-journey taken by uplifted dark piece is that it starts from
the far end of a pattern inward, towards its initial position. There is no step between
initial position of uplifted dark piece and distant starting field of sense-journey,
\alg{\textbackslash} (backslash) is used to separate them, like so
\alg{Ne1\textasciitilde{}Wc3\textasciitilde{}Hd4\textasciitilde{}Ic5"Bc5\textbackslash{}z11..n7}.
If initial position is omitted, separator (i.e. backslash) is still written, like so
\alg{Ne1\textasciitilde{}Wc3\textasciitilde{}Hd4\textasciitilde{}Ic5"B\textbackslash{}z11..n7}.
Now, in \algfmt{XAN} with gathered plies and noted initial, and starting positions it would look
like so\newline
\alg{[Nf4-e1]\textasciitilde{}[We1-c3]\textasciitilde{}[Hc3-d4]\textasciitilde{}[Id4-c5]"}\newline
\alg{[Bc5\textbackslash{}z11-n7]}.
% \alg{[Nf4-e1]\textasciitilde{}[We1-c3]\textasciitilde{}[Hc3-d4]\textasciitilde{}[Id4-c5]"[Bc5,z11-n7]}.

Failed sense-journey is noted with \alg{'} (single apostrophe) after uplifting a
piece. Optionally, oblationed piece can be written after \alg{'}. In this
\hyperref[fig:scn_o_46_sense_journey_failed]{failed sense-journey example} all
step-fields are blocked, so uplifted dark Bishop is oblationed, which is written as
\alg{Id4\textasciitilde{}Ic5'}, and in \algfmt{XAN}, with plies gathered and oblationed
piece, as \alg{[Ic3-d4]\textasciitilde{}[Id4-c5]'B}.

\vfill

\subsection*{Syzygy, demoting to Pawn}
\addcontentsline{toc}{subsection}{Syzygy, demoting to Pawn}
\label{sec:Appendix/Notation/Syzygy, demoting to Pawn}

Demoting to Pawn is noted by writing \alg{>} (greater-than), optionally followed by
disambiguating position, i.e. one of rank, file or rank + file, in that order. Optional
disambiguation can be preceded by piece which was demoted, and demotion-field can be
written instead of just a disambiguation. If writing just demoted piece is enough to
identify which one is it, and where, demoting position does not need to be written.

In this \hyperref[fig:scn_d_22_syzygy_2_stars_init]{syzygy example}, if Monolith was
moved by light player, then either light Wave or light Bishop could be demoted to Pawn.
To be able to distinguish which one is it, either demoting position or piece has to be
written. If Bishop was chosen, that would be written as \alg{Mm12>p}. Since there is
only one light Bishop in demoting-to-Pawn syzygy, the same move could be also written
as \alg{Mm12>B}. In \algfmt{XAN}, with piece and demotion-field noted it would be
\alg{Mn14-m12>Bp8}. There is no plies gathering, since only Monolith moved, so there
is only one ply.

\subsection*{Syzygy, resurrection}
\addcontentsline{toc}{subsection}{Syzygy, resurrection}
\label{sec:Appendix/Notation/Syzygy, resurrection}

% \TODO :: fix lmodern

Resurrection is written by appending \alg{\$} (dollar sign) after the move, followed
by piece which was resurrected. If Wave or Starchild has been resurrected on an empty
miracle-field, position is appended after the piece.

If resurrecting opponent's piece, \alg{\$\$} (double dollar) sign is appended after
the move, followed by a piece to resurrect. If opponent's Wave or Starchild was
resurrected on empty miracle-field, position is appended after the piece.

Resurrection is always optional, and does not have to be performed, which could
be written by appending \alg{\$\$\$} (triple dollar) after the ply. Since there
are no actual side-effects to failed resurrection, \alg{\$\$\$} is optional.

If \hyperref[fig:scn_o_51_syzygy_starchild_init]{resurrection by light player} ended by
\hyperref[fig:scn_o_52_syzygy_starchild_end]{resurrecting own, light Queen}, it would be
written as \alg{It9\$Q}, and in \algfmt{XAN} it would be \alg{Ii23-t9\$Q}.

If, in the same example light player resurrected opponent's, dark Queen instead, it would
be written as \alg{It9\$\$Q}, and in \algfmt{XAN} it would be \alg{Ii23-t9\$\$Q}.

If previous example ended by
\hyperref[fig:scn_o_53_syzygy_starchild_resurrection]{resurrecting Starchild}, it would
be written as \alg{It9\$Iu8}. In \algfmt{XAN}, it would be \alg{Ii23-t9\$Iu8}.

If all fields suitable for resurrection are occupied, it is written as \alg{It9\$\$\$},
in \algfmt{XAN} it would be \alg{Ii23-t9\$\$\$}. Writing it as \alg{It9}, or \alg{Ii23-t9}
is fine, it just doesn't note failed intention.

% \vfill % Otherwise whole section is positioned at the bottom of a page (!?).

\subsection*{Teleportation}
\addcontentsline{toc}{subsection}{Teleportation}
\label{sec:Appendix/Notation/Teleportation}

Teleportation is noted by separating plies with \alg{|} (vertical bar) instead of
usual \alg{\textasciitilde{}} (tilde), followed by field at which piece emerged.
Optionally, teleporting piece can be written before emerging field. If Wave teleported,
vertical bar is followed by Wave and its destination field, or movement, optionally
followed by activated pieces' plies, if there were any.

If piece teleported, but there was no empty portal-field, teleportation failed,
and is noted with \alg{|||} (triple vertical bar), optionally followed by oblationed
piece. The same notation is used for teleported Wave, if all step-fields are blocked,
or located off-board.

This \hyperref[fig:scn_n_02_teleport_init]{teleportation example} would be written as
\alg{Ba18|q18}, in \algfmt{XAN} it would be \alg{Bd15-a18|Bq18}.

This \hyperref[fig:scn_n_03_teleport_move_2]{blocked teleportation example} would be
written as \alg{Ra18|||}, or in \algfmt{XAN} it would be \alg{Ra13-a18|||R}.

In this \hyperref[fig:scn_n_04_teleport_move_3]{Wave teleporting example}, followed by
\hyperref[fig:scn_n_05_teleport_end]{Wave teleported example}, if activated Pyramid
would move 2 fields upward, complete move would be written as\newline
\alg{Eg15\textasciitilde{}Wa18|Wl4\textasciitilde{}Al6}. In \algfmt{XAN}, with plies
gathering, it would be\newline
\alg{[Ei11-g15]\textasciitilde{}[Wg15-a18]|[Wr1-l4]\textasciitilde{}[Al4-l6]}.

If previous example ended with
\hyperref[fig:scn_n_06_teleport_wave_blocked]{teleported Wave blocked} example,
it would be written as \alg{Eg15\textasciitilde{}Wa18|||}, and in \algfmt{XAN} it
would be \alg{[Ei11-g15]\textasciitilde{}[Wg15-a18]|||W}.

This \hyperref[fig:scn_d_18_teleporting_wave_cascade]{cascading teleportation example}
would be written by sequencing teleportations like so
\alg{Eh2\textasciitilde{}Wb4|Wa24|Wr4\textasciitilde{}Bt6}, if activated Bishop would
take upper-right diagonal. In \algfmt{XAN}, with plies gathering, it would be\newline
\alg{[Ej6-h2]\textasciitilde{}[Wh2-b4]|[Wm18-a24]|[Wx1-r4]\textasciitilde{}}\newline
\alg{[Br4-t6]}.

Starchild and Wave activated by it cannot teleport, which is written with \alg{||}
(double vertical bar), followed by destination field at which piece emerged.
Destination field can be optionally preceded by emerging piece. So,
\hyperref[fig:scn_o_07_starchild_not_moving_monolith_init]{this example} would be
written as \alg{Ic3||b3}. In \algfmt{XAN}, with ply gathering and emerging piece
it would be \alg{[If5-c3]||Ib3}.

Note, if there is no empty portal-field around Monolith (or a Star), piece is
oblationed, and is written as failed teleportation, i.e. with \alg{|||}, optionally
followed by oblationed piece. So, if Starchild would be oblationed in previous
example, it would be written as \alg{Ic3|||}, or, in \algfmt{XAN} as
\alg{[If5-c3]|||I}.

\subsection*{Pawn-sacrifice}
\addcontentsline{toc}{subsection}{Pawn-sacrifice}
\label{sec:Appendix/Notation/Pawn-sacrifice}

Pawn-sacrifice is written by separating plies with \alg{;;} (double semicolon)
instead of usual \alg{\textasciitilde{}} (tilde), followed by capturing and displacing steps.
All displacing steps must be written; capturing steps can be omitted if only a
single path exists; captured and displaced pieces are not needed, as they can
only be Pawns.

This \hyperref[fig:scn_tr_27_pawn_sacrifice_init]{Pawn-sacrifice} followed by
\hyperref[fig:scn_tr_29_pawn_sacrifice_end]{tagged light Serpent's ply} is
written as \alg{Sb4\textasciitilde{}Ab2;;}\newline
\alg{S..b6*..b8*.c9<c10..e9\^{}W..g9*.h8<h9}.\newline
In \algfmt{XAN}, with plies gathering, it would be\newline
\alg{[Sg3-b4]\textasciitilde{}[Ab4-b2];;}\newline
\alg{[Sb4..b6*..b8*.c9<c10..e9\^{}W..g9*.h8<h9]}.

If there aren't enough Pawns captured or displaced to isolate only one path an
additional, movement-only steps needs to be written as well. In previous example,
if Serpent stopped at \alg{b8}, at least two different paths are possible. Previous
example started with \alg{Sb4\textasciitilde{}Ab2;;Sb4.c5.b6*.c7.b8*} path, other
possible path is \alg{Sb4\textasciitilde{}Ab2;;Sb4.a5.b6*.a7.b8*}. Again, for longer
paths care must be taken to write step(s) which really differentiate paths, otherwise
written path might inadvertently also denote others.

Care must also be taken to properly use step separator \alg{.} (dot), and multiple
step separator \alg{..} (two dots). Step separator \alg{.} is for separating two
steps, where one step immediately follows the other. Multi-step separator \alg{..}
is for separating two steps which have at least one unwritten step in-between.
For instance, first, short path in previous paragraph (i.e.
\alg{Sb4\textasciitilde{}Ab2;;Sb4.c5.b6*.c7.b8*}) shouldn't be shortened to
\alg{Sb4\textasciitilde{}Ab2;;S..b6*..b8*}, as it would also denote
\alg{Sb4\textasciitilde{}Ab2;;Sb4.c3.d4.c5.d6.c7.b6*.a7.b8*}, a very different path indeed.

\subsection*{Off-board traversal}
\addcontentsline{toc}{subsection}{Off-board traversal}
\label{sec:Appendix/Notation/Off-board traversal}

Steps onto virtual, off-board fields are not written. For trance-journey, each possible
destination field designates unique path on its own, so additional fields are necessary
only if there is some kind of interactions between entranced Shaman and pieces on its
step-fields.

For a Wave activated by Serpent, noting destination field might be enough, if destination
field is on a different file and a different rank than starting field. If destination field
is on the same rank or on the same file as starting field, then first step needs to be noted
as well. In this \hyperref[fig:scn_tr_36_serpent_activated_wave_ply]{Wave activated by Serpent}
example, if destination field is \alg{j4}, then there is only one path leading to it, and
it's \alg{Se5\textasciitilde{}We5.f4.g5.h4.i5.j4}, so it can be noted just as
\alg{Se5\textasciitilde{}Wj4}.

If destination field is \alg{i5}, which is on the same rank as Wave's ply starting field,
then \alg{Se5\textasciitilde{}Wi5} might be interpreted as either
\alg{Se5\textasciitilde{}We5.f4.g5.h4.i5}, or as\newline
\alg{Se5\textasciitilde{}We5.f6.g5.h6.i5}, so first step is needed, like so
\alg{Se5\textasciitilde{}W.f4..i5}.

If Wave activated by Serpent is blocked from reaching destination field using only on-board
step-fields, then only one path exists, and only destination field is needed. For instance,
if in this
\hyperref[fig:scn_tr_37_wave_out_of_board]{Wave off-board example} dark Knight were located at
\alg{u9}, and destination field is \alg{v10} it would block Wave's ply
\alg{Sv6\textasciitilde{}W.u7.v8.u9.v10}, and only path available to Wave would be off-board,
i.e. \alg{Sv6\textasciitilde{}W..v8..v10}.

For a Wave activated by Unicorn or Centaur, noting destination field might be enough, if
destination field does not share file, rank or diagonal with starting field. For instance, if
destination in this \hyperref[fig:scn_mv_030_wave_off_board]{Wave off-board example} is field 2,
then it can be noted just as \alg{Uo3\textasciitilde{}Wp12}. If, in the same example,
destination is field 1, then original path is \alg{Uo3\textasciitilde{}W.m4.p6.n7..o10},
the other available path is \alg{Uo3\textasciitilde{}W..n6.p7.m9.o10}, so at least one
other step is needed to distinguish between the two paths.

\subsection*{Losing tags}
\addcontentsline{toc}{subsection}{Losing tags}
\label{sec:Appendix/Notation/Losing tags}

Losing tag is a side-effect of a tagged piece being moved, captured, etc. As such,
losing tag can also
\hyperref[tbl:Appendix/Summary/Accompanying-losing-tags]{accompany some other side-effects}.
So, losing tag is denoted immediately after piece symbol, but before its starting
position (regardless, if it's disambiguation or a field), before any steps, or
destination field, depending on what is available. Writing lost tag is completely
optional, it's meant to remind readers what happened and when.

Losing ability to castle is denoted with \alg{\&\&} (double ampersand).
\hyperref[sec:Appendix/Notation/Castling]{Using previous castling examples}, if
Rooks moved to their destination fields without castling, it would be written as
\alg{R\&\&e1}, and \alg{R\&\&q1}, and in \algfmt{XAN} it would be \alg{R\&\&a1-e1},
and \alg{R\&\&y1-q1}. Using just disambiguation, it could be written as e.g.
\alg{R\&\&a..e1}, and \alg{R\&\&y..q1}.\newline
\indent
If, in the first example, there were opponent's Bishop on light Rook's neighboring-field
capturing it, it would be written as \alg{Ba1*R\&\&}, or as \alg{Bb2-a1*R\&\&} in
\algfmt{XAN}.

If \hyperref[fig:scn_aoa_04_delayed_promo_pawn_2_tagged]{Pawn tagged for promotion}
moves before actual promotion, it losses its tag, which is denoted with \alg{==}
(double equal sign), like so \alg{==e12}. In \algfmt{XAN} the same move would be
noted as \alg{P==e11-e12}.

If the same example contained e.g. dark Bishop at \alg{f12}, Pawn tagged for promotion
could capture it, which would be written as \alg{==f12*B}, and in \algfmt{XAN} as
\alg{P==e11-f12*B}. When just disambiguation is needed, it could be written as
\alg{==ef12*B}, or e.g. as \alg{P==e.f12*B}.

If, in the same situation, light Pawn instead of moving was captured by dark Unicorn
on a field where it was tagged for promotion, it would be written as \alg{Ue11*P==},
and in \algfmt{XAN} it would be \alg{Ud7-e11*P==}.

Losing ability to rush is denoted with \alg{::} (double colon). For instance, in
this \hyperref[fig:scn_n_03_teleport_move_2]{teleporting example}, after dark Rook's
failed teleportation, light Bishop could capture dark Pawn on its initial position,
which would be denoted as \alg{Br17*P::}, and in \algfmt{XAN} it would be
\alg{Bq18-r17*P::}.

In this \hyperref[fig:scn_mv_036_activating_rush_pawn_init]{activating Pawns example},
light Pawn on the left is being activated, and can capture dark Knight. This would
be written as \alg{Re2\textasciitilde{}Wc2\textasciitilde{}P::b3*N}, and in \algfmt{XAN}
it would be \alg{[Re6-e2]\textasciitilde{}[We2-c2]\textasciitilde{}[P::c2-b3*N]}.

\subsection*{Combining side-effects}
\addcontentsline{toc}{subsection}{Combining side-effects}
\label{sec:Appendix/Notation/Combining side-effects}

Rarely, two side-effects can accompany the same step. For instance, Pawn capturing
opponent's piece can also be promoted, if capture happened on opponent's figure row.
Generally, notation follows order of actions on a chessboard, in this case a capture
would be written before promotion.

As an example, light Pawn on a second to last rank could capture dark Rook, and
then be promoted after the same step. In One variant, if light Pawn and dark Rook
occupy \alg{d} and \alg{e} files, respectively, this could be written as
\alg{e26*R=Q} or, in \algfmt{XAN} it would be \alg{Pd25-e26*R=Q}.

Care should be taken when omitting optional parts of notation. For instance,
notation for capturing is optional, but it's also possible to omit just captured
piece. Notation for immediate promotion is mandatory, but symbol is not. This
gives us possibility to rewrite first notation of previous move as \alg{e26*Q},
which would be interpreted very differently.

\subsection*{Default pathing}
\addcontentsline{toc}{subsection}{Default pathing}
\label{sec:Appendix/Notation/Default pathing}

In Classical Chess, all pieces have exactly one path from starting field to destination,
so to specify unique path only destination is needed when writing movement of a piece.
This is no longer so for newly added pieces. For instance, Serpent can have multiple
paths leading to the same destination; even though a single path cannot contain loops
different paths can overlap, and they can have different lengths.

In \hyperref[fig:scn_tr_05_serpent_end]{this Serpent's movement example}, assuming that
starting position is \alg{c3}, and destination \alg{g3}, depicted complete path is
\alg{Sc3.d4.c5.d6.e5.f6.g5.f4.g3}. Even with as few steps as possible, to have unique path
it would still need to be written as \alg{S..d6..f6..g3}.

However, one of the shortest paths (here, e.g.\newline
\alg{Sc3.d4.e3.f4.g3}) would result in exactly the same outcome, namely Serpent moved from
\alg{c3} onto \alg{g3}, with no additional interactions taking place. Even with additional
side-effects, most of the time it does not matter if a piece made long or short path to e.g.
capture opponent's piece.

The only time when length of a path is important is in a cascade, when momentum is build-up
by first piece, and spent by others. For instance, in a
\hyperref[fig:scn_tr_12_serpent_path_short]{Serpent activating Pyramid example}, depending on
a path taken by Serpent, Pyramid might get 4, 8 or 12 momentum when activated.

Even so, exact path taken by a piece is not important, only amount of a momentum gathered,
and spent. To correctly support movement of pieces in a cascade, momentum built by first
piece has to be maximized, and momentum spent by activated pieces minimized. So, all movement
can be written with just destination field; path is assumed to be the shortest possible for
all pieces, except for the first piece in a cascade, which is assumed to be taking the longest
path available.

First example here can then be written as \alg{Sg3}, this would be taken as if
\alg{Sc3.d4.e3.f4.g3} is written, i.e. light Serpent would be taking the shortest path
available. Next example would be written as \alg{Sc7\textasciitilde{}Ad7}, if Pyramid moved
1 field to the right; movement is assumed to be\newline
% \alg{[Sc3.d2.e3.f2.g3.f4.g5.f6.e5.d6.c5.b6.c7]\textasciitilde{}[Ac7-d7]},
\alg{[Sc3.d2.e3.f2.g3.f4.g5.f6.e5.d6.c5.b6.c7]}\newline
\alg{\textasciitilde{}[Ac7-d7]}, that is, light Serpent would take the longest path
available, since it's the first piece in a cascade.

\subsection*{Move symbols, annotations}
\addcontentsline{toc}{subsection}{Move symbols, annotations}
\label{sec:Appendix/Notation/Move symbols, annotations}

Placeholder for a move is \alg{...} (three dots), usually used to resume game score
after commentary, see\newline
\href{https://en.wikipedia.org/wiki/Algebraic\_notation\_(chess)\#Notation\_for\_a\_series\_of\_moves}
{https://en.wikipedia.org/wiki/\newline
Algebraic\_notation\_(chess)\#Notation\_for\_a\_series\_of\_moves}.

Checks are noted with \alg{+} (plus sign), checkmates are noted with \alg{\#} (hashtag),
these are optional in \algfmt{CAN}, see \algfmt{FIDE~C.13}. Self-checkmates are written
as stand-alone \alg{\#} (hashtag) on opponent's turn as a complete move, which ends
a game, like so:

\algcycpar
\algcyc{92.}{...}{\#}{Light player checkmated self.}
\algcycparend

Note, self-checkmate is a claim that opponent checkmated self, it has to be validated
by e.g. arbiters. If it's not valid, self-checkmate is rejected as if player tried to
perform an invalid move, and game continues with a player putting forward such a claim
still "on turn".

Resigns are written with \alg{\#\#} (double hashtag) as a complete move, which also
ends a game, like so:

\algcycpar
\algcyc{92.}{...}{\#\#}{Dark player resigns.}
\algcycparend

Draw offer is noted with \alg{(=)} (equal sign in brackets), see \algfmt{FIDE~C.12};
it's written immediately following a completed move notation (which might include check),
like so:

\algcycpar
\algcyc{71.}{Nb3+(=)}{...}{Light player offers draw.}
\algcycparend

Draw offer can be accepted, as long as draw offered by the opponent is valid, i.e. not
canceled, by writing \alg{(==)} (two equal signs in brackets) as a complete move, which
ends a game, like so:

\algcycpar
\algcyc{82.}{...}{(==)}{Dark player accepts draw offer.}
\algcycparend

Draw offer can be canceled by writing \alg{()} (empty pair of brackets), after a complete
move, like so:

\algcycpar
\algcyc{79.}{Bc7+()}{...}{Light player cancels draw offer.}
\algcycparend

Forced draw, i.e. draw by rules, is written with \alg{(===)} (three equal signs in
brackets) as a complete move, this also ends a game, like so:

\algcycpar
\algcyc{82.}{...}{(===)}{Draw forced, by the rules.}
\algcycparend

In \algfmt{NAN} checks are optional; checkmates, self-checkmates, resigns, accepted
and forced draws are all mandatory, to set game score end. Draw offers, cancelations
are mandatory to set limits within which draw offer is valid, and can be accepted by
the opponent.

Annotations are written at the end of a complete move, draw offer, e.g. \alg{ef8*!},
\alg{Nb3(=)=}. It is recommended to use \alg{\_} (underscore) to separate \algfmt{AN}
and annotations, like so \alg{ef8*\_!}, \alg{Nb3(=)\_=}. Usage of underscore is
mandatory when annotation can be confused for a regular chess \algfmt{AN}.

For instance, \alg{e8=} might be Pawn tagged for promotion, or both players have equal
chances of winning, see\newline
\href{https://en.wikipedia.org/wiki/Algebraic\_notation\_(chess)#Annotation\_symbols}
{https://en.wikipedia.org/wiki/Algebraic\_notation\_(chess)\newline
\#Annotation\_symbols}. In such a case, regular chess \algfmt{AN} is assumed, i.e. it
is Pawn tagged for promotion. If annotation is meant instead, it has to be written as
\alg{e8\_=}.

\clearpage % ..........................................................

\section*{Summary}
\addcontentsline{toc}{section}{Summary}
\label{sec:Appendix/Summary}

Now that all symbols have been introduced, they're gathered here according to their purpose.

\subsection*{Side-effects}
\addcontentsline{toc}{subsection}{Side-effects}
\label{sec:Appendix/Summary/Side-effects}

\begin{table}[!h]
\centering
\begin{tabular}{ rlc }
\toprule % ===========================================================================
\textbf{Symbol}             & \textbf{Side-effect}          & \textbf{Mandatory?}   \\
\midrule % ---------------------------------------------------------------------------
\alg{*}                     & capturing                     & -                     \\
\alg{<}                     & displacement                  & +\footnotemark[1]     \\
\cmidrule{1-3} % .....................................................................
\alg{==}                    & lost promotion tag            & -                     \\
\alg{::}                    & lost rushing tag              & -                     \\
\alg{\&\&}                  & lost castling tag             & -                     \\
\cmidrule{1-3} % .....................................................................
\alg{:}                     & en passant                    & -                     \\
\alg{\&}                    & castling                      & -                     \\
\alg{=}                     & promotion                     & +\footnotemark[2]     \\
\alg{=}                     & tag for promotion             & +                     \\
\alg{\%}                    & conversion                    & +                     \\
\alg{\%\%}                  & failed conversion, oblation   & +                     \\
\alg{\^{}}                  & transparency                  & -                     \\
\alg{/}                     & divergence                    & -                     \\
\alg{>}                     & syzygy, demoting to Pawn      & +                     \\
\alg{\$}                    & syzygy, resurrection          & +                     \\
\cmidrule{1-3} % .....................................................................
\multirow{2}{*}{\alg{\$\$}} & syzygy, resurrecting          & \multirow{2}{*}{+}    \\
                            & opponent's piece              &                       \\
\cmidrule{1-3} % .....................................................................
\alg{\$\$\$}                & failed resurrection           & -                     \\
\bottomrule % ========================================================================
\end{tabular}
\caption{Side-effects}
\label{tbl:Appendix/Summary/Side-effects}
\end{table}

\footnotetext[1]{Displaced piece is optional.}
\footnotetext[2]{Promotion symbol is optional.}

Every side-effect is result of a single step. Most side-effects can occur only on
a last step of a ply, these are called ply side-effects. Capturing, displacement
and losing (promotion, rushing, or castling) tags are both step and ply side-effects,
others are purely ply side-effects.

In Classical Chess capturing opponent's piece is the most prevalent side-effect
since there aren't all that many others, and those can be done once per Pawn (en
passant, promotion), or once per whole game (castling). So, in \algfmt{CAN}
capturing is optional, see \algfmt{FIDE~C.9}; this is so in \algfmt{NAN} as well.

Most mandatory side-effects are marked as such, because otherwise \algfmt{AN}
would lack information to describe what happened. Displacement has to have a
destination field where a piece has been displaced. Promotion needs a promoted-to
piece written, otherwise it's assumed that a Pawn has been tagged for promotion.
Demoting to Pawn syzygy has to have, at very least, a disambiguation (or piece,
if unique) written to be able to find which piece has been demoted, and on which
field. Resurrection syzygy must have a piece which has been resurrected, if
initiating Starchild was not oblationed then destination field as well.

Other mandatory side-effects are designated as such, because otherwise it could
be assumed that targeted piece has been captured. Conversion and failed conversion
are such side-effects.

It is recommended to also write optional side-effects and their data, for not much
more effort reader is presented with much easier to understand notation. Compare
minimalistic notation with slightly more verbose version:

\noindent
\alg{Re2\textasciitilde{}Wc2\textasciitilde{}b3} vs.\newline
\alg{Re2\textasciitilde{}Wc2\textasciitilde{}b3*} vs.\newline
\alg{Re2\textasciitilde{}Wc2\textasciitilde{}P::b3*N},

\noindent
\alg{He12@H..q16..k14..c18} vs.\newline
\alg{He12@H..q16*..k14*..c18} vs.\newline
\alg{He12@H\textbackslash{}w18..q16*P..k14*N..c18},

\noindent
\alg{He12@@} vs.\newline
\alg{He12@@P,B,R,R,N,B,N} vs.\newline
\alg{He12@@Pq16,Bp14,Rd20,Rg6,Nk14,Bj12,Nd10}.

\clearpage % ..........................................................

\subsubsection*{Side-effects on pieces}
\addcontentsline{toc}{subsubsection}{Side-effects on pieces}
\label{sec:Appendix/Summary/Side-effects/Side-effects on pieces}

\begin{table}[!h]
\centering
\begin{tabular}{ lcc }
\toprule % =====================================================
\textbf{Piece} & \textbf{Disposable?} & \textbf{Diverging?}   \\
\midrule % -----------------------------------------------------
Pawn           & +                    & +                     \\
Knight         & +                    & +                     \\
Bishop         & +                    & +                     \\
Rook           & +                    & +                     \\
Queen          & +                    & +                     \\
King           & -                    & -                     \\
\cmidrule{1-3} % ...............................................
Pegasus        & +                    & +                     \\
Pyramid        & +                    & +                     \\
Unicorn        & +                    & *\footnotemark[2]     \\
Wave           & +                    & *\footnotemark[3]     \\
Star           & -                    & -                     \\
Centaur        & +                    & -                     \\
Scout          & +                    & +                     \\
Grenadier      & +                    & +                     \\
Serpent        & +                    & -                     \\
Shaman         & +                    & +                     \\
Monolith       & -                    & -                     \\
Starchild      & *\footnotemark[1]    & -                     \\
\bottomrule % ==================================================
\end{tabular}
\caption{Side-effects on pieces}
\label{tbl:Appendix/Summary/Side-effects/Side-effects on pieces}
\end{table}

\footnotetext[1]{Unlike other disposable pieces, Starchild cannot be neither
teleported, nor converted.}
\footnotetext[2]{While Unicorn can be diverted, Wave activated by Unicorn cannot
be diverted.}
\footnotetext[3]{Wave activated by another Wave can be diverted, if activating Wave
could be diverted; that depends on which last material (non-Wave) piece was preceding
those Waves in a cascade.}

Disposable pieces are all that can be captured, that is all but Kings, Stars and
Monoliths. Disposable pieces can also be displaced, oblationed, and resurrected.
Pawn can be promoted to any other disposable piece.

All disposable pieces, except Starchild, can also be teleported and converted.
Most disposable pieces, except Centaur, Serpent, and Starchild, can also be
diverted. Waves activated by disposable pieces can be diverted, except if
activated by Unicorn, Centaur, Serpent, or Starchild.

Non-disposable pieces (Kings, Stars and Monoliths) cannot be captured, displaced,
teleported, converted, oblationed, resurrected, or diverted. Pawn cannot be
promoted to any non-disposable piece.

\clearpage % ..........................................................

\subsection*{Accompanying losing tags}
\addcontentsline{toc}{subsection}{Accompanying losing tags}
\label{sec:Appendix/Summary/Accompanying-losing-tags}

\begin{table}[!h]
\centering
\begin{tabular}{ rlccc }
\toprule % ==================================================================================================================
\textbf{Sym.}               & \textbf{Side-effect}  & \multicolumn{3}{c}{ \textbf{Accompanying losing tags} }              \\
                                                    \cmidrule{3-5} % ........................................................
                            &                       & \emph{castling}      & \emph{promoting}      & \emph{rushing}        \\
\midrule % ------------------------------------------------------------------------------------------------------------------
\alg{*}                     & capture               & +                    & +                     & +                     \\
\alg{<}                     & displacement          & +                    & +                     & +                     \\
\alg{:}                     & en passant            & -                    & -                     & -                     \\
\alg{\&}                    & castle                & *                    & -                     & -                     \\
\alg{=}                     & promotion             & -                    & *                     & -                     \\
\alg{=}                     & tag for promotion     & -                    & -                     & -                     \\
\alg{\%}                    & conversion            & -                    & +                     & -                     \\
\cmidrule{1-5} % ............................................................................................................
\multirow{2}{*}{\alg{\%\%}} & failed conversion,    & \multirow{2}{*}{-}   & \multirow{2}{*}{-}    & \multirow{2}{*}{-}    \\
                            & oblation              &                      &                       &                       \\
\cmidrule{1-5} % ............................................................................................................
\alg{\^{}}                  & transparency          & -                    & -                     & -                     \\
\alg{/}                     & divergence            & -                    & -                     & -                     \\
\cmidrule{1-5} % ............................................................................................................
\multirow{2}{*}{\alg{>}}    & syzygy,               & \multirow{2}{*}{+}   & \multirow{2}{*}{-}    & \multirow{2}{*}{-}    \\
                            & demoting to Pawn      &                      &                       &                       \\
\cmidrule{1-5} % ............................................................................................................
\alg{\$}                    & syzygy, resurrection  & -                    & -                     & -                     \\
\cmidrule{1-5} % ............................................................................................................
\multirow{2}{*}{\alg{\$\$}} & syzygy, resurrecting  & \multirow{2}{*}{-}   & \multirow{2}{*}{-}    & \multirow{2}{*}{-}    \\
                            & opponent's piece      &                      &                       &                       \\
\cmidrule{1-5} % ............................................................................................................
\alg{\$\$\$}                & failed resurrection   & -                    & -                     & -                     \\
\bottomrule % ===============================================================================================================
\end{tabular}
\caption{Accompanying losing tags}
\label{tbl:Appendix/Summary/Accompanying-losing-tags}
\end{table}

Table above lists most side-effects in rows. Some of those side-effects could also cause
targeted piece to lose its promoting, rushing, or castling tag. Losing tags (accompanying
those side-effects) are listed as columns. For instance, Rook still holding its castling
tag can be captured, which is indicated by + (plus sign), under column \emph{castling},
and row \emph{capture}.

Some combinations are not possible, which is noted by - (minus sign). For example, Pawn
holding promoting tag cannot be captured by en passant move, as indicated by - at
\emph{promoting} column and \emph{en passant} row. This is so because Pawn can get
promoting tag only on opponent's side of chessboard, while it can rush, and be subjected
to en passant only on own side of a chessboard.

Two combinations are actually using a tag, those are indicated by * (asterisk). Using a
tag implicitly loses it, since no tag can be repeatedly applied (e.g. Pawn after promotion
cannot be promoted again), so using a tag is not written as tag loss in \algfmt{AN}. For
instance, a Rook having castling tag can castle, as found under \emph{castling} column,
and \emph{castle} row. Another instance is a Pawn holding promoting tag which it can use,
found in \emph{promoting} column and \emph{promotion} row.

\clearpage % ..........................................................

\subsection*{Path separators}
\addcontentsline{toc}{subsection}{Path separators}
\label{sec:Appendix/Summary/Path separators}

\begin{table}[!h]
\centering
\begin{tabular}{ rl }
\toprule % ============================================================
\textbf{Symbol}         & \textbf{Separates}                         \\
\midrule % ------------------------------------------------------------
\alg{.}                 & single steps                               \\
\alg{..}                & multiple steps                             \\
\alg{-}                 & steps and destination field                \\
\alg{\textbackslash}    & reposition                                 \\
\cmidrule{1-2} % ......................................................
\alg{\textasciitilde{}} & plies                                      \\
\alg{|}                 & teleportation                              \\
\alg{||}                & failed teleportation, re-emergence         \\
\alg{|||}               & failed teleportation, oblation             \\
\alg{@}                 & trance-journey                             \\
\alg{@@}                & dual trance-journey, oblation              \\
\alg{@@@}               & failed trance-journey, oblation            \\
\alg{;;}                & Pawn-sacrifice                             \\
\alg{"}                 & sense-journey                              \\
\alg{'}                 & failed sense-journey, oblation             \\
\cmidrule{1-2} % ......................................................
\alg{,}                 & items in a list                            \\
\alg{[}, \alg{]}        & ply gathering                              \\
\bottomrule % =========================================================
\end{tabular}
\caption{Path separators}
\label{tbl:Appendix/Summary/Path separators}
\end{table}

First 3 symbols (\alg{.}, \alg{..}, and \alg{-}) separates steps within a single ply,
the next one (i.e. \alg{\textbackslash}) separates initial and starting positions.
Other symbols (\alg{\textasciitilde{}}, \alg{|}, \alg{@}, \alg{;;}, and \alg{"}) are
separators between plies; or ply and \textendash{}by extension\textendash{} move
terminators (\alg{||}, \alg{|||}, \alg{@@}, \alg{@@@}, and \alg{'}).\newline
\indent
Items separator (\alg{,}) is used where multiple items needs to be listed for a single
step (or a ply), e.g. pieces captured in a dark Shaman's dual trance-journey.\newline
\indent
Ply gathering symbols (\alg{[}, \alg{]}) are just wrappers around plies to visually
enhance them, making them easier to tell apart, but otherwise does not contribute any
new information.

\clearpage % ..........................................................

\subsection*{Move symbols}
\addcontentsline{toc}{subsection}{Move symbols}
\label{sec:Appendix/Summary/Move symbols}

\begin{table}[!h]
\centering
\begin{tabular}{ rl }
\toprule % =============================================
\textbf{Symbol}      & \textbf{Status}                \\
\midrule % ---------------------------------------------
\alg{+}              & check                          \\
\alg{\#}             & checkmate                      \\
\alg{\#}             & self-checkmate                 \\
\alg{\#\#}           & resign                         \\
\cmidrule{1-2} % .......................................
\alg{(=)}            & draw offered                   \\
\alg{()}             & draw offer withdrawn           \\
\alg{(==)}           & draw accepted                  \\
\alg{(===)}          & draw by rules                  \\
\cmidrule{1-2} % .......................................
\alg{\_}             & annotations separator          \\
\alg{...}            & placeholder for a move         \\
\bottomrule % ==========================================
\end{tabular}
\caption{Move symbols}
\label{tbl:Appendix/Summary/Move symbols}
\end{table}

Check notation is optional in \algfmt{NAN}, just like in \algfmt{CAN}, see
\algfmt{FIDE~C.9}. Checkmate, self-checkmate and resign notations are mandatory,
to note end of a game score (list of moves by both players).

All of draw notations are mandatory in \algfmt{NAN}; offering draw and canceling
offer because they give an opportunity window within which draw offer can be
accepted by the opponent. Accepted draw and draw by rules notations are mandatory,
since they also note end of a game score.

Draw offer does not expire, and can be issued multiple times. If a draw offer is
not valid anymore it has to be canceled; a single \alg{()} cancels draw offer,
regardless how many times it was offered prior to cancelation.

\clearpage % ..........................................................

\subsection*{Initial setups}
\addcontentsline{toc}{subsection}{Initial setups}
\label{sec:Appendix/Summary/Initial setups}

\begin{table}[!h]
\centering
\begin{tabular}{ lrr }
\toprule % ===============================================
\textbf{Variant}        & \textbf{No. of Pawn rows}     \\
                                            \cmidrule{3-3}
              \multicolumn{3}{r}{ \textbf{Figure row} } \\
\midrule % -----------------------------------------------
Classical Chess         &                     1         \\
                   \multicolumn{3}{r}{ \alg{RNBQKBNR} } \\
\cmidrule{1-3} % .........................................
Croatian Ties           &                     1         \\
                 \multicolumn{3}{r}{ \alg{RENBQKBNER} } \\
\cmidrule{1-3} % .........................................
Mayan Ascendancy        &                     1         \\
               \multicolumn{3}{r}{ \alg{REANBQKBNAER} } \\
\cmidrule{1-3} % .........................................
Age of Aquarius         &                     1         \\
             \multicolumn{3}{r}{ \alg{REAUNBQKBNUAER} } \\
\cmidrule{1-3} % .........................................
Miranda's Veil          &                     1         \\
           \multicolumn{3}{r}{ \alg{REAUWNBQKBNWUAER} } \\
\cmidrule{1-3} % .........................................
Nineteen                &                     2         \\
         \multicolumn{3}{r}{ \alg{TRNBWEUAQKAUEWBNRt} } \\
\cmidrule{1-3} % .........................................
Hemera's Dawn           &                     2         \\
       \multicolumn{3}{r}{ \alg{TRNBCWEUAQKAUEWCBNRt} } \\
\cmidrule{1-3} % .........................................
Tamoanchan Revisited    &                     2         \\
     \multicolumn{3}{r}{ \alg{TRNBSWUECAQKACEUWSBNRt} } \\
\cmidrule{1-3} % .........................................
Conquest of Tlalocan    &                     2         \\
   \multicolumn{3}{r}{ \alg{TRNBSCUWEAHQKHAEWUCSBNRt} } \\
\cmidrule{1-3} % .........................................
Discovery               &                     2         \\
   \multicolumn{3}{r}{ \alg{TRNBSCUWEAHQKHAEWUCSBNRt} } \\
\cmidrule{1-3} % .........................................
One                     &                     2         \\
 \multicolumn{3}{r}{ \alg{TRNBSICUEWAHQKHAWEUCISBNRt} } \\
\bottomrule % ============================================
\end{tabular}
\caption{Initial setups of light figures}
\label{tbl:Appendix/Summary/Initial setups of light figures}
\end{table}

\clearpage % ..........................................................

Initial setups table contains complete row of figures for light player, at the
beginning of a match. In this table, lower case letters are used to denote dark
pieces. In later variants, dark Star is positioned in bottom right corner of a
chessboard, which is indicated with \alg{t}.

Dark player's setup is mirrored, with all figures switched to opposite of light
player's setup. So, for Nineteen variant \alg{TRNBWEUAQKAUEWBNRt} becomes
\alg{trnbwguaqkaugwbnrT} for dark player.

Each variant can have 1 or 2 rows of Pawns for each player, in front of its
figures. For light player, Pawn rows are rows 2 (and 3, in later variants).
For dark player Pawn rows are 2nd to last (and 3rd to last, in later variants).

\clearpage % ..........................................................

\subsubsection*{Scouts}
\addcontentsline{toc}{subsubsection}{Scouts}
\label{sec:Appendix/Summary/Initial setups/Scouts}

In addition to two rows of Pawns, most of later variants also feature Scouts.\newline

\begin{table}[!h]
\centering
\begin{tabular}{ lr }
\toprule % ===========================================================================
\textbf{Variant}                        & \textbf{Light Scouts setups}              \\
\midrule % ---------------------------------------------------------------------------
\multirow{2}{*}{Hemera's Dawn}          & \alg{c4}, \alg{g4}, \alg{n4}, \alg{r4},   \\
                                        & \alg{d5}, \alg{f5}, \alg{o5}, \alg{q5}    \\
\cmidrule{1-2} % .....................................................................
\multirow{2}{*}{Tamoanchan Revisited}   & \alg{g4}, \alg{k4}, \alg{l4}, \alg{p4},   \\
                                        & \alg{h5}, \alg{j5}, \alg{m5}, \alg{o5}    \\
\cmidrule{1-2} % .....................................................................
\multirow{4}{*}{Conquest of Tlalocan}   & \alg{d4}, \alg{h4}, \alg{i4}, \alg{l4},   \\
                                        & \alg{m4}, \alg{p4}, \alg{q4}, \alg{u4},   \\
                                        & \alg{e5}, \alg{g5}, \alg{j5}, \alg{l5},   \\
                                        & \alg{m5}, \alg{o5}, \alg{r5}, \alg{t5}    \\
\cmidrule{1-2} % .....................................................................
\multirow{4}{*}{Discovery}              & \alg{d4}, \alg{h4}, \alg{i4}, \alg{l4},   \\
                                        & \alg{m4}, \alg{p4}, \alg{q4}, \alg{u4},   \\
                                        & \alg{e5}, \alg{g5}, \alg{j5}, \alg{l5},   \\
                                        & \alg{m5}, \alg{o5}, \alg{r5}, \alg{t5}    \\
\cmidrule{1-2} % .....................................................................
\multirow{4}{*}{One}                    & \alg{e4}, \alg{i4}, \alg{j4}, \alg{m4},   \\
                                        & \alg{n4}, \alg{q4}, \alg{r4}, \alg{v4},   \\
                                        & \alg{f5}, \alg{h5}, \alg{k5}, \alg{m5},   \\
                                        & \alg{n5}, \alg{p5}, \alg{s5}, \alg{u5}    \\
\bottomrule % ========================================================================
\end{tabular}
\caption{Light Scouts setups}
\label{tbl:Appendix/Summary/Initial setups/Light Scouts setups}
\end{table}

Table above contains initial positions of Scouts for light player.

\clearpage % ..........................................................

\begin{table}[!h]
\centering
\begin{tabular}{ lr }
\toprule % ===============================================================================
\textbf{Variant}                        & \textbf{Dark Scouts setups}                   \\
\midrule % -------------------------------------------------------------------------------
\multirow{2}{*}{Hemera's Dawn}          & \alg{c17}, \alg{g17}, \alg{n17}, \alg{r17},   \\
                                        & \alg{d16}, \alg{f16}, \alg{o16}, \alg{q16}    \\
\cmidrule{1-2} % .........................................................................
\multirow{2}{*}{Tamoanchan Revisited}   & \alg{g19}, \alg{k19}, \alg{l19}, \alg{p19},   \\
                                        & \alg{h18}, \alg{j18}, \alg{m18}, \alg{o18}    \\
\cmidrule{1-2} % .........................................................................
\multirow{4}{*}{Conquest of Tlalocan}   & \alg{d21}, \alg{h21}, \alg{i21}, \alg{l21},   \\
                                        & \alg{m21}, \alg{p21}, \alg{q21}, \alg{u21},   \\
                                        & \alg{e20}, \alg{g20}, \alg{j20}, \alg{l20},   \\
                                        & \alg{m20}, \alg{o20}, \alg{r20}, \alg{t20}    \\
\cmidrule{1-2} % .........................................................................
\multirow{4}{*}{Discovery}              & \alg{d21}, \alg{h21}, \alg{i21}, \alg{l21},   \\
                                        & \alg{m21}, \alg{p21}, \alg{q21}, \alg{u21},   \\
                                        & \alg{e20}, \alg{g20}, \alg{j20}, \alg{l20},   \\
                                        & \alg{m20}, \alg{o20}, \alg{r20}, \alg{t20}    \\
\cmidrule{1-2} % .........................................................................
\multirow{4}{*}{One}                    & \alg{e23}, \alg{i23}, \alg{j23}, \alg{m23},   \\
                                        & \alg{n23}, \alg{q23}, \alg{r23}, \alg{v23},   \\
                                        & \alg{f22}, \alg{h22}, \alg{k22}, \alg{m22},   \\
                                        & \alg{n22}, \alg{p22}, \alg{s22}, \alg{u22}    \\
\bottomrule % ============================================================================
\end{tabular}
\caption{Dark Scouts setups}
\label{tbl:Appendix/Summary/Initial setups/Dark Scouts setups}
\end{table}

Table above contains initial positions of Scouts for dark player.

\clearpage % ..........................................................

\subsubsection*{Grenadiers}
\addcontentsline{toc}{subsubsection}{Grenadiers}
\label{sec:Appendix/Summary/Initial setups/Grenadiers}

Some of Pawns found in two initial rows are exchanged for Grenadiers, in most of later variants.\newline

\begin{table}[!h]
\centering
\begin{tabular}{ lr }
\toprule % ===========================================================================
\textbf{Variant}                        & \textbf{Light Grenadiers setups}          \\
\midrule % ---------------------------------------------------------------------------
\multirow{2}{*}{Hemera's Dawn}          & \alg{c3}, \alg{g3}, \alg{n3}, \alg{r3},   \\
                                        & \alg{d2}, \alg{f2}, \alg{o2}, \alg{q2}    \\
\cmidrule{1-2} % .....................................................................
\multirow{2}{*}{Tamoanchan Revisited}   & \alg{g3}, \alg{k3}, \alg{l3}, \alg{p3},   \\
                                        & \alg{h2}, \alg{j2}, \alg{m2}, \alg{o2}    \\
\cmidrule{1-2} % .....................................................................
\multirow{4}{*}{Conquest of Tlalocan}   & \alg{d3}, \alg{h3}, \alg{i3}, \alg{l3},   \\
                                        & \alg{m3}, \alg{p3}, \alg{q3}, \alg{u3},   \\
                                        & \alg{e2}, \alg{g2}, \alg{j2}, \alg{l2},   \\
                                        & \alg{m2}, \alg{o2}, \alg{r2}, \alg{t2}    \\
\cmidrule{1-2} % .....................................................................
\multirow{4}{*}{Discovery}              & \alg{d3}, \alg{h3}, \alg{i3}, \alg{l3},   \\
                                        & \alg{m3}, \alg{p3}, \alg{q3}, \alg{u3},   \\
                                        & \alg{e2}, \alg{g2}, \alg{j2}, \alg{l2},   \\
                                        & \alg{m2}, \alg{o2}, \alg{r2}, \alg{t2}    \\
\cmidrule{1-2} % .....................................................................
\multirow{4}{*}{One}                    & \alg{e3}, \alg{i3}, \alg{j3}, \alg{m3},   \\
                                        & \alg{n3}, \alg{q3}, \alg{r3}, \alg{v3},   \\
                                        & \alg{f2}, \alg{h2}, \alg{k2}, \alg{m2},   \\
                                        & \alg{n2}, \alg{p2}, \alg{s2}, \alg{u2}    \\
\bottomrule % ========================================================================
\end{tabular}
\caption{Light Grenadiers setups}
\label{tbl:Appendix/Summary/Initial setups/Light Grenadiers setups}
\end{table}

Table above contains initial positions of Grenadiers for light player.

\clearpage % ..........................................................

\begin{table}[!h]
\centering
\begin{tabular}{ lr }
\toprule % ===============================================================================
\textbf{Variant}                        & \textbf{Dark Grenadiers setups}               \\
\midrule % -------------------------------------------------------------------------------
\multirow{2}{*}{Hemera's Dawn}          & \alg{c18}, \alg{g18}, \alg{n18}, \alg{r18},   \\
                                        & \alg{d19}, \alg{f19}, \alg{o19}, \alg{q19}    \\
\cmidrule{1-2} % .........................................................................
\multirow{2}{*}{Tamoanchan Revisited}   & \alg{g20}, \alg{k20}, \alg{l20}, \alg{p20},   \\
                                        & \alg{h21}, \alg{j21}, \alg{m21}, \alg{o21}    \\
\cmidrule{1-2} % .........................................................................
\multirow{4}{*}{Conquest of Tlalocan}   & \alg{d22}, \alg{h22}, \alg{i22}, \alg{l22},   \\
                                        & \alg{m22}, \alg{p22}, \alg{q22}, \alg{u22},   \\
                                        & \alg{e23}, \alg{g23}, \alg{j23}, \alg{l23},   \\
                                        & \alg{m23}, \alg{o23}, \alg{r23}, \alg{t23}    \\
\cmidrule{1-2} % .........................................................................
\multirow{4}{*}{Discovery}              & \alg{d22}, \alg{h22}, \alg{i22}, \alg{l22},   \\
                                        & \alg{m22}, \alg{p22}, \alg{q22}, \alg{u22},   \\
                                        & \alg{e23}, \alg{g23}, \alg{j23}, \alg{l23},   \\
                                        & \alg{m23}, \alg{o23}, \alg{r23}, \alg{t23}    \\
\cmidrule{1-2} % .........................................................................
\multirow{4}{*}{One}                    & \alg{e24}, \alg{i24}, \alg{j24}, \alg{m24},   \\
                                        & \alg{n24}, \alg{q24}, \alg{r24}, \alg{v24},   \\
                                        & \alg{f25}, \alg{h25}, \alg{k25}, \alg{m25},   \\
                                        & \alg{n25}, \alg{p25}, \alg{s25}, \alg{u25}    \\
\bottomrule % ============================================================================
\end{tabular}
\caption{Dark Grenadiers setups}
\label{tbl:Appendix/Summary/Initial setups/Dark Grenadiers setups}
\end{table}

Table above contains initial positions of Grenadiers for dark player.

\clearpage % ..........................................................

\subsubsection*{Monolith initial positions}
\addcontentsline{toc}{subsubsection}{Monolith initial positions}
\label{sec:Appendix/Summary/Monolith initial positions}

\begin{table}[!h]
\centering
\begin{tabular}{ lrr }
\toprule % =====================================================
\textbf{Variant}      & \multicolumn{2}{c}{ \textbf{Side} }   \\
                      \cmidrule{2-3} % .........................
                      & \emph{light}  & \emph{dark}           \\
\midrule % -----------------------------------------------------
Discovery             &     \alg{b7}  &    \alg{w18}          \\
One                   &     \alg{b8}  &    \alg{y19}          \\
\bottomrule % ==================================================
\end{tabular}
\caption{Monolith initial positions}
\label{tbl:Appendix/Summary/Monolith initial positions}
\end{table}

Table above contains initial positions of both Monoliths, one located on light
side of chessboard, the other on dark side.

\clearpage % ..........................................................

\subsection*{Movement limits}
\addcontentsline{toc}{subsection}{Movement limits}
\label{sec:Appendix/Summary/Movement limits}

\begin{table}[!h]
\centering
\begin{tabular}{ lrrrr }
\toprule % ===============================================================================================
\textbf{Variant}      & \textbf{Scout}  & \multicolumn{2}{c}{ \textbf{Grenadier} }  & \textbf{Serpent}  \\
                                        \cmidrule{3-4} % ............................
                      &                 &       \emph{hor.} &          \emph{vert.} &                   \\
\midrule % -----------------------------------------------------------------------------------------------
Hemera's Dawn         &               5 &                 3 &                     2 &               --- \\
Tamoanchan Revisited  &               6 &                 4 &                     2 &                14 \\
Conquest of Tlalocan  &               6 &                 4 &                     2 &                15 \\
Discovery             &               6 &                 4 &                     2 &                15 \\
One                   &               7 &                 5 &                     3 &                16 \\
\bottomrule % ============================================================================================
\end{tabular}
\caption{Movement limits}
\label{tbl:Appendix/Summary/Movement limits}
\end{table}

Movement limits table contains maximum number of steps Scouts, Grenadiers, and
Serpents can make, depending on which variant is being played.

Grenadier's limits in the table above apply when there are no opponent's pieces
on its grenadier-fields. Horizontal and vertical movement limits for Grenadier
are different, and can be found in columns \emph{hor.}, and \emph{vert.},
respectively.

When there are opponent's pieces on its grenadier-fields, Grenadier can make 1
step up or down. In close quarters, Grenadier can make 1 step more than the count
of opponent's pieces on its grenadier-fields to the left or to the right. After
each step, Grenadier can end its ply with forking capture-step.\newline
\indent
So, in close quarters, Grenadier's movement limits are 2 steps when starting
a ply with a vertical step, and 10 steps when starting with a horizontal step,
regardless which variant is being played.

Serpent is introduced in Tamoanchan Revisited variant, so its movement limit is
not defined for Hemera's Dawn variant.\newline
\indent
Monolith can make one step more than the number of Pawns player has.
Starchild can make only one step in a ply.

\clearpage % ..........................................................

\subsection*{Rushing limits}
\addcontentsline{toc}{subsection}{Rushing limits}
\label{sec:Appendix/Summary/Rushing limits}

\begin{table}[!h]
\centering
\begin{tabular}{ lcrrcrr }
\toprule % =================================================================================================
\emph{Ranks}          & & \multicolumn{2}{c}{ \textbf{Light} } & & \multicolumn{2}{c}{ \textbf{Dark} }    \\
\cmidrule{1-1}          \cmidrule{3-4}                           \cmidrule{6-7} % ..........................
\textbf{Variant}      & & \emph{min} & \emph{max}              & & \emph{min} & \emph{max}                \\
\midrule % -------------------------------------------------------------------------------------------------
Classical Chess       & & \alg{4}    & \alg{4}                 & & \alg{5}    & \alg{5}                   \\
% \cmidrule{1-7} % ...........................................................................................
Croatian Ties         & & \alg{4}    & \alg{5}                 & & \alg{6}    & \alg{7}                   \\
Mayan Ascendancy      & & \alg{4}    & \alg{6}                 & & \alg{7}    & \alg{9}                   \\
Age of Aquarius       & & \alg{4}    & \alg{7}                 & & \alg{8}    & \alg{11}                  \\
Miranda's Veil        & & \alg{4}    & \alg{8}                 & & \alg{9}    & \alg{13}                  \\
Nineteen              & & \alg{4}    & \alg{9}                 & & \alg{10}   & \alg{15}                  \\
Hemera's Dawn         & & \alg{4}    & \alg{10}                & & \alg{11}   & \alg{17}                  \\
Tamoanchan Revisited  & & \alg{4}    & \alg{11}                & & \alg{12}   & \alg{19}                  \\
Conquest of Tlalocan  & & \alg{4}    & \alg{12}                & & \alg{13}   & \alg{21}                  \\
Discovery             & & \alg{4}    & \alg{12}                & & \alg{13}   & \alg{21}                  \\
One                   & & \alg{4}    & \alg{13}                & & \alg{14}   & \alg{23}                  \\
\bottomrule % ==============================================================================================
\end{tabular}
\caption{Rushing limits, ranks}
\label{tbl:Appendix/Summary/Rushing limits}
\end{table}

Table above contains minimum and maximum rank a rushing private can reach. All
privates can rush up to, and including, the farthest rank on their own side of
a chessboard. So, destination ranks reachable by light and dark players do not
overlap when rushing.

Rank limits are based on movement of privates closest to initial rank of own
figures. For light player table contains limits for privates on the second rank,
and for dark player on the second to last rank.

In later variants, as more privates are added to initial setup, those on more
forward positions cannot reach all of destination ranks listed above when rushing.\newline
\indent
For instance, in Discovery variant light Scouts at initial rank 5 can only rush
between, and including, ranks 7 to 12.

\clearpage % ..........................................................

\subsection*{Castling limits}
\addcontentsline{toc}{subsection}{Castling limits}
\label{sec:Appendix/Summary/Castling limits}

\begin{table}[!h]
\centering
\begin{tabular}{ lcrrcrcrr }
\toprule % ===================================================================================================================================================================
\emph{Files}          & & \multicolumn{2}{c}{ \textbf{Queen-side} }             & & \textbf{King}               & & \multicolumn{2}{c}{ \textbf{King-side} }                \\
\cmidrule{1-1}          \cmidrule{3-4}                                            \cmidrule{6-6}                  \cmidrule{8-9} % ...........................................
\textbf{Variant}      & & \emph{min}                & \emph{max}                & & \emph{pos}                  & & \emph{min}                  & \emph{max}                \\
\midrule % -------------------------------------------------------------------------------------------------------------------------------------------------------------------
Classical Chess       & & \alg{c}                   & \alg{c}                   & & \alg{e}                     & & \alg{g}                     & \alg{g}                   \\
% \cmidrule{1-9} % ...........................................................................................................................................................
Croatian Ties         & & \alg{c}                   & \alg{d}                   & & \alg{f}                     & & \alg{h}                     & \alg{i}                   \\
\cmidrule{1-9} % .............................................................................................................................................................
Mayan                 & & \multirow{2}{*}{\alg{c}}  & \multirow{2}{*}{\alg{e}}  & & \multirow{2}{*}{\alg{g}}    & & \multirow{2}{*}{\alg{i}}    & \multirow{2}{*}{\alg{k}}  \\
Ascendancy            & &                           &                           & &                             & &                             &                           \\
\cmidrule{1-9} % .............................................................................................................................................................
Age of Aquarius       & & \alg{c}                   & \alg{f}                   & & \alg{h}                     & & \alg{j}                     & \alg{m}                   \\
Miranda's Veil        & & \alg{c}                   & \alg{g}                   & & \alg{i}                     & & \alg{k}                     & \alg{o}                   \\
Nineteen              & & \alg{d}                   & \alg{h}                   & & \alg{j}                     & & \alg{l}                     & \alg{p}                   \\
Hemera's Dawn         & & \alg{d}                   & \alg{i}                   & & \alg{k}                     & & \alg{m}                     & \alg{r}                   \\
\cmidrule{1-9} % .............................................................................................................................................................
Tamoanchan            & & \multirow{2}{*}{\alg{d}}  & \multirow{2}{*}{\alg{j}}  & & \multirow{2}{*}{\alg{l}}    & & \multirow{2}{*}{\alg{n}}    & \multirow{2}{*}{\alg{t}}  \\
Revisited             & &                           &                           & &                             & &                             &                           \\
\cmidrule{1-9} % .............................................................................................................................................................
Conquest of           & & \multirow{2}{*}{\alg{d}}  & \multirow{2}{*}{\alg{k}}  & & \multirow{2}{*}{\alg{m}}    & & \multirow{2}{*}{\alg{o}}    & \multirow{2}{*}{\alg{v}}  \\
Tlalocan              & &                           &                           & &                             & &                             &                           \\
\cmidrule{1-9} % .............................................................................................................................................................
Discovery             & & \alg{d}                   & \alg{k}                   & & \alg{m}                     & & \alg{o}                     & \alg{v}                   \\
One                   & & \alg{d}                   & \alg{l}                   & & \alg{n}                     & & \alg{p}                     & \alg{x}                   \\
\bottomrule % ================================================================================================================================================================
\end{tabular}
\caption{Castling limits, files}
\label{tbl:Appendix/Summary/Castling limits}
\end{table}

Table above contains minimum and maximum files a castling King can reach, separated
for Queen-side and King-side castling. Light and dark Kings share the same file at
the beginning of a game, listed in the middle column.\newline
\indent
Rook always ends on a field immediately to the right of a King in Queen-side castling,
and immediately to the left in King-side castling.\newline
\indent
For instance, in
\hyperref[sec:Appendix/Notation/Castling]{previous castling} examples,
\hyperref[fig:age_of_aquarius_castling_left_04]{Queen-side castling}
\alg{Kd\&e} is comprised of \alg{Kh1-d1} and \alg{Ra1-e1}, while
\hyperref[fig:one_castling_right_04]{King-side castling} \alg{Kr\&q} is a
composite of \alg{Kn1-r1} and \alg{Ry1-q1}.

\clearpage % ..........................................................

\subsection*{Transparency of pieces}
\addcontentsline{toc}{subsection}{Transparency of pieces}
\label{sec:Appendix/Summary/Transparency of pieces}

\begin{table}[!h]
\centering
\begin{tabular}{ lll }
\toprule % ===========================================================================================
\textbf{Stationary piece}           & \textbf{is ... to}            & \textbf{Moving piece}         \\
\midrule % -------------------------------------------------------------------------------------------
material\footnotemark[1],           & \multirow{2}{*}{opaque}       & material\footnotemark[1],     \\
except Starchild                    &                               & except Starchild              \\
\cmidrule{1-3} % .....................................................................................
Wave                                & transparent                   & all, except Monolith          \\
all, except Monolith                & transparent                   & Wave                          \\
Monolith                            & opaque                        & all, except Starchild         \\
all, except Starchild               & opaque                        & Monolith                      \\
Starchild                           & transparent                   & all                           \\
\bottomrule % ========================================================================================
\end{tabular}
\caption{Transparency of pieces}
\label{tbl:Appendix/Summary/Transparency of pieces}
\end{table}

\footnotetext[1]{Material is any piece, except Wave.}

A stationary \hyperref[fig:scn_mv_007_wave_is_transparent]{piece is transparent}
if moving piece is not blocked by it, and can continue movement past it; otherwise,
it's opaque. Transparency does not define if a moving piece can interact in any
other way with stationary piece, or not.

To utilize transparency, moving piece has to be able to make at least two steps;
one towards stationary piece, and one away. So, a single-step piece (e.g. Knight,
King, ...) cannot use transparency, even if it encounters transparent piece (e.g.
Wave). The same applies to activated pieces which doesn't have at least two momentum.
This is also the reason why in the table there is no record with Starchild listed
under \emph{moving piece} column.

\clearpage % ..........................................................

\subsection*{Piece activations}
\addcontentsline{toc}{subsection}{Piece activations}
\label{sec:Appendix/Summary/Piece activations}

\begin{table}[!h]
\centering
\begin{tabular}{ lll }
\toprule % ===============================================================================================
\multicolumn{2}{c}{ \textbf{Activating} }                               & \textbf{Activated}            \\
\cmidrule{1-2} % .........................................................................................
\emph{Piece}                            & \emph{At field}               &                               \\
\midrule % -----------------------------------------------------------------------------------------------
any\footnotemark[1]                     & capture-                      & own Pyramid                   \\
any\footnotemark[1]                     & any                           & own Wave                      \\
\cmidrule{1-3} % .........................................................................................
\multirow{2}{*}{Wave}                   & \multirow{2}{*}{any}          & opponent's Wave, any own      \\
                                        &                               & except King, Pyramid          \\
\cmidrule{1-3} % .........................................................................................
\multirow{2}{*}{Wave\footnotemark[2]}   & \multirow{2}{*}{capture-}     & opponent's Wave, any          \\
                                        &                               & own except King               \\
\cmidrule{1-3} % .........................................................................................
Shaman                                  & trance-                       & any Shaman, Starchild         \\
Starchild                               & step-                         & own Wave, Starchild           \\
Starchild                               & miracle-                      & any Star                      \\
\cmidrule{1-3} % .........................................................................................
\multirow{2}{*}{Starchild}              & \multirow{2}{*}{uplifting-}   & any\footnotemark[3] except King, Wave, \\
                                        &                               & Monolith, any Star            \\
\bottomrule % ============================================================================================
\end{tabular}
\caption{Piece activations}
\label{tbl:Appendix/Summary/Piece activations}
\end{table}

\footnotetext[1]{Any piece, except Stars, Monolith.}
\footnotetext[2]{If activated on capture-fields.}
\footnotetext[3]{Uplifted Starchild cannot take sense-journey, only uplift other piece.}

Activated own piece means it belongs to the same player (light or dark) as activating
piece, otherwise it's opponent's piece. Activated pieces labeled as any can be either
own or opponent's.\newline
\indent
For instance, any piece can activate own Wave, but only Wave can activate opponent's
Wave. Another example, Shaman can entrance any Shaman, Starchild, i.e. either own
or opponent's.

\clearpage % ..........................................................

\subsection*{Movement of Wave}
\addcontentsline{toc}{subsection}{Movement of Wave}
\label{sec:Appendix/Movement of Wave}

\begin{table}[!h]
\centering
\begin{tabular}{ ll }
\toprule % ===============================================================
\textbf{Activated by}       & \textbf{Moves like}                       \\
\midrule % ---------------------------------------------------------------
Pawn                        & Pawn\footnotemark[1]                      \\
Knight                      & Pegasus                                   \\
Bishop                      & Bishop                                    \\
Rook                        & Rook                                      \\
Queen                       & Queen                                     \\
King                        & Queen                                     \\
\cmidrule{1-2} % .........................................................
Pegasus                     & Pegasus                                   \\
Pyramid                     & Rook                                      \\
Unicorn                     & Centaur\footnotemark[1]\textsuperscript{,}\footnotemark[2] \\
Wave                        & activating Wave                           \\
Star                        & ---                                       \\
Centaur                     & Centaur                                   \\
Scout                       & Scout\footnotemark[1]                     \\
Grenadier                   & Queen\footnotemark[4]                     \\
Serpent                     & Serpent\footnotemark[1]\textsuperscript{,}\footnotemark[3] \\
Shaman                      & Shaman                                    \\
Monolith                    & ---                                       \\
Starchild                   & Starchild                                 \\
\bottomrule % ============================================================
\end{tabular}
\caption{Movement of Wave}
\label{tbl:Appendix/Movement of Wave}
\end{table}

\footnotetext[1]{Unlimited steps.}
\footnotetext[2]{Unrestricted two initial step choices.}
\footnotetext[3]{Two alternating steps.}
\footnotetext[4]{Horizontal and vertical steps are just movement steps.
Diagonal steps are capturing steps.}

Activated Wave generally moves the same way as activating piece. Wave can make
multiple steps in chosen direction, even if activating piece can make only a
single step. For instance, Wave activated by Knight moves like a Pegasus.

Activated Wave generally has the same initial steps as activating piece starting
a new ply. Once direction is chosen, Wave cannot change its heading for the
remainder of its ply. For instance,
\hyperref[fig:scn_mv_020_wave_activated_by_king]{Wave activated by King} cannot
change its movement to horizontal once it started moving diagonally.

Wave can take any initial step, even those activating piece could not. For instance,
\hyperref[fig:scn_hd_25_scout_activating_wave_step_fields_end]{Wave activated by Scout}
can take diagonal step, even if Scout itself would not be able to because diagonal
field does not host opponent's piece.

\hyperref[fig:scn_mv_027_wave_activation_by_unicorn_first_step]{Wave activated by Unicorn}
moves similar to Centaur, it has to keep alternating between two initially chosen
steps. Unlike Centaur, Wave is not restricted in choosing second step based on a
choice of the first one. Wave can choose any two consecutive steps Unicorn could
make from its starting field and first step-field.

\hyperref[fig:scn_tr_33_serpent_activating_wave]{Wave activated by Serpent} moves
like a Serpent, but it's restricted to alternating between two initially chosen
diagonal steps, which also has to be on different diagonals.

Wave can also be activated by another Wave, including opponent's. Activated Wave
inherits its initial steps from (and behaves the same as) activating Wave;
inheritance chain starts with last material (non-Wave) piece which precedes Waves
in a cascade. Such a material piece is called
\hyperref[sec:Terms/Activator]{activator}.

Wave cannot capture opponent's pieces, all of its fields are steps-fields. When
refering to Wave's capture-fields it's just a shorthand for capture-fields of its
activator. Pyramid can be activated by Wave only if both activating Wave and its
activator has been moving over their capture-fields; other Waves between the two
can move over their step-fields.\newline
\indent
For instance,
\hyperref[fig:scn_mv_024_wave_activated_by_capture_pawn]{Wave activated by Pawn on its capture-field}
has the same initial steps as
\hyperref[fig:scn_mv_022_wave_activated_by_step_pawn]{Wave activated by Pawn on its step-field};
but, only the former can activate Pyramids, if moving diagonally, over its
capture-fields.

\hyperref[fig:scn_hd_54_grenadier_activated_wave_step_field]{Wave activated by Grenadier}
moves like a Queen; difference is that only diagonal steps are capture-steps, both
horizontal and vertical steps are just movement steps.

Wave cannot be activated by a Star or a Monolith. Wave can teleport, if activated
by any piece, but Starchild. Wave activated by Starchild cannot neither teleport,
nor activate a Star.

All other properties of Wave movement remains the same, regardless which piece
activated Wave, and on which (step- or capture-) field:
\vspace*{-1.1\baselineskip}
\begin{itemize}
    % \vspace*{-0.7\baselineskip}
    \item Wave can step over pieces (except Monolith);
    \vspace*{-0.7\baselineskip}
    \item Wave cannot capture opponent's pieces;
    \vspace*{-0.7\baselineskip}
    \item Wave can activate own pieces, except King (Pyramid only on capture-fields);
    \vspace*{-0.7\baselineskip}
    \item Wave can activate opponent's Wave;
    \vspace*{-0.7\baselineskip}
    \item Wave can leave chessboard, as long as its ply end on a chessboard field;
    \vspace*{-0.7\baselineskip}
    \item Wave transfers all of received momentum to a piece it activates.
\end{itemize}

\clearpage % ..........................................................

\subsection*{Piece actions}
\addcontentsline{toc}{subsection}{Piece actions}
\label{sec:Appendix/Summary/Piece actions}

\begin{table}[!h]
\centering
\begin{tabular}{ llll }
\toprule % ===========================================================================================================
\textbf{Piece}              & \textbf{Fields}           & \multicolumn{2}{c}{ \textbf{Action at ... step} }         \\
                                                        \cmidrule{3-4} % .............................................
                            &                           & \emph{any}\footnotemark[1]    & \emph{last}               \\
\midrule % -----------------------------------------------------------------------------------------------------------
\multirow{4}{*}{King}       & \multirow{4}{*}{any\footnotemark[2]}
                                                        & \multirow{4}{*}{---}          & step, capture,            \\
                            &                           &                               & castling,                 \\
                            &                           &                               & activate own              \\
                            &                           &                               & Wave, Pyramid             \\
\cmidrule{1-4} % .....................................................................................................
                            & \multirow{4}{*}{any\footnotemark[2]}
                                                        &                               & step, capture,            \\
Knight,                     &                           & ---\footnotemark[3],          & teleport,                 \\
Unicorn                     &                           & divergence\footnotemark[4]    & activate own              \\
                            &                           &                               & Wave, Pyramid             \\
% \cmidrule{1-4} % .....................................................................................................
\bottomrule % ========================================================================================================
\end{tabular}
\caption{Piece actions in a single ply, part 1}
\label{tbl:Appendix/Summary/Piece actions, part 1}
\end{table}

\footnotetext[1]{Any step except the last one.}
\footnotetext[2]{Step- and capture-fields are the same.}
\footnotetext[3]{When not diverging.}
\footnotetext[4]{Only when diverging, for only one additional step.}

Table on this and next few pages shows what actions a piece can do, and on what
fields. Actions are usually side-effects to a step (e.g. a capture), and sometimes
ply separators if more than just a simple cascade (e.g. teleportation). Steps without
side-effect are written just as \emph{step} in \emph{action} columns.

Single-step pieces have available actions listed in the \emph{last} step column,
and preceding steps are left empty, i.e. containing just \emph{---} (long dash).

Divergence is always listed in \emph{any} step column, because a piece continues
its ply after diverging. As a result, no piece can diverge on its last step.

\clearpage % ..........................................................

\begin{table}[!h]
\centering
\begin{tabular}{ llll }
\toprule % ===========================================================================================================
\textbf{Piece}              & \textbf{Fields}           & \multicolumn{2}{c}{ \textbf{Action at ... step} }         \\
                                                        \cmidrule{3-4} % .............................................
                            &                           & \emph{any}\footnotemark[1]    & \emph{last}               \\
\midrule % -----------------------------------------------------------------------------------------------------------
Bishop,                     & \multirow{4}{*}{any\footnotemark[2]}
                                                        & \parbox[b][1pt][s]{12ex}{step,\\transparency,\\divergence}
                                                                                        & step, capture,            \\
Rook,                       &                           &                               & teleport,                 \\
Queen,                      &                           &                               & activate own              \\
Pegasus                     &                           &                               & Wave, Pyramid             \\
\cmidrule{1-4} % .....................................................................................................
\multirow{7}{*}{Pyramid}    & \multirow{7}{*}{any\footnotemark[2]}
                                                        &                               & step, capture,            \\
                            &                           &                               & teleport,                 \\
                            &                           & step,                         & Pawn-sacrifice,           \\
                            &                           & transparency,                 & promoting\footnotemark[3] Pawn, \\
                            &                           & divergence                    & conversion,               \\
                            &                           &                               & activate own              \\
                            &                           &                               & Wave, Pyramid             \\
% \cmidrule{1-4} % .....................................................................................................
\bottomrule % ========================================================================================================
\end{tabular}
\caption{Piece actions in a single ply, part 2}
\label{tbl:Appendix/Summary/Piece actions, part 2}
\end{table}

\footnotetext[1]{Any step except the last one.}
\footnotetext[2]{Step- and capture-fields are the same.}
\footnotetext[3]{Includes tagging Pawn for promotion.}

Usually, actions on the last step in a ply are different than those on preceding
steps. For instance, most pieces can capture opponent's piece only on the last step.

Transparency and divergence are momentum restricted; activated piece has to have at
least one momentum when transparent or divergent piece is encountered to perform
action.

The same applies when activating pieces other than Wave or Starchild; activating
piece has to have at least one momentum to activate, e.g. a Pyramid.

\clearpage % ..........................................................

\begin{table}[!h]
\centering
\begin{tabular}{ llll }
\toprule % ===========================================================================================================
\textbf{Piece}              & \textbf{Fields}           & \multicolumn{2}{c}{ \textbf{Action at ... step} }         \\
                                                        \cmidrule{3-4} % .............................................
                            &                           & \emph{any}\footnotemark[1]    & \emph{last}               \\
\midrule % -----------------------------------------------------------------------------------------------------------
\multirow{4}{*}{Pawn}       & \multirow{4}{*}{step-}    & ---\footnotemark[3],          & step, teleport,           \\
                            &                           & step\footnotemark[4],         & promotion,                \\
                            &                           & transparency\footnotemark[4], & activate own              \\
                            &                           & divergence\footnotemark[5]    & Wave                      \\
\cmidrule{1-4} % .....................................................................................................
\multirow{5}{*}{Pawn}       & \multirow{5}{*}{capture-} & \multirow{5}{*}{---}          & capture, teleport,        \\
                            &                           &                               & promotion\footnotemark[6],\\
                            &                           &                               & en passant,               \\
                            &                           &                               & activate own              \\
                            &                           &                               & Wave, Pyramid             \\
% \cmidrule{1-4} % .....................................................................................................
\bottomrule % ========================================================================================================
\end{tabular}
\caption{Piece actions in a single ply, part 3}
\label{tbl:Appendix/Summary/Piece actions, part 3}
\end{table}

\footnotetext[1]{Any step except the last one.}
\footnotetext[2]{Step- and capture-fields are the same.}
\footnotetext[3]{When not rushing, diverging.}
\footnotetext[4]{Only when rushing.}
\footnotetext[5]{Only when diverging, for only one additional step.}
\footnotetext[6]{Including capture + promotion as side-effects of a single step.}

Transparency is not available to pieces which can make only one step in a single
ply, because moving piece has to make at least one step away from encountered,
transparent piece. Similar to divergence, no piece can use transparency on its
last step.

Single-step pieces include Knight and Unicorn on previous page, and Pawn in the
table above. Unlike Knight and Unicorn, Pawn can make more than one step when
rushing, and then it can use transparency to e.g. step over Wave as if not
present.

\clearpage % ..........................................................

\begin{table}[!h]
\centering
\begin{tabular}{ llll }
\toprule % ===========================================================================================================
\textbf{Piece}              & \textbf{Fields}           & \multicolumn{2}{c}{ \textbf{Action at ... step} }         \\
                                                        \cmidrule{3-4} % .............................................
                            &                           & \emph{any}\footnotemark[1]    & \emph{last}               \\
\midrule % -----------------------------------------------------------------------------------------------------------
\multirow{5}{*}{Wave}       & \multirow{5}{*}{step\footnotemark[2]-}
                                                        &                               & step, teleport,           \\
                            &                           & step,                         & activate any              \\
                            &                           & transparency,                 & Wave, own                 \\
                            &                           & divergence\footnotemark[5]    & pieces except             \\
                            &                           &                               & King, Pyramid             \\
\cmidrule{1-4} % .....................................................................................................
\multirow{5}{*}{Wave}       & \multirow{5}{*}{capture\footnotemark[3]-}
                                                        &                               & step, teleport,           \\
                            &                           & step,                         & activate any              \\
                            &                           & transparency,                 & Wave, own                 \\
                            &                           & divergence\footnotemark[5]    & pieces except             \\
                            &                           &                               & King                      \\
\cmidrule{1-4} % .....................................................................................................
\multirow{5}{*}{Wave}       & \multirow{5}{*}{miracle\footnotemark[4]-}
                                                        & \multirow{5}{*}{---}          & step, teleport,           \\
                            &                           &                               & activate any              \\
                            &                           &                               & Wave, own                 \\
                            &                           &                               & pieces except             \\
                            &                           &                               & King                      \\
\cmidrule{1-4} % .....................................................................................................
Star                        & step-                     & ---                           & step                      \\
% \cmidrule{1-4} % .....................................................................................................
\bottomrule % ========================================================================================================
\end{tabular}
\caption{Piece actions in a single ply, part 4}
\label{tbl:Appendix/Summary/Piece actions, part 4}
\end{table}

\footnotetext[1]{Any step, except the last one.}
\footnotetext[2]{Activated on a step-field.}
\footnotetext[3]{Activated on a capture--field.}
\footnotetext[4]{Activated on a miracle-field.}
\footnotetext[5]{Cannot diverge if activated by Unicorn, Centaur, or Serpent.}

Wave cannot capture, and so all fields it's traversing are actually step-fields.
In the table above, \emph{fields} column for Wave notes over which kind of fields
activating piece was traveling. For instance, if activated by rushing Pawn, Wave
cannot activate Pyramid, because rushing Pawn moves over its own step-fields, and
so first row applies.

\clearpage % ..........................................................

\begin{table}[!h]
\centering
\begin{tabular}{ llll }
\toprule % ===========================================================================================================
\textbf{Piece}              & \textbf{Fields}           & \multicolumn{2}{c}{ \textbf{Action at ... step} }         \\
                                                        \cmidrule{3-4} % .............................................
                            &                           & \emph{any}\footnotemark[1]    & \emph{last}               \\
\midrule % -----------------------------------------------------------------------------------------------------------
\multirow{4}{*}{Centaur}    & \multirow{4}{*}{any\footnotemark[2]}
                                                        &                               & step, capture,            \\
                            &                           & step,                         & teleport,                 \\
                            &                           & transparency                  & activate own              \\
                            &                           &                               & Wave, Pyramid             \\
\cmidrule{1-4} % .....................................................................................................
\parbox[b][1pt][s]{9ex}{Scout,\\Grenadier} % \multirow{3}{*}{troopers}
                            & \multirow{3}{*}{step-}    & step,                         & step, teleport,           \\
                            &                           & transparency,                 & activate own              \\
                            &                           & divergence                    & Wave                      \\
\cmidrule{1-4} % .....................................................................................................
                            & \multirow{4}{*}{capture-} & \multirow{4}{*}{---}          & capture, teleport,        \\
Scout,                      &                           &                               & en passant,               \\
Grenadier                   &                           &                               & activate own              \\
                            &                           &                               & Wave, Pyramid             \\
\cmidrule{1-4} % .....................................................................................................
\multirow{5}{*}{Serpent}    & \multirow{5}{*}{any\footnotemark[2]}
                                                        & \parbox[b][1pt][s]{13ex}{step,\\transparency,\\displacement\footnotemark[3],\\capture\footnotemark[4]\textsuperscript{,}\footnotemark[5]}
                                                                                        & step, capture\footnotemark[5], \\
                            &                           &                               & teleport,                 \\
                            &                           &                               & displacement\footnotemark[3], \\
                            &                           &                               & activate own              \\
                            &                           &                               & Wave, Pyramid             \\
% \cmidrule{1-4} % .....................................................................................................
\bottomrule % ========================================================================================================
\end{tabular}
\caption{Piece actions in a single ply, part 5}
\label{tbl:Appendix/Summary/Piece actions, part 5}
\end{table}

\footnotetext[1]{Any step, except the last one.}
\footnotetext[2]{Step- and capture-fields are the same.}
\footnotetext[3]{Serpent can displace only Pawns.}
\footnotetext[4]{Only after Pawn-sacrifice.}
\footnotetext[5]{After Pawn-sacrifice, Serpent can capture only Pawns.}

Only after Pawn-sacrifice Serpent can capture at any step; only Pawns can be
captured, regardless if opponent's or own. Otherwise, Serpent (without Pawn-sacrifice
tag) can capture any opponent's piece, but only at the very last step in its ply,
just like all the other \hyperref[sec:Terms/Materiel]{materiel pieces} except Shaman.\newline
\indent
Unlike other \hyperref[sec:Terms/Material]{material pieces}, Centaur and Serpent
cannot diverge.

\clearpage % ..........................................................

\begin{table}[!h]
\centering
\begin{tabular}{ llll }
\toprule % ===========================================================================================================
\textbf{Piece}              & \textbf{Fields}           & \multicolumn{2}{c}{ \textbf{Action at ... step} }         \\
                                                        \cmidrule{3-4} % .............................................
                            &                           & \emph{any}\footnotemark[1]    & \emph{last}               \\
\midrule % -----------------------------------------------------------------------------------------------------------
\multirow{3}{*}{Shaman}     & \multirow{3}{*}{step-\footnotemark[2]}
                                                        & step,                         & step, teleport,           \\
                            &                           & transparency,                 & activate own              \\
                            &                           & divergence                    & Wave                      \\
\cmidrule{1-4} % .....................................................................................................
\multirow{3}{*}{Shaman}     & \multirow{3}{*}{capture-} & \parbox[b][1pt][s]{10ex}{capture,\\divergence}
                                                                                        & capture, teleport,        \\
                            &                           &                               & activate own              \\
                            &                           &                               & Wave, Pyramid             \\
\cmidrule{1-4} % .....................................................................................................
Shaman                      & trance-                   & ---                           & trance-journey            \\
\cmidrule{1-4} % .....................................................................................................
\parbox[b][1pt][s]{9ex}{entranced\\Shaman}
                            & \multirow{3}{*}{step-\footnotemark[3]}
                                                        & step, capture,                & \parbox[b][1pt][s]{14ex}{step, capture,\\displacement}
                                                                                                                    \\
                            &                           & transparency,                 &                           \\
                            &                           & displacement                  &                           \\
% \cmidrule{1-4} % .....................................................................................................
\bottomrule % ========================================================================================================
\end{tabular}
\caption{Piece actions in a single ply, part 6}
\label{tbl:Appendix/Summary/Piece actions, part 6}
\end{table}

\footnotetext[1]{Any step, except the last one.}
\footnotetext[2]{Ordinary step, i.e. not during trance-journey.}
\footnotetext[3]{Only during trance-journey.}

Captures during double trance-journey are mandatory, unlike regular trance-journey
during which both captures and displacements are optional.

Note that double trance-journey itself is optional undertaking, dark Shaman entranced
by other dark Shaman can choose whether to take on regular or double trance-journey.

\clearpage % ..........................................................

\begin{table}[!h]
\centering
\begin{tabular}{ llll }
\toprule % ===========================================================================================================
\textbf{Piece}              & \textbf{Fields}           & \multicolumn{2}{c}{ \textbf{Action at ... step} }         \\
                                                        \cmidrule{3-4} % .............................................
                            &                           & \emph{any}\footnotemark[1]    & \emph{last}               \\
\midrule % -----------------------------------------------------------------------------------------------------------
\multirow{2}{*}{Monolith}   & \multirow{2}{*}{step-}    & \multirow{2}{*}{step}         & step, demoting            \\
                            &                           &                               & to Pawn                   \\
\cmidrule{1-4} % .....................................................................................................
\multirow{3}{*}{Starchild}  & \multirow{3}{*}{step-}    & \multirow{3}{*}{---}          & step, resurrection,       \\
                            &                           &                               & activate own              \\
                            &                           &                               & Wave, Starchild           \\
\cmidrule{1-4} % .....................................................................................................
\multirow{5}{*}{Starchild}  & \multirow{5}{*}{miracle-} & \multirow{5}{*}{---}          & activate any Star,        \\
                            &                           &                               & opponent's                \\
                            &                           &                               & Starchild,                \\
                            &                           &                               & own pieces                \\
                            &                           &                               & except King               \\
\cmidrule{1-4} % .....................................................................................................
Starchild                   & uplifting-                & ---                           & sense-journey             \\
% \cmidrule{1-4} % .....................................................................................................
\bottomrule % ========================================================================================================
\end{tabular}
\caption{Piece actions in a single ply, part 7}
\label{tbl:Appendix/Summary/Piece actions, part 7}
\end{table}

\footnotetext[1]{Any step, except the last one.}

While Waves, Starchilds are resurrected on miracle-fields, to initiate resurrection
Starchild has to step into appropriate syzygy, so resurrection is an action following
step onto step-field, not miracle-field.

\clearpage % ..........................................................

\section*{Grammar}
\addcontentsline{toc}{section}{Grammar}
\label{sec:Appendix/Grammar}

\begin{table}[!h]
\centering
\begin{tabular}{ ll }
\toprule % ===============================================================
\textbf{Entity}             & \textbf{Meaning}                          \\
\midrule % ---------------------------------------------------------------
\algfmt{< >}                & named entity, to be substituted           \\
\algfmt{|}                  & choice between 2 values, can be chained   \\
\algfmt{[ ]}                & optional item(s)                          \\
\algfmt{( )}                & grouping item(s), scope                   \\
\algfmt{\{ \}}              & repeating item(s), one or more times      \\
% \algfmt{\#number}           & numbered reference                        \\
\algfmti{\$}                & line comment                              \\
% \algfmt{?}                  & empty value                               \\
\alg{\_}                    & space                                     \\
\alg{value}                 & verbatim \algfmt{AN} value                \\
\algcty{value}              & compatibility value, for \algfmt{CAN}     \\
\bottomrule % ============================================================
\end{tabular}
\caption{Grammar}
\label{tbl:Appendix/Grammar}
\end{table}

Here, notational grammar is described in more formal, concise way. Annotations
are not covered, as they are short-hand for non-standardized commentary.

Verbatim values (e.g. \alg{x}) are as they appear in \algfmt{AN}, compatibility
values (e.g. \algcty{x}) are used exclusively for \algfmt{CAN}. Empty value is
represented just with an empty group, i.e. \algfmt{()}; it's used to denote when
there is no special move status, like check or checkmate.

Value separator \algfmt{|} is used to present choice between 2 values, e.g.
\alg{a} \algfmt{|} \alg{b}. Choices can be chained, as in
\alg{a} \algfmt{|} \alg{b} \algfmt{|} \alg{c}.

Optional items are enclosed in \algfmt{[ ]} for items to appear or not. Items are
grouped by enclosing them in \algfmt{( )}, which envelops them into a scope.
Repeating items are enclosed in \algfmt{\{ \}}, to be repeated at least once.

\clearpage % ..........................................................

Formatting and spacing is added to improve legibility, normally \algfmt{AN} is
written without any gaps. When space is needed, it is written as \alg{\_}
(underscore). Choices and groups are also valid regardless of formatting,
spacing, e.g.
\begin{alltt}
<abc> = \alg{def}
    \alg{ghi}
  | \alg{jkl}
\end{alltt}
means \algfmt{<abc>} is to be replaced with either \alg{defghi} or \alg{jkl}.

Choices are capturing complete values separated by \algfmt{|}, or to the either
end of definition. For instance,
\begin{alltt}
<abc> = \alg{a} | \alg{b} <cde> \alg{f} | \alg{g}
\end{alltt}
has one choice with 3 distinct values, \alg{a}, \alg{b}\algfmt{<cde>}\alg{f} and
\alg{g}; and \emph{not} two choices, each with 2 distinct values, namely \alg{a} or
\alg{b}, \alg{f} or \alg{g}.

Groups can contain other groups, if they do, they are valid until first matching
closing bracket. Each closing bracket always closes exactly one group. For instance,
\begin{alltt}
<abc> = \alg{a} [ \alg{b} [ \alg{c} ] \alg{d} ] \alg{e}
\end{alltt}
gives \alg{abcde}, \alg{abde}, \alg{ae} for \algfmt{<abc>}.
Brackets cannot overlap, i.e. if group contains other group, it must be contained in
its entirety. For instance:
\begin{alltt}
<abc> = \alg{a} [ \alg{b} ( \alg{c} ] \alg{d} ) \alg{e}
\end{alltt}
is not valid example, because first group \algfmt{[ ]} is closed containing
dangling (open, but not closed) second group \algfmt{( )}.

\clearpage % ..........................................................

Choices are fully contained within enclosing group. For instance,
\begin{alltt}
<abc> = \alg{d} | \alg{e} [ \alg{f} | \alg{g} ] \alg{h} | \alg{i}
\end{alltt}
has two choices. Choice inside option group \algfmt{[ ]} has 2 possible values,
either \alg{f} or \alg{g}. Global choice (not contained in any group) has 3 possible
values: \alg{d}, \alg{e}\algfmt{[}\alg{f}\algfmt{|}\alg{g}\algfmt{]}\alg{h} or \alg{i}.
This gives \alg{d}, \alg{efh}, \alg{egh}, \alg{eh} and \alg{i} as possible values for
\algfmt{<abc>}.

Choices can be limited in scope by enclosing them in \algfmt{( )},
for instance
\begin{alltt}
<abc> = \alg{d} | \alg{e} ( \alg{f} | \alg{g} ) \alg{h} | \alg{i}
\end{alltt}
has two choices. Choice inside group \algfmt{( )} has 2 possible values,
either \alg{f} or \alg{g}; unlike previous example this group is not optional, so one
value is guaranteed to appear. Global choice (not contained in any group) has 3 possible
values: \alg{d}, \alg{e}\algfmt{(}\alg{f}\algfmt{|}\alg{g}\algfmt{)}\alg{h}
or \alg{i}. This gives \alg{d}, \alg{efh}, \alg{egh}, and \alg{i} as possible values
for \algfmt{<abc>}.

Items enclosed in \algfmt{\{ \}} brackets are to be repeated at least once. For instance,
\begin{alltt}
<abc> = \alg{d} \{ \alg{e} \} \alg{f}
\end{alltt}
gives \alg{def}, \alg{deef}, \alg{deeef}, \alg{deeeef}, ... for \algfmt{<abc>}.\newline
\indent
Note, this is different than definition in
\href{https://en.wikipedia.org/wiki/Extended_Backus%E2%80%93Naur_form#Table_of_symbols}{Extended Backus–Naur form},
which states that items are optionally repeated, i.e. zero or more times.\newline
\indent
To have the same definition here, repeating group can be enclosed into option group,
like so
\begin{alltt}
<abc> = \alg{d} [ \{ \alg{e} \} ] \alg{f}
\end{alltt}
which gives \alg{df}, \alg{def}, \alg{deef}, \alg{deeef}, ... for \algfmt{<abc>}.

Line comments are written using \algfmti{\$}, everything from \algfmti{\$} up to the
end of line is disregarded. For instance, this example
\begin{alltt}
\algfmti{\$ Entity <abc> ...}
<abc> = \algfmti{\$ ... is either value d or \alg{e}.}
  \alg{d} \algfmti{\$ Value can be commented with no}
    \algfmti{\$ additional formatting, as in d.}
| \alg{e} \algfmti{\$ Value in comment with added weight}
    \algfmti{\$ (as in \alg{e}) shouldn't be confused for}
    \algfmti{\$ compatibility value; it's used just}
    \algfmti{\$ for additional visual emphasis.}
\end{alltt}
is by definition the same as the next one:
\begin{alltt}
<abc> = \alg{d} | \alg{e}
\end{alltt}

Rule definitions can be chained, so
\begin{alltt}
<abc> = <def> = \alg{g} [ \alg{h} ] \alg{i}
\end{alltt}
gives both entities \algfmt{<abc>} and \algfmt{<def>} the same definition,
i.e. it's just a shorthand for
\begin{alltt}
<def> = \alg{g} [ \alg{h} ] \alg{i}
<abc> = <def>
\end{alltt}

\clearpage % ..........................................................

Empty value is represented with an empty group \algfmt{()}, like so
\begin{alltt}
<abc> = () | \alg{d}
\end{alltt}

While this is effectively the same as optional value, e.g.
\begin{alltt}
<abc> = [ \alg{d} ]
\end{alltt}

or usage of optional entity, e.g.
\begin{alltt}
<abc> = \alg{d}
<def> = \alg{g} [ <abc> ]
\end{alltt}
the first definition is used when neither value, nor entity is optional,
based on external context.

Grammar is written without context, to determine which pieces, files and ranks
are available for a particular variant see
\hyperref[tbl:Appendix/Introduction/Variants]{Variants},
\hyperref[tbl:Appendix/Introduction/Chessboards]{Chessboards} and
\hyperref[tbl:Appendix/Introduction/Pieces]{Pieces} tables.

Side-effects, and some path separators are introduced by pieces, and so could
be missing, if variant prior to One is being played. For instance, first
divergent piece (Shaman) is introduced in Conquest of Tlalocan variant, so any
variant before that will not be using divergence side-effect. Similarly, none
of teleportation path separators are used before Nineteen variant, when first
teleporting piece (Star) is introduced.

\clearpage % ..........................................................

\begin{alltt}
<pawn> = [\alg{P}]
<classic-figure> = \alg{N} | \alg{B} | \alg{R} | \alg{Q} | \alg{K}
<trooper> = \alg{O} | \alg{G}
<private> = <pawn> | <trooper>
<passive-figure> = \alg{A} | \alg{W}

<new-active-figure> = \alg{E} | \alg{U} | \alg{C} | \alg{S} | \alg{H}
                    | <trooper>

<capturing-active-figure> =
  <new-active-figure> | <classic-figure>

<capturing-active-piece> =
  <capturing-active-figure> | <pawn>

<active-figure> =
  <capturing-active-figure> | \alg{I}

<active-piece> =
  <active-figure> | <pawn>

<capturing-piece> =
  <capturing-active-piece> | \alg{A}

<activateable-figure> = \alg{N} | \alg{B} | \alg{R} | \alg{Q} | \alg{I}
                      | <new-active-figure>
                      | <passive-figure>

<activateable-piece> =
  <activateable-figure> | <pawn>

<disposable-figure> =
<promote-to-figure> =
  <activateable-figure>
\end{alltt}

\clearpage % ..........................................................

\begin{alltt}
<disposable-piece> =
  <disposable-figure> | <pawn>

<piece-transparent-to-wave> =
  <disposable-piece> | \alg{K} | \alg{T}

<file> = \alg{a} | \alg{b} | \alg{c} | \alg{d} | \alg{e} | \alg{f} | \alg{g} | \alg{h}
| \alg{i} | \alg{j} | \alg{k} | \alg{l} | \alg{m} | \alg{n} | \alg{o} | \alg{p} | \alg{q}
| \alg{r} | \alg{s} | \alg{t} | \alg{u} | \alg{v} | \alg{w} | \alg{x} | \alg{y} | \alg{z}

<rank> = \alg{1} | \alg{2} | \alg{3} | \alg{4} | \alg{5} | \alg{6} | \alg{7} | \alg{8} | \alg{9}
| \alg{10} | \alg{11} | \alg{12} | \alg{13} | \alg{14} | \alg{15} | \alg{16} | \alg{17} | \alg{18}
| \alg{19} | \alg{20} | \alg{21} | \alg{22} | \alg{23} | \alg{24} | \alg{25} | \alg{26}

<field> = <file><rank>
<disambiguation> = <file> | <rank> | <field>
<step> = \alg{.}[\alg{.}]<field>

<losing-promotion-tag> = [\alg{==}]
<losing-rushing-tag> = [\alg{::}]
<losing-castling-tag> = [\alg{\&\&}]

<losing-pawn-tag> =
  <losing-rushing-tag>
| <losing-promotion-tag>

<figure-losing-tag> =
  \alg{R}<losing-castling-tag>
| <trooper><losing-rushing-tag>

<active-figure-losing-tag> =
  <active-figure>
| <figure-losing-tag>
\end{alltt}

\clearpage % ..........................................................

\begin{alltt}
<active-piece-losing-tag> =
  <active-figure-losing-tag>
| <pawn><losing-pawn-tag>

<capturing-active-figure-losing-tag> =
  <capturing-active-figure>
| <figure-losing-tag>

<capturing-active-piece-losing-tag> =
  <capturing-active-figure-losing-tag>
| <pawn><losing-pawn-tag>

<capturing-piece-losing-tag> =
  <capturing-active-piece-losing-tag>
| <capturing-piece>

<activateable-figure-losing-tag> =
  <activateable-figure>
| <capturing-active-figure-losing-tag>

<activateable-piece-losing-tag> =
  <activateable-figure-losing-tag>
| <pawn><losing-pawn-tag>

<disposable-figure-losing-tag> =
  <disposable-figure>
| <capturing-active-figure-losing-tag>

<disposable-piece-losing-tag> =
  <disposable-figure-losing-tag>
| <pawn><losing-pawn-tag>

<capturing-side-effect> =
  [\alg{*}[<disposable-piece-losing-tag>]]
\end{alltt}

\clearpage % ..........................................................

\begin{alltt}
<en-passant-side-effect> =
  [\alg{:}[[<private>](<rank> | <field>)]]

<castling-side-effect> =
  [\alg{&}[[\alg{R}](<file> | <field>)]]

<promoting-side-effect> =
  [\alg{=}]<promote-to-figure>

<delayed-promotion-side-effect> = [\alg{=}]

<converting-side-effect> =
  \alg{%}[<disposable-piece-losing-tag>]
| \alg

<pawn-displacement-side-effect> =
  \alg{<}<pawn><losing-pawn-tag><disambiguation>

<displacement-side-effect> =
  \alg{<}[<disposable-piece-losing-tag>]<field>

<transparency-side-effect> = [\alg{\^{}}[\alg{W}|\alg{I}]]
<transparency-starchild-side-effect> = [\alg{\^{}}[\alg{I}]]
<divergence-side-effect> = [\alg{/}[\alg{H}|\alg{I}]]

<wave-transparency-side-effect> =
  [\alg{\^{}}[<piece-transparent-to-wave>]]

<demoting-side-effect> =
  \alg{>}[<disposable-figure-losing-tag>]
    <disambiguation>
| \alg{>}<disposable-figure-losing-tag>
    [<disambiguation>]
\end{alltt}

\clearpage % ..........................................................

\begin{alltt}
<resurrecting-side-effect> =
  \alg{$}[\alg{$}]<disposable-piece>
| \alg{$}[\alg{$}](\alg{W}|\alg{I})<field>
| [\alg{$$$}]

<stepping-no-side-effects> =
  [<disambiguation>]<field>
| [<disambiguation>]\{<step>\}[\alg{-}<field>]
| [<field>\alg{-}]<field>

<stepping> =
  <stepping-no-side-effects>
| [<disambiguation>]
  \{<step>[<transparency-side-effect>
          | <divergence-side-effect>]\}
  (<step> | \alg{-}<field>)

<wave-stepping> =
  <stepping-no-side-effects>
| [<disambiguation>]
  \{<step>[<wave-transparency-side-effect>
          | <divergence-side-effect>]\}
  (<step> | \alg{-}<field>)

<serpent-stepping> =
  <stepping-no-side-effects>
| [<disambiguation>]
  \{<step>[<pawn-displacement-side-effect>
          | <transparency-side-effect>]\}
  (<step> | \alg{-}<field>)

\end{alltt}

\clearpage % ..........................................................

\begin{alltt}
<shaman-stepping> =
  <stepping-no-side-effects>
| [<disambiguation>]
  \{<step>[<transparency-side-effect>
          | <divergence-side-effect>
          | <capturing-side-effect>]\}
  (<step> | \alg{-}<field>)

<pawn-promotion-ply> =
  <pawn><losing-promotion-tag><stepping>
  (<promoting-side-effect>
   | <delayed-promotion-side-effect>)

<capturing-pawn-promotion-ply> =
  <pawn><losing-promotion-tag><stepping>
  [<capturing-side-effect>]
  <promoting-side-effect>

<pyramid-promoting-ply> =
  \alg{A}<stepping>
  (<promoting-side-effect>
   | <delayed-promotion-side-effect>)

<pyramid-converting-ply> =
  \alg{A}<stepping><converting-side-effect>

<teleportation-ply> =
  \alg{|}[<disposable-piece>]<field>
| \alg{||}[<disposable-piece>]<field>
| \alg{|||}[<disposable-piece>]
\end{alltt}

\clearpage % ..........................................................

\begin{alltt}
<pawn-sacrifice-init> =
  \alg{S}<serpent-stepping>
  \alg{~A}<stepping>[\alg{*}[<losing-pawn-tag>]]\alg{;;}

<pawn-sacrifice-steps> =
  \alg{S}[<disambiguation>]
    \{<step>[\alg{*}[<losing-pawn-tag>]
            | <pawn-displacement-side-effect>
            | <transparency-side-effect>]\}
| \alg{S}<stepping-no-side-effects>

<pawn-sacrifice-ply> =
  <pawn-sacrifice-init>
  [<pawn-sacrifice-steps>]

<starting-ply> =
  <active-piece-losing-tag><stepping>
| \alg{S}<serpent-stepping>
| \alg{H}<shaman-stepping>
| <pawn-promotion-ply>
| \alg{K}<losing-castling-tag><stepping>
| \alg{I}<stepping-no-side-effects>
    [<resurrecting-side-effect>]
| <pawn-sacrifice-ply>

<cascading-plies> =
  [\{(\alg{~W}<wave-stepping> | \alg{~A}<stepping>)\}]
  \{\alg{~W}<wave-stepping>\}
  [ [\{((\alg{~}|\alg{|})\alg{W}<wave-stepping>
       | \alg{~A}<stepping>)\}]
    \{(\alg{~}|\alg{|})\alg{W}<wave-stepping>\} ]
\end{alltt}

\clearpage % ..........................................................

\begin{alltt}
<stand-alone-ply> =
  <capturing-piece-losing-tag><stepping>
  [(<capturing-side-effect>
    | <teleportation-ply>)]
| <private-losing-tag><stepping>
  [(<en-passant-side-effect>
    | <teleportation-ply>)]
| \alg{S}<serpent-stepping>
  [(<capturing-side-effect>
    | <teleportation-ply>)]
| \alg{H}<shaman-stepping>
  [(<capturing-side-effect>
    | <teleportation-ply>)]
| <capturing-pawn-promotion-ply>
| <pawn-sacrifice-ply>[<teleportation-ply>]

<terminating-ply> =
  <stand-alone-ply>
| <activateable-piece-losing-tag><stepping>
  [<teleportation-ply>]
| <pyramid-promoting-ply>
| <pyramid-converting-ply>

<cascaded-ply> =
  <activateable-piece-losing-tag><stepping>
| \alg{S}<serpent-stepping>
| \alg{H}<shaman-stepping>
| <pawn-promotion-ply>
| <passive-piece><stepping>
| <pawn-sacrifice-ply>
\end{alltt}

\clearpage % ..........................................................

\begin{alltt}
<trance-journey-init> =
  \alg{H}<stepping-no-side-effects>

<journey-start> =
  [<disambiguation>][\alg{\textbackslash}<field>]

<trance-journey> =
  <trance-journey-init>\alg{@H}<journey-start>
  \{<step>
    [(\alg{<}[<disposable-piece-losing-tag>]<field>
      | \alg{*}[<disposable-piece-losing-tag>)]\}
| <trance-journey-init>\alg{@H}
    [<disambiguation>]<field>
| <trance-journey-init>\alg{@@}
  [<disposable-piece-losing-tag>[<field>]
    [\{,<disposable-piece-losing-tag>
      [<field>]\}]]
| <trance-journey-init>\alg{@@@}[\alg{H}]

<sense-journey-init> =
  (\alg{I}|\alg{H})<stepping-no-side-effects>
  \alg{~I}<stepping-no-side-effects>

<sense-journey> =
  <sense-journey-init>\alg{"}
  <activateable-piece-losing-tag>
  <journey-start>
  <stepping-no-side-effects>
| <sense-journey-init>\alg{'}
  [<activateable-piece-losing-tag>]
\end{alltt}

\clearpage % ..........................................................

\begin{alltt}
<star-movement-ply> =
  \alg{I}<stepping-no-side-effects>
    [<resurrecting-side-effect>]
    \alg{~}\alg{T}<stepping-no-side-effects>

<monolith-stepping> =
  <stepping-no-side-effects>
| [<disambiguation>]
  \{<step>
    [<transparency-starchild-side-effect>]\}
  (<step> | \alg{-}<field>)

<monolith-ply> =
  \alg{M}<monolith-stepping>
  [<demoting-side-effect>]

<king-castling-ply> =
  \alg{K}(<file> | <stepping-no-side-effects>)
  [<castling-side-effect>]

<cascade> =
  <stand-alone-ply>
| <starting-ply>
  [\{<cascading-plies>\alg{~}<cascaded-ply>\}]
  [<cascading-plies>
    [\alg{~}(<terminating-ply>
      | <star-movement-ply>)]]
| [<starting-ply>
    [\{<cascading-plies>\alg{~}<cascaded-ply>\}]
    <cascading-plies>\alg{~}]
  (<trance-journey> | <sense-journey>)
| <star-movement-ply>
| <monolith-ply>
| <king-castling-ply>
\end{alltt}

\clearpage % ..........................................................

\begin{alltt}
<status> = ()
         | [\alg{+}]
         | [\alg{+}]\alg{(=)}
         | [\alg{+}]\alg{()}
         | \alg{#}
         | \algcty{++}
         | \alg{(=)}
         | \alg{()}

<compatibility-capture> =
  <classic-piece>[<disambiguation>][\algcty{x}]
    <field>[<status>]
| <pawn>[<disambiguation>][\algcty{x}]<field>
    <promoting-side-effect>[<status>]
| <file>[\algcty{x}]<field>[<status>][\algcty{\_e.p.}]

<compatibility-castling> =
  \algcty{O-O-O}
| \algcty{0-0-0}
| \algcty{O-O}
| \algcty{0-0}

<move> =
  <cascade><status>
| <compatibility-capture>
| <compatibility-castling>
| \alg{#}
| \alg{##}
| \alg{(==)}
| \alg{(===)}
\end{alltt}

\clearpage % ..........................................................

Notational grammar isn't exact, some things are difficult to formalize, some
are too cumbersome. For instance, any ply can be gathered in \alg{[ ]} (square
brackets); adding them to all ply definitions would make grammar significantly
more complex. Also, compatibility notation for capture, castling, and checkmate
is valid only for Classical Chess, and nowhere else.

Another example, Shaman stepping between divergences consists of ordinary steps
interspersed with transparency, or it's all capture-steps. However, all three
side-effects (diverging, transparency, and capturing) are always optional to
write down, and so the two sets of different steps can be indistinguishable.

Additionally, grammar does not have access to an external context. So, defining
entity and values which are not optional, but might not be present is not really
possible. Move status is one such example; except for checks, all other values
are mandatory to write, but most of the time there is no status to report. So,
move status could be optimized out as e.g. an optional entity; while technically
true, this would not be entirely correct.

\clearpage % ..........................................................
% ---------------------------------------------------- Appendix chapter
