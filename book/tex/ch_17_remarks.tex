
% Copyright (c) 2015 - 2021 Mario Mlačak, mmlacak@gmail.com
% Licensed and published as Public Domain work.

% Remarks chapter -----------------------------------------------------
\chapter*{Remarks}
\addcontentsline{toc}{chapter}{Remarks}
\label{ch:Remarks}

If making own chessboard, small markings can be placed onto these locations, to ease setting
up pieces before match.

\section*{Well defined game}
\addcontentsline{toc}{section}{Well defined game}
\label{sec:Remarks/Well defined game}

Well defined game is such where all information pertainable to a game
is plainly visible on a board. Chess in its origin is very close to
that goal, with the exceptions being castling, and turn.

\subsection*{Chips}
\addcontentsline{toc}{subsection}{Chips}
\label{sec:Remarks/Chips}
Chips ...
Coins could be used insead of chips.

\subsubsection*{Castling-chip}
\addcontentsline{toc}{subsubsection}{Castling-chip}
\label{sec:Remarks/Chips/Castling-chip}
...

\subsubsection*{Turn-chip}
\addcontentsline{toc}{subsubsection}{Turn-chip}
\label{sec:Remarks/Chips/Turn-chip}
Turn-chip, also Zed, in algebraic notation Z, is a single chip used to
denote which player's turn is ongoing. It's used in positional notation,
where color of, otherwise empty, field occupied by Zed denotes which
player "has the move".

\subsubsection*{Promoting-chip}
\addcontentsline{toc}{subsubsection}{Promoting-chip}
\label{sec:Remarks/Chips/Promoting-chip}
...

\section*{Non-movement rules}
\addcontentsline{toc}{section}{Non-movement rules}
\label{sec:Remarks/Non-movement rules}

50-move rule, ...

Destination filed == starting field, ...

...

\clearpage % ..........................................................
% ----------------------------------------------------- Remarks chapter
