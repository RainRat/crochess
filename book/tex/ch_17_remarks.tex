
% Copyright (c) 2015 - 2021 Mario Mlačak, mmlacak@gmail.com
% Published as Public Domain work, under CC0 1.0 Universal Public Domain Dedication. See LICENSING, COPYING files for details.

% Remarks chapter -----------------------------------------------------
\chapter*{Remarks}
\addcontentsline{toc}{chapter}{Remarks}
\label{ch:Remarks}

This chapter contains some suggestions to improve gameplay, chessboard
designs.

\section*{Well-defined game}
\addcontentsline{toc}{section}{Well-defined game}
\label{sec:Remarks/Well-defined game}

Well-defined game is one where all information related to game is plainly
visible on a board. Chess in its origin is very close to that goal, with
the exceptions being ability of pieces to castle, rush, and notation for
turn; later, tag for promotion is added to the mix. Pawn-sacrifice tag
does not belong to this list, because it has to be used in the very same
move in which it's obtained.

\subsection*{Chips}
\addcontentsline{toc}{subsection}{Chips}
\label{sec:Remarks/Well-defined game/Chips}

Chip is device, similar in appearance to poker chip, which can be put
underneath a piece to denote its status. For instance, yellow chip can be
put under Pawn to denote its inherited ability to rush. When that Pawn is
moved (or captured) its chip is removed from chessboard.

Similarly, if Pawn is tagged for promotion, e.g. red chip is placed
underneath it, which is removed from chessboard when that Pawn gets promoted,
moved, captured, or converted.

For castling, nominally 3 chips has to be used, 2 for Rooks and 1 for King.
It's enough if just Rooks have their chips, if King ever moves, both Rooks
would lose their chips.

Chip for denoting turn is different, it is placed on an empty field in the
same color to the player which turn is ongoing. This is meant more for
readers to have indicated which player is to play, on a chessboard positions
printed in books, magazines, etc.

In casual games coins or small paper clips could be used instead of chips.

\subsection*{Chessboard}
\addcontentsline{toc}{subsection}{Chessboard}
\label{sec:Remarks/Well-defined game/Chessboard}

Small markings can be placed onto initial positions of Scouts, Monoliths
or whole set of pieces, to ease setting up pieces before match.

Due to chessboard being relatively large in later variants, it might help to
write \algfmt{AN} position onto each field, twice, each oriented towards one
player's seat, to speed-up finding positions.

\section*{Classical Chess, expanded}
\addcontentsline{toc}{section}{Classical Chess, expanded}
\label{sec:Remarks/Classical Chess, expanded}

\TODO

\subsection*{Classical Chess 14}
\addcontentsline{toc}{subsection}{Classical Chess 14}
\label{sec:Remarks/Classical Chess, expanded/Classical Chess 14}

\TODO

\subsection*{Classical Chess 20}
\addcontentsline{toc}{subsection}{Classical Chess 20}
\label{sec:Remarks/Classical Chess, expanded/Classical Chess 20}

\TODO

\subsection*{Classical Chess 26}
\addcontentsline{toc}{subsection}{Classical Chess 26}
\label{sec:Remarks/Classical Chess, expanded/Classical Chess 26}

\TODO

\clearpage % ..........................................................
% ----------------------------------------------------- Remarks chapter
