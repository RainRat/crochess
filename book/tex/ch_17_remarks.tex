
% Copyright (c) 2015 - 2021 Mario Mlačak, mmlacak@gmail.com
% Licensed and published as Public Domain work.

% Remarks chapter -----------------------------------------------------
\chapter*{Remarks}
\addcontentsline{toc}{chapter}{Remarks}
\label{ch:Remarks}

This chapter contains some suggestions to improve gameplay, chessboard
designs.

\section*{Well defined game}
\addcontentsline{toc}{section}{Well defined game}
\label{sec:Remarks/Well defined game}

Well defined game is one where all information related to game is plainly
visible on a board. Chess in its origin is very close to that goal, with
the exceptions being ability of pieces to castle, rush, and notation for
turn; later, tag for promotion is added to the mix. Pawn-sacrifice tag
does not belong to this list, because it has to be used in the very same
move in which it's obtained.

\subsection*{Chips}
\addcontentsline{toc}{subsection}{Chips}
\label{sec:Remarks/Chips}

Chip is device, similar in appearance to poker chip, only smaller so it fits
onto individual fields, with some margin. Chip can be put underneath a piece
to denote its status, and it's held to a piece it belongs to by e.g. magnets.

For instance, yellow chip can be put under Pawn to denote its inherited ability
to rush. When that Pawn is moved (or captured) its chip moves with it due to
magnets, and then is removed from piece (and chessboard) by hand.

Similarly, if Pawn is tagged for promotion, e.g. red chip is placed underneath
it, which is removed from chessboard when that Pawn gets promoted, moved or
captured.

For castling, nominally 3 chips (for instance, green) has to be used, 2 for
Rooks and 1 for King. It's enough if just Rooks have their chips, if King ever
moves, both Rooks would lose their chips.

Chip (for instance, blue) for denoting turn is different, it is placed on an
empty field in the same color to the player which turn is ongoing. This is
meant more for readers to have indicated which player is to play, on a
chessboard positions printed in books, magazines, etc.

All chips should be circular, and the same size, large enough to accomodate
Pawns, Rooks, and Kings. All chips should have raised and lowered border, so
that their cross-section resembles letter \algfmt{I}, when placed upright.
All fields should have thin circular cut-out (only perimeter!), into which
all chips fit tightly enough to stay in place during gameplay, but loose
enough so normal movement of pieces is not affected.

Yellow and red chips can be combined into a single chip, because Pawn to gain
promotion opportunity has to lose rushing ability first. This could be done by
e.g. painting 8 dashes of the same length on a chip rim alternating in yellow
and red. Protective cover over chip rim could expose only one color through
small windows, by turning that cover would change exposed color of a chip.

In casual games coins or small paper clips could be used instead of chips.

\subsection*{Chessboard}
\addcontentsline{toc}{subsection}{Chessboard}
\label{sec:Remarks/Chessboard}

Small markings can be placed onto initial positions of pieces, to ease setting
up pieces before match. These markings can be placed inside circular cut-outs
made for chips.

Due to chessboard being relatively large in later variants, to speed-up finding
positions, it might help to write \algfmt{AN} position onto each field, twice,
each oriented towards one player's seat. Those \algfmt{AN} positions should be
placed outside of a circular cut-outs for chips, preferrably in to the corners
of each field.

\clearpage % ..........................................................
% ----------------------------------------------------- Remarks chapter
